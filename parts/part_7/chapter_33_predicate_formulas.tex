\chapter{Predicate Formulas}
\label{chapter:predicate-formulas}
\marginurl{%
  Predicate Formulas:\\\noindent
  Introduction to Mathematical Logic \#5
}{youtu.be/yb9NvmXyFfg}

In the previous chapters we studied propositional logic. But in real mathematics
there are many non propositional formulas. For example, we may wish to prove
that if a relation $R$ on $M$ is transitive, then
\[
  (R(w, x) \land R(x, y) \land R(y, z)) \implies R(w, z)
\]
is true for any $w, x, y, z \in M$. This chapter defines a logical
system that allows us to formally prove such statements.

Let us write the previous statement in a formula-like form:
\begin{multline*}
    \overbrace{
        \left(
          \forall x, y, z \in M ~ (R(x, y) \land R(y, z)) \implies R(x, z)
         \right)
    }^{R \text{ is transitive}} \implies \\
    \underbrace{(
        \forall w, x, y, z \in M ~ (R(w, x) \land R(x, y) \land R(y, z))
        \implies
        R(x, z)
    )}_\text{the desired conclusion}.
\end{multline*}
Note that there are several things we need to explain if we wish to define
formally formulas like this:
\begin{itemize}
    \item we need to explain what kind of sets we can use (in this case we need
        to define $M$),
    \item we need to explain what kind of relations we can use (in this case we
        need to define $R$),
\end{itemize}

Another example of a statement we may wish to prove is saying that if
$f : M \to M$ is an inverse of itself (i.e. $f(f(x)) = f(x)$ for any $x \in M$),
then $f(f(f(x))) = f(x)$ for any $x \in M$; more formally, we may wish to prove
the statement
\[
    \underbrace{
        (\forall x \in M ~ f(f(x)) = x)}_{f \text{is an inverse of itself}}
    \implies
    \underbrace{
        (\forall x \in M ~ f(f(f(x))) = f(x))
    }_\text{the desired conclusion}.
\]
In order to explain what we mean by such formulas
\begin{itemize}
  \item we need to explain what kind functions we can use (in this case we need
    to define $f$).
\end{itemize}

\paragraph{Signature.}
In predicate logic, formula uses just symbols for all these objects. We specify
these symbols only when we wish to compute actual truth value of the formula.
We also assume that all the quantifiers are over the same set so we do not need
a symbol for the set $M$.

Signature is the way to define the list of all these symbols, it consists of
three objects:
\begin{itemize}
  \item a set (possibly empty) of symbols for relations,
  \item a set (possibly empty) of symbols for functions,
  \item arities of these functions and relations (i.e. how many arguments they
    may take).
\end{itemize}
An example of a signature is a triple $(\set{\text{``R''}}, \set{\text{``f''}},
a)$, where
\[
  a(s) = \begin{cases}
    2 & \text{if } s =  \text{``R''} \\
    1 & \text{if } s =  \text{``f''}
  \end{cases}.
\]
This signature is enough to define the formulas we discussed. Now we are ready
to define the predicate formulas.

\begin{definition}
  Let $\signature{S} = (S_\mathrm{rel}, S_\mathrm{fun}, a)$ be a signature and $V$
  be some set.

  We say that $t$ is a \emph{term} in the signature $\signature{S}$ over the variables
  from $V$ if
  \begin{itemize}
    \item either $t$ is equal to $x$ for $x \in V$,
    \item or $t$ is equal to $f(t_1, \dots, t_{a(f)})$, where
      $f \in S_\mathrm{fun}$ and $t_1$, \dots, $t_{a(f)}$ are terms in the
      signature $\mathcal{S}$ over the variables from $V$.
  \end{itemize}

  We say that $\phi$ is a \emph{predicate formula} in the signature
  $\signature{S}$ over the variables from $V$ if
  \begin{itemize}
    \item either $\phi$ is equal to $R(t_1, \dots, t_{a(R)})$, where
      $R \in S_\mathrm{rel}$ and $t_1$, \dots, $t_{a(R)}$ are terms in the
      signature $\signature{S}$ over the variables from $V$,
    \item or $\phi$ is equal to $(\psi_1 \land \psi_2)$, or
      $(\psi_1 \lor \psi_2)$, or $(\psi_1 \limplies \psi_2)$, where $\psi_1$ and
      $\psi_2$ are predicate formulas in the signature $\signature{S}$ over the
      variables from $V$,
    \item or $\phi$ is equal to $(\lnot \psi)$, where $\psi$ is a predicate
      formula in the signature $\signature{S}$ over the variables from $V$,
    \item or $\phi$ is equal to $\left( \exists x_i~\psi \right)$ or 
      $\left( \forall x_i~\psi \right)$ where
      $\psi$ is a predicate formula in the signature $\signature{S}$ over the
      variables from $V$.
  \end{itemize}

  We denote by $\pred{V}{\signature{S}}$ the set of all predicate formulas in
  $\signature{S}$ over the variables from $V$; we also denote by
  $\term{V}{\signature{S}}$ the set of all terms in $\signature{S}$ over the
  variables from $V$.
\end{definition}

In order to compute the truth value of a predicate formula, we need to specify
the values of all the free variables and all the symbols from the signature.
The specification of the symbols from the signature is called structure; i.e.
a structure for a signature 
$\signature{S} = (S_\mathrm{rel}, S_\mathrm{fun}, a)$
is a triple $\structure{M} = (M, F_\mathrm{rel}, F_\mathrm{fun})$ such that
\begin{itemize}
  \item $F_\mathrm{rel}$ assigns a relation to every relation symbol in
    $\signature{S}$; i.e., 
    $F_\mathrm{rel} : S_\mathrm{rel} \to \bigcup_{i = 0}^\infty 2^{M^i}$
    such that $F_\mathrm{rel}(R) \in 2^{M^{a(R)}}$ and
  \item $F_\mathrm{fun}$ assignes a function to every function symbol in
    $\signature{S}$; i.e.,
    $F_\mathrm{fun} : S_\mathrm{fun} \to \bigcup_{i = 0}^\infty M^{M^i}$
    such that $F_\mathrm{fun}(f) \in M^{M^{a(f)}}$.
\end{itemize}
The set $M$ in the structure is called the \emph{domain} of the structure and
denoted by $\domain{\structure{M}}$. We also denote $F_\mathrm{rel}(R)$ by
$R^{\structure{M}}$ and $F_\mathrm{fun}(f)$ by $f^{\structure{M}}$.


\begin{definition}
  Let $\signature{S} = (S_\mathrm{rel}, S_\mathrm{fun}, a)$ be a signature and
  $\structure{M}$ be a structure for $\signature{S}$.

  Let $\rho : V \to \domain{\structure{M}}$.
  The valuation of terms in $\signature{S}$ over the variables from $V$
  corresponding to the assignement $\rho$ is the function 
  $\substitute{\cdot}{\rho, \structure{M}} : 
    \term{V}{\signature{S}} \to \domain{\structure{M}}$ such that
  \begin{itemize}
    \item $\substitute{x}{\rho, \structure{M}} = \rho(x)$ for any $x \in V$,
    \item $\substitute{f(t_1, \dots, t_\ell)}{\rho, \structure{M}} = 
      f^{\structure{M}}(
        \substitute{t_1}{\rho, \structure{M}},
        \dots,
        \substitute{t_\ell}{\rho, \structure{M}}
      )$ \,
      for any function symbol $f \in S_\mathrm{fun}$ and 
      terms $t_1, \dots, t_\ell \in \term{V}{\signature{S}}$.
  \end{itemize}

  Similarly we may define valuations of predicate formulas (we will abuse the
  notation and denote them by $\substitute{\cdot}{\rho, \structure{M}}$ too).
  The valuation of predicate formulas in $\signature{S}$ over the variables from
  $V$ corresponding to the assignement $\rho$ is the function
  $\substitute{\cdot}{\rho, \structure{M}} :
    \pred{V}{\signature{S}} \to \domain{\structure{M}}$ such that
  \begin{itemize}
    \item $\substitute{R(t_1, \dots, t_\ell)}{\rho, \structure{M}} = 
      R^{\structure{M}}(
        \substitute{t_1}{\rho, \structure{M}}),
        \dots,
        \substitute{t_\ell}{\rho, \structure{M}})
      $\,
      for any relation symbol $R \in S_\mathrm{rel}$ 
      and terms $t_1, \dots, t_\ell \in \term{V}{\signature{S}}$,
    \item 
      $
        \substitute{\psi_1 \land \psi_2}{\rho, \structure{M}} = 
        \substitute{\psi_1}{\rho, \structure{M}} \land
        \substitute{\psi_2}{\rho, \structure{M}}
      $\,
      for any $\psi_1, \psi_2 \in \pred{V}{\signature{S}}$,
    \item 
      $
        \substitute{\psi_1 \lor \psi_2}{\rho, \structure{M}} = 
        \substitute{\psi_1}{\rho, \structure{M}} \lor
        \substitute{\psi_2}{\rho, \structure{M}}
      $\,
      for any $\psi_1, \psi_2 \in \pred{V}{\signature{S}}$,
    \item 
      $
        \substitute{\psi_1 \limplies \psi_2}{\rho, \structure{M}} = 
        \substitute{\psi_1}{\rho, \structure{M}} \limplies
        \substitute{\psi_2}{\rho, \structure{M}}
      $\,
      for any $\psi_1, \psi_2 \in \pred{V}{\signature{S}}$,
    \item 
      $
        \substitute{\lnot \psi}{\rho, \structure{M}} = 
        \lnot \substitute{\psi_1}{\rho, \structure{M}}
      $\,
      for any $\psi \in \pred{V}{\signature{S}}$,
    \item for $\psi \in \pred{V}{\signature{S}}$,
      $
        \substitute{\exists x \  \psi}{\rho, \structure{M}} = 
        \ltrue  
      $ iff 
      $\substitute{\psi}{\assign{\rho}{x}{v}, \structure{M}} = \ltrue$
      for some $v \in \domain{\structure{M}}$, and
    \item for $\psi \in \pred{V}{\signature{S}}$,
      $
        \substitute{\forall x \  \psi}{\rho, \structure{M}} = 
        \ltrue  
      $ iff 
      $\substitute{\psi}{\assign{\rho}{x}{v}, \structure{M}} = \ltrue$
      for all $v \in \domain{\structure{M}}$.
  \end{itemize}

  Let $V = \set{x_1, \dots, x_n}$ then we denote the assignement $\rho : V \to
  \domain{\structure{M}}$ such that $\rho(x_i) = v_i$ by 
  $x_1 = v_1, \dots, x_n = v_n$.
\end{definition}

We say that $\structure{M}$ is a model of a formula $\phi$
(written $\structure{M} \models \phi$)\footnote{
  Sometimes ``$\structure{M}$ is a model of $\phi$'' is written
  as $\models_{\structure{M}} \phi$.
} over the variables
from $V$ iff $\substitute{\psi}{\rho, \structure{M}} = \ltrue$ for all $\rho : V
\to \domain{\structure{M}}$.

We also say that $\phi$ is true in $\structure{M}$ if $\structure{M} \models \phi$,
and we say that $\phi$ is false in $\structure{M}$ if $\structure{M} \not\models \phi$.

Let us consider an example:
\begin{itemize}
  \item First, we define a signature
    $\signature{S} = (\set{=, <}, \set{+, \cdot}, a)$ (we write 
    $S = (=, <; +, \cdot)$ when arities of the functions and relations are clear
    from the context),
    where $a(x) = 2$ for any $x \in \set{<, =, +, \cdot}$.
  \item After this we define the structure $\structure{M}$ in signature
    $\signature{S}$ such that
    \begin{gather*}
      f^{\structure{M}}(x, y) = \begin{cases}
        x \cdot y & \text{if } f \text{ is } \cdot \\
        x + y & \text{if } f \text{ is } +
      \end{cases} \\
      \text{and} \\
      R^{\structure{M}}(x, y) \begin{cases}
        x = y & \text{if } R \text{ is } = \\
        x < y & \text{if } R \text{ is } <
      \end{cases}
    \end{gather*}
    Note that such a definition is pretty cumbersome, especially considering the
    fact that we use standard $+$ instead of the symbol $+$,
    standard $=$ instead of the symbol $=$ etc. So in similar cases we
    write $\structure{M} = (\R; =, <; +, \cdot)$.
  \item Finally, we consider the formulas in $\signature{S}$ over the set of
    variables $\set{x, y, z}$:
    \begin{gather*}
      \forall x \ \forall y \ x + y = y + x \\
      \text{and} \\
      \forall x \ \forall y \ \forall z \ (x < y \implies x + z < y + z).
    \end{gather*}
    (Note that we write $a = b$ instead of $=(a, b)$ and $a + b$ instead of
    $a + b$, this is a common notation when the standard mathematical
    operations and relations are used in the signature.)
\end{itemize}
The first formula says that addition is commutative, which is true in $\R$, so
the value of the formula is true in $\structure{M}$. 
(Note that we do not mention the values of the variables $x$ and $y$
since both of them are not free.) Indeed, consider $a, b \in \R$, note that
$\substitute{x + y}{x = a, y = b, \structure{M}} = a + b = b + a = 
  \substitute{y + x}{x = a, y = b, \structure{M}}$. Hence, 
$\substitute{\forall x \ \forall y \ x + y = y + x}{\rho, \structure{M}} = \ltrue$
for all $\rho$.

The second formula says that the inequalities are additive, so it should be also
true with respect to the structure $\structure{M}$.

\begin{chapterendexercises}
  \exercise
    Show that the second formula is true in $\structure{M}$.
  \exercise
    Let us consider a signature $(=; +, \cdot, 0, 1)$ and two models with this
    signature: $\structure{R} = (\R; =; +, \cdot, 0, 1)$, and
    $\structure{Q} = (\Q; =; +, \cdot, 0, 1)$.
    Find a predicate formula $\phi$ in this signature such that
    $\structure{R} \models \phi$ but $\structure{Q} \not\models \phi$.
\end{chapterendexercises}

