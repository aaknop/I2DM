\chapter{Fixed Point Theorem}
Probbaly one of the most surprinsg phenomenon in the world of esoteric 
programming is existence of quines, the programs that print themself.\marginurl{%
Probably the most impresive example of a quine is Quine Relay, a Ruby program
  that generates Rust program that generates Scala program that generates
  \dots (through 128 languages in total)\dots REXX program that generates the
  original Ruby code again.
}{github.com/mame/quine-relay}
This chapter proves existence of such programs in almost all programming
languages.

\begin{theorem}[Kleene's Fixed Point Theorem]
\label{theorem:kleene}
  Let $h : \N \to \N$ be a total computbale function, and let $U : \N^2 \to \N$
  be a G\"odel universal function. Then there is $n \in \N$ such that $U_n$ is
  equal to $U_{h(n)}$.
\end{theorem}
\begin{proof}
  Let $f : \N \to \N$ be a computable function such that no computable function
  $g : \N \to \N$ can differ from $f$ everywhere, such a function exists by
  \Cref{theorem:intersecting-function}. Note that there is a total computable
  function $g : \N \to \N$ such that $U_{f(n)} = U_{g(n)}$ provided that $f(n)$
  is defined. Indeed, let us consider $V(x, y) = U(f(x), y)$; since $U$ is a
  G\"odel universal function, there is a total function $g(n)$ such that
  $V(x, y) = U(g(x), y)$.

  Assume for the sake of contradiction that $U_n \neq U_{h(n)}$ for all $n \in
  \N$. Let $t : \N \to \N$ be a total computable function such that $t(n) =
  h(g(n))$. It is clear that if $f$ is different from $t$ evrywhere, which
  contradictis to the definition of $f$.
\end{proof}

\begin{corollary}
  Let $U : \N^2 \to \N$ be a G\"odel universal function. Then there is $n \in
  \N$ such that $U(n, x) = n$ for all $x \in \N$.
\end{corollary}
\begin{proof}
  Let $q : \N \to \N$ be a computable total function such that $U(q(n), x) = n$
  for all $x \in \N$ (such a function exists since $U$ is a G\"odel universal
  function). Note that there is $n$ such that $U_n$ is equal to $U_{q(n)}$ which
  implies that $U(n, x) = U(q(n), x) = n$ for all $x \in \N$.
\end{proof}

\begin{exercise}
  Prove that there is a program on the programing of your choice that prints its
  text backwards.
\end{exercise}

Potentially, the function $h$ in Kleene's fixed point theorem
(\Cref{theorem:kleene}) may depend on a parameter; however, even in this case
there is a fixed point theorem.
\begin{theorem}
\label{theorem:kleene-parameter}
  Let $h : \N^2 \to \N$ be a total computbale function, and let $U : \N^2 \to \N$
  be a G\"odel universal function. Then there is total computable function $m
  \in \N$ such that $U(h(p, m(p)), x) = U(m(p), x)$ for all $p, x \in \N$
\end{theorem}

\begin{chapterendexercises}
  \exercise Show that there are different $p, q \in \N$ such that $U(p, x) = q$
    and $U(q, x) = p$ for all $x \in \N$.
  \exercise Let $h : \N \to \N$ be a total computbale function, and let 
    $U : \N^2 \to \N$ be a G\"odel universal function. Show that
    there are infinetely many $n \in \N$ such that $U_n = U_{h(n)}$.
  \exercise Prove \Cref{theorem:kleene-parameter}.

\end{chapterendexercises}
