\chapter{Predicates and Connectives}
\label{chapter:predicates}
\marginurl{%
  Connectives and Propositions:\\\noindent
  Introduction to Mathematical Reasoning \#5
}{youtu.be/0unvlq2OTaE}
\section{Propositions and Predicates}
In the previous chapters we used the word ``statement'' without any even
relatively formal definition of what it means. In this chapter we are going to
give a semi-formal definition and discuss how to create complicated statements
from simple statements.

It is difficult to give a formal definition of what a mathematical statement is,
hence, we are not going to do it in this book. The goal of this section is to
enable the reader to recognize mathematical statements.

A \emph{proposition} or a mathematical statement is a declarative sentence
which is either true or false but not both. Consider the following list of
sentences.
\begin{enumerate}
  \item $2 \times 2 = 4$
  \item $\pi = 4$
  \item $n$ is even
  \item 32 is special
  \item The square of any odd number is odd.
  \item The sum of any even number and one is prime.
\end{enumerate}
Of those, the first two are propositions; note that this says nothing about
whether they are true or not. Actually, the first is true and the second is
false. However, the third sentence becomes a proposition only when the value
of $n$ is fixed. The fourth is not a proposition. Finally, the last
two are propositions (the fifth is true and the sixth is false).

The third statement is somewhat special, because there is a simple way to make
it a proposition: one just needs to fix the value of the variables. Such
sentences are called predicates and the variables that need to be specified are
called free variables of these predicates.

Note that the fourth sentence is also interesting, since if we define what it
means to be special, the phrase became a proposition.
Mathematicians tend to do such things to give mathematical meanings to
everyday words.

\section{Connectives}

Mathematicians often need to decide whether a given proposition is true or
false. Many statements are complicated and constructed from simpler statements
using \emph{logical connectives}. For example we may consider the following
statements:
\begin{enumerate}
  \item $3 > 4$ and $1 < 1$;
  \item $1 \times 2 = 5$ or $6 > 1$.
\end{enumerate}

\paragraph{Logical connective ``OR''.}
The second statement is an example of usage of this connective. The statement
``P or Q'' is true if and only if at least one of P and Q is true. We may
define the connective using the truth table of it.
\begin{center}
    \begin{tabular}{l l l}
        \toprule
        P & Q & P or Q \\
        \midrule
        T & T & T \\
        T & F & T \\
        F & T & T \\
        F & F & F \\
        \bottomrule
  \end{tabular}
\end{center}

The or connective is also called \emph{disjunction} and the disjunction of $P$
and $Q$ is often dented as $P \lor Q$.

\nomenclature[L]{$P \lor Q$}{denotes the statement saying that at least one of
$P$ and $Q$ is true}

\begin{warning}
  Note that in everyday speech ``or'' is often used in the exclusive case, like
  in the sentence ``we need to decide whether it is an insect or a spider''.
  In this case the precise meaning of ``or'' is made clear by the context.
  However, mathematical language should be formal, hence, we always use ``or''
  inclusively.
\end{warning}

\paragraph{Logical connective ``AND''.}
The first statement is an example of this connective. The statement ``P and Q''
is true if and only if both P and Q are true. We may define the
connective using the truth table of it.
\begin{center}
  \begin{tabular}{l l l}
      \toprule
      P & Q & P and Q \\
      \midrule
      T & T & T \\
      T & F & F \\
      F & T & F \\
      F & F & F \\
      \bottomrule
  \end{tabular}
\end{center}

The and connective is also called \emph{conjunction} and the conjunction of
$P$ and $Q$ is often dented as $P \land Q$.

\nomenclature[L]{$P \land Q$}{denotes the statement saying that $P$ and $Q$
are both true}

\begin{warning}
  Not all the properties of ``and'' from everyday speech are captured by
  logical conjunction. For example, ``and'' sometimes implies order. For
  example, ``They got married and had a child'' in common language means that
  the marriage came before the child. The word ``and'' can also imply a
  partition of a thing into parts, as ``The American flag is red, white, and
  blue.'' Here it is not meant that the flag is at once red, white, and blue,
  but rather that it has a part of each color.
\end{warning}

\paragraph{Logical connective ``NOT''.}
The last connective is called \emph{negation} and examples of usage of it are
the following:
\begin{enumerate}
  \item 5 is not greater than 8;
  \item Does not exist an integer $n$ such that $n^2 = 2$.
\end{enumerate}

Note that it is not straightforward where to put the negation in these
sentences.

The negation of a statement $P$ is denoted as $\lnot P$ (sometimes it is also
denoted as $\sim P$).
\nomenclature[L]{$\lnot P$}{denotes the statement saying that $P$ is false}



\begin{chapterendexercises}
  \exercise Construct truth tables for the statements
    \begin{itemize}
      \item not ($P$ and $Q$);
      \item (not $P$) or (not $Q$);
      \item $P$ and (not $Q$);
      \item (not $P$) or $Q$;
    \end{itemize}
  \exercise[recommended] Consider the statement ``All gnomes like cookies''. Which of
    the following statements is the negation of the above statement?
    \begin{itemize}
      \item All gnomes hate cookies.
      \item All gnomes do not like cookies.
      \item Some gnomes do not like cookies.
      \item Some gnomes hate cookies.
      \item All creatures who like cookies are gnomes.
      \item All creatures who do not like cookies are not gnomes.
    \end{itemize}
  \exercise Using truth tables show that the following statements are
    equivalent:
    \begin{itemize}
      \item $P \implies Q$,
      \item $(P \lor Q) \iff Q$
        ($A \iff B$ is the same as $(A \implies B) \land (B \implies A)$),
      \item $(P \land Q) \iff P$
    \end{itemize}
  \exercise Prove that three connectives ``or'', ``and'', and ``not'' can
    all be written in terms of the single connective ``notand'' where ``$P$
    notand $Q$'' is interpreted as ``not ($P$ and $Q$)'' (this operation is
    also known as Sheffer stroke or NAND).
  \exercise Show the same statement about the connective ``notor'' where
    ``$P$ notor $Q$'' is interpreted as ``not ($P$ or $Q$)'' (this operation is
    also known as Peirce's arrow or NOR).
\end{chapterendexercises}
