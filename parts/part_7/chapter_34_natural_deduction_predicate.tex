\chapter{Natural Deduction}
\marginurl{%
  Natural Deduction for Predicate Logic:\\\noindent
  Introduction to Mathematical Logic \#6
}{youtu.be/GVht3ES2qqo}
By analogy with the tautology, in the predicate logic we wish to prove that a
formula is true, whenever the structure and the values of the variables we
choose. Such formulas are called \textit{logically valid}.

In addition, we may define semantic implication for predicate formulas.
We say that a set of predicate formulas $\Sigma$ in a signature $\mathcal{S}$
semantically implies a formula $\phi$ ($\Sigma \models \phi$) in the signature
iff any structure with the signature $\mathcal{S}$ modeling $\Sigma$ models
$\phi$ as well.

Natural deduction for the predicate formulas is defined in the same manner as
the natural deduction for the propositional formulas but now the lines are
predicate formulas and we can use four additional rules.

\paragraph{Universal quantifier.}
The first logically-valid formula we use as a rule is
$A(x) \implies (\forall y \ A(y))$,
this rule allows us to introduce a universal quantifier.
In order to use the following rule, $x$ should not be a free variable of
an open hypothesis.
\[
  \begin{nd}
    \have [m] {1} {A(x)}
    \have [~] {2} {\forall y \  A(y)} \Ai{1}
  \end{nd}
\]
The second logically-valid formula we use as a rule says that if a statement is
true for all the values of a variable, then it is also true when you substitute
some specific term instead of the variable, i.e. $(\forall x \ A(x)) \implies
A(t))$, this rule allows us to eliminate an universal quantifier.
\[
  \begin{nd}
    \have [m] {1} {\forall x \  A(x)}
    \have [~] {2} {A(t)} \Ae{1}
  \end{nd}
\]

\paragraph{Existential quantifier.}
The first formula for the exisential quantifier says that you can name any term
in the formula by a variable and formula is still true for some value of the
variable. The corresponding formula is $A(t) \implies (\exists x \ A(x))$.
\[
  \begin{nd}
    \have [m] {1} {A(t)}
    \have [~] {2} {\exists x \ A(x)} \Ei{1}
  \end{nd}
\]

The last rule says that if $A(x)$ is true for some $x$ and we know that $A(y)$
implies $B$, then we can derive $B$ (note that this is true only when
$y$ is not used in $B$). Thus we can apply the following rule when $y$ is not
be a free variable neither of $B$ nor of any open hypothesis.
\[
  \begin{nd}
    \have [m] {1} {\exists x \ A(x)}
    \open
      \hypo [i] {2} {A(y)}
      \have[j] {3} {B}
    \close
    \have [~] {4} {B} \Ee{1, 2-3}
  \end{nd}
\]
\section{Examples of Derivations}
First example $\forall x \ F(x) \lor \lnot(\forall x \ F(x))$ is a special form
of the law of excluded middle, which we proved
in the previous chapter. However, in order to emphasize that the propositional
logic can prove all the statements provable in the predicate case we present
the proof of this statement as well.

\noindent $
  \begin{nd}
    \hypo {1} {}
    \open
      \hypo {2} {\lnot (\forall x \  F(x) \lor \lnot (\forall x \  F(x)))}
      \open
        \hypo {3} {\forall x \  F(x)}
        \have {4} {\forall x \  F(x) \lor \lnot (\forall x \  F(x))} \oi{3}
        \have {5} {\perp} \ne{2, 4}
      \close
      \have {6} {\lnot (\forall x \  F(x))} \ni{3-5}
      \have {7} {\forall x \  F(x) \lor \lnot (\forall x \  F(x))} \oi{6}
      \have {8} {\perp} \ne{2, 8}
    \close
    \have {9} {\forall x \  F(x) \lor \lnot (\forall x \  F(x))} \by{IP}{2-8}
  \end{nd}
$

Unfortunately, this example just shows that a statement provable in the
propositional logic can be proven in the predicate logic. The next example is
an example that cannot be expressed in the propositional logic, we
prove that if we know that
$\forall x \forall y \ R(x, y) \implies R(y, x)$, the we can derive
$\forall x \forall y \ ((R(x, y) \implies R(y, x)) \land
  (R(y, x) \implies R(x, y)))$.


\noindent $
  \begin{nd}
    \hypo {1} {\forall x \forall y \ R(x, y) \implies R(y, x)}

    \have {2} {\forall y \ R(x', y) \implies R(y, x')} \Ae{1}
    \have {3} {R(x', y') \implies R(y', x')} \Ae{2}
    \have {4} {\forall y \ R(y', y) \implies R(y, y')} \Ae{1}
    \have {5} {R(y', x') \implies R(x', y')} \Ae{4}
    \have {6} {(R(x', y') \implies R(y', x')) \land R(y', x') \implies R(x', y')}
              \ai{3, 5}
    \have {6} {\forall y \ (R(x', y) \implies R(y, x')) \land
      (R(y, x') \implies R(x', y))}
              \Ai{6}
    \have {6} {\forall x \forall y \ (R(x, y) \implies R(y, x)) \land
      (R(y, x) \implies R(x, y))}
              \Ai{7}
  \end{nd}
$

\section{Soundness and Completeness}
Like in the propositional case, the most important properties of the natural
deduction are the following two theorems.

\begin{theorem}[completeness of natural deductions, G\"odel]
    Let $\phi$ be a predicate formula. If $\phi$ is logically valid, then
    there is a proof of $\phi$. Moreover, if $\Sigma \models \phi$,
    for some finite set of predicate formulas $\Sigma$, then there is a
    derivation of $\phi$ from $\Sigma$.
\end{theorem}

\begin{theorem}[soundness of natural deductions]
    Let $\phi$ be a predicate formula. If there is a proof of $\phi$, then
    $\phi$ is logically valid. Moreover, if there is a derivation of $\phi$ from
    $\Sigma$, for some finite set of predicate formulas $\Sigma$, then
    $\Sigma \models \phi$.
\end{theorem}

\begin{chapterendexercises}
  \exercise Give a natural deduction derivation of
    $\forall x \ A(x) \implies \forall x \ B(x)$ from
    $\forall x \ (A(x) \implies B(x))$.
  \exercise Give a natural deduction derivation of
    $\exists x \ ( A(x) \lor B(x))$ from
    $\exists x \  A(x) \lor \exists x \  B(x)$.
\end{chapterendexercises}
