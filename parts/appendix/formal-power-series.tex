\chapter{Formal Power Series}
\label{chapter:formal-power-series}
Formal power series is an algebraic analogy of power series from analysis.
A formal power series is something like
$a_0 + x a_1 + x^2 a_2 + \dots a_n x^n + \dots$; to describe such an object it
is enough to define the sequence $\set{a_n}_{n \ge 0}$ since $x$ is a variable.
\begin{definition}
  We say that $F(x)$ is a \emph{formal power series} in the variable $x$, if
  $F(x) = \set{f_n}_{n \ge 0}$. To distinguish between formal power series and
  sequences, we write formal power series as $\sum_{n \ge 0} f_n x^n$.
  We say that $f_n$ is the coefficient of $x^n$ in $F(x)$.

  We say that two formal power series $F(x)$ and $G(x)$ are equal iff for all
  $n \ge  0$, the coefficients of $x^n$ in $F(x)$ and $G(x)$ are the same.

  The set of all the power series in the variable $x$ is denoted as
  $\R[[x]]$.
\end{definition}
\nomenclature[S]{$\R[[x]]$}{denotes the set all the power series in the
variable $x$}

\section{Arithmetic Operations}

We can perform all the standard operations with the formal power series:
\begin{gather*}
  \sum_{n \ge 0} a_n x^n \pm \sum_{n \ge 0} b_n x^n =
  \sum_{n \ge 0} (a_n \pm b_n) x^n, \\
  c \sum_{n \ge 0} a_n x^n =
  \sum_{n \ge 0} (c a_n) x^n,
  \\
  \text{and} \\
  \sum_{n \ge 0} a_n x^n \sum_{n \ge 0} b_n x^n =
  \sum_{n \ge 0} (\sum_{k = 0}^n a_{k} b_{n - k}) x^n.
\end{gather*}

These operations satisfy all the properties we may expect from them.
\begin{theorem}
  Let $F(x)$, $G(x)$, and $H(x)$ be some formal power series. Then the following
  equalities hold:
  \begin{itemize}
    \item $(F(x) + G(x)) + H(x) = F(x) + (G(x) + H(x))$,
    \item $F(x) + G(x) = G(x) + F(x)$,
    \item $(F(x) G(x)) H(x) = F(x) (G(x) H(x))$,
    \item $F(x) G(x) = G(x) F(x)$, and
    \item $(F(x) + G(x)) H(x) = F(x)H(x) + G(x)H(x)$.
  \end{itemize}
\end{theorem}

For example, $(1 - x) (1 + x + x^2 + \dots) = 1$. Thus we can say that the
series $(1 - x)$ has an inverse, and that inverse is equal to $1 + x + x^2 +
\dots$.
\begin{theorem}
  A formal power series $\sum_{n \ge 0} f_n x^n$ has an inverse iff
  $f_0 \neq 0$ and moreover this inverse is unique.
\end{theorem}
\begin{proof}
  Assume that a power series $F(x) = \sum_{n \ge 0} f_n x^n$ has an inverse
  $G(x) = \sum_{n \ge 0} g_n x^n$. In this case $F \cdot G = 1$ i.e.
  $f_0 g_0 = 1$ and $f_0 \neq 0$. Moreover,
  $\sum_{k = 0}^n f_k g_{n - k} = 0$; from which we can conclude that
  \begin{equation}
    \label{equation:inverse-of-formal-power-series}
    g_n = -\frac{1}{f_0} \sum_{k > 0} f_k g_{n - k}.
  \end{equation}
  This determine $g_n$ uniquely, as stated.

  Conversely, if $f_0 \neq 0$, (\ref{equation:inverse-of-formal-power-series})
  determines the sequence $\set{g_n}_{n \ge 0}$.
\end{proof}

\section{Composition}
Another operation we may need to perform is composition; a composition of
the power series $F(x)$ and $G(x)$ is a power series $F(G(x))$; i.e.
$F(G(x)) = \sum_{n \ge 0} a_n G^n(x)$, where $F(x) = \sum_{n \ge 0} a_n x^n$
Note that the composition is well-defined iff the coefficient of $x^0$ in $G(x)$
is $0$ or if $F(x)$ is a polynomial.

\section{Derivative}
Let $F(x) = \sum_{n \ge 0} f_n x^n$ be a formal power series. Then the
derivative $F'(x)$ (we also denote it as $\frac{d}{dx} F(x)$) of $F(x)$ is
equal to
$\sum_{n \ge 1} n f_n x^{n - 1} = \sum_{n \ge 0} (n + 1) f_{n + 1} x^n$.

The derivatives of formal power series satisfy the same properties as
derivatives of functions.
\begin{theorem}
  Let $F(x)$, $G(x)$, and $H(x)$ be some formal power series. Then the following
  equalities hold:
  \begin{itemize}
    \item $\frac{d}{dx}(F(x) + G(x)) = F'(x) + G'(x)$, and
    \item $\frac{d}{dx}(F(x) G(x)) = F'(x)G(x) + F(x)G'(x)$.
  \end{itemize}
\end{theorem}

As a corollary of these statements we can derive a formula for the derivative of
$1 / F(x)$.
\begin{corollary}
  Let $F(x)$ be a formal power series such that $1 / F(x)$ exists. In this
  case $\frac{d}{dx} \frac{1}{F(x)} = -\frac{F'(x)}{F^2(x)}$.
\end{corollary}
\begin{proof}
  Note that $F(x) \frac{1}{F(x)} = 1$. Hence,
  $\frac{d}{dx}(F(x) \frac{1}{F(x)}) = 0$. Using the formula for the derivative
  of a product we may conclude that $F'(x)\frac{1}{F(x)} +
  F(x)\frac{d}{dx}\frac{1}{F(x)} = 0$. As a result,
  $-\frac{F'(x)}{F^2(x)} = \frac{d}{dx} \frac{1}{F(x)}$.
\end{proof}

\begin{remark}
  If $F'(x) = 0$, then $F(x) = a_0$.
\end{remark}

We denote the formal power series $\sum_{n \ge 0} \frac{1}{n!} x^n$ by $e^x$
(since the Taylor series of $e^x$ is equal to
$\sum_{n \ge 0} \frac{1}{n!} x^n$).
\begin{remark}
  If $F'(x) = F(x)$, then $F(x) = c e^x$ for some $c \in \R$.
\end{remark}
