\chapter{Propositional Formulas}
\label{chapter:propositional-formulas}
\marginurl{%
  Propositional Formulas:\\\noindent
  Introduction to Mathematical Logic \#1
}{youtu.be/X0797bVFf3Y}


This part, as it follows from the title, is devoted to mathematical logic,
a mathematical approach to a branch of philosophy called logic. Logic studies
reasoning and mathematical logic studies mathematical reasoning. As we have
mentioned in \Cref{chapter:proofs} proofs in mathematics consists of
\emph{sentences} of a certain structure that are connected by implications.
In addition,  as we discussed in \Cref{chapter:predicates}, we can build larger
sentences from smaller ones using connectives.

Note that in real life the sentences are written using common English which is
ambiguous and therefore hard for analysis.
So to create a formal description of mathematics we need to create an
artificial formal language for mathematics.

First
(\Cref{chapter:propositional-formulas,chapter:propositional-truth,chapter:propositional-deduction}) 
we will define a language for propositional (sentential) logic; i.e. the logic
which deals only with propositions. Later (\Cref{chapter:predicate-formulas}) we
extend it to a logic which also takes properties of individuals into account.

The process of formalization of propositional logic consists of two main parts:
\begin{itemize}
  \item present a formal language,
  \item specify a procedure for obtaining valid or true propositions.
\end{itemize}

\section{Definition of Formulas}

Statements in  propositional logic are either some independent atomic
statements, or are formed from the atomic one using connectives.

In other words, statements in propositional logic can be defined using
propositional formulas (also known as sentential formulas or Boolean formulas).
\begin{definition}
  We say that a finite sequence $\phi$ of elements of the set
  $V \cup \set{\lnot, \lor, \land, \limplies, \text{``(''}, \text{``)''}}$
  is a propositional formula on the variables from $V$ if
  \begin{itemize}
    \item either $\phi$ is equal to $x$ for some $x \in V$,
    \item or $\phi$ is equal to $(\psi_1 \land \psi_2)$, or
      $(\psi_1 \lor \psi_2)$, or $(\psi_1 \limplies \psi_2)$,\footnote{%
        The symbol $\limplies$ is used to denote the implication.
        Due to historical reasons the standard symbol $\implies$ is rarely
        used as a connective in mathematical logic; hence, we will use
        $\limplies$ instead of $\implies$ in this part of the book.
        It is important to note that, sometimes the symbol $\supset$ is also
        used instead of $\implies$.
      }
      where $\psi_1$ and $\psi_2$ are propositional formulas on the variables
      from $V$,
    \item or $\phi$ is equal to $(\lnot \psi)$, where $\psi$ is a propositional
      formula on the variables from $V$.
  \end{itemize}

  We denote the set of all propositional formulas by $\prop{V}$.
\end{definition}

For example, $((x_1 \lor (\lnot x_2)) \land x_3)$ is a propositional formula on
the variables from $\set{x_1, x_2, x_3}$ (we also say that it is a formula on
$x_1$, $x_2$, $x_3$).

\begin{exercise}
  Write the definition of propositional formulas using the terminology
  ``the set generated by \dots from \dots'' (see \Cref{chapter:functions}).
\end{exercise}

Hereafter when naming formulas, we will not mention explicitly all the
parenthesis. To establish a more compact notation, we adopt the following
conventions.
\begin{itemize}
  \item The outermost parentheses do not need to be explicitly mentioned; e.g.,
    we write ``$A \land B$'' to refer to $(A \land B)$.
  \item The negation symbol applies to as little as possible.
    For example, $\lnot A \land B$ denotes $(\lnot A) \land B$;
    i.e., $((\lnot A) \land B)$. Which is not the same as
    $(\lnot (A \land B))$.
  \item The conjunction and disjunction symbols apply to as little as possible,
    given that convention 2 is to be observed. For example,
    $A \land B \limplies \lnot C \lor D$ is
    $((A \land B) \limplies ((\lnot C) \lor D))$.
  \item Where one connective symbol is used repeatedly, grouping is to the
    right: $A \land B \land C$ is $A \land (B \land C)$,
    $A \limplies B \limplies C$ is $A \limplies (B \limplies C)$.
\end{itemize}

Interpreting propositional logic is not difficult since the considered entities
have a simple structure. The propositions are built up from rough blocks by
adding connectives. The simplest parts (atoms) are of the form ``cows are
animals'', ``Earth is flat'', ``$2 \times 2 = 2$'', which are simply true or
false. We extend this assignment of truth values to composite propositions, by
reflection on the meaning of the logical connectives.

\begin{definition}
  A function $v : \prop{V} \to \set{\ltrue, \lfalse}$ is a valuation if
  \begin{itemize}
    \item $v(\lnot \psi) = \lnot v(\psi)$,
    \item $v(\psi_1 \land \psi_2) = v(\psi_1) \land v(\psi_2)$,
    \item $v(\psi_1 \lor \psi_2) = v(\psi_1) \lor v(\psi_2)$, and
    \item $v(\psi_1 \limplies \psi_2) = v(\psi_1) \limplies v(\psi_2)$.
  \end{itemize}
\end{definition}

We may note that all the valuations are actually can be defined by the values
of variables.
\begin{theorem}
  Let $\rho : V \to \set{\ltrue, \lfalse}$ be a function (we say that $\rho$ is
  a propositional assignement). Then there is a unique valuation
  $\substitute{\cdot}{\rho} : \prop{V} \to \set{\ltrue, \lfalse}$ such that
  $\substitute{x}{\rho} = \rho(x)$ for any $x \in V$.
\end{theorem}

Since any valuation can be defined by the values assigned to variables, we need
to introduce the following notation.
If $V = \set{x_1, \dots, x_n}$ and $v_1, \dots, v_n \in \set{\ltrue, \lfalse}$,
then $\substitute{\cdot}{x_1 = v_1, \dots, x_n = v_n}$ denotes the valuation
such that $\substitute{x_i}{x_1 = v_1, \dots, x_n = v_n} = v_i$ for each
$i \in \range{n}$.


For example, the value of a formula $(x_1 \land \lnot x_2) \lor x_3$ when
$\ltrue$ is substituted as the value of $x_1$, $\ltrue$ is substituted as the
value of $x_2$, and $\lfalse$ is substituted as the value of $x_3$ is equal to
$(\ltrue \land \lfalse) \lor \lfalse = \lfalse$.

Note that if $\phi$ is a formula on the variables from $V$ it does not mean that
all the variables from $V$ have to be used.
For example, $x_1$ is a formula on the variables from $\set{x_1, x_2}$; however,
$x_2$ is not used in the formula.

\begin{exercise}
  Give a formal definition (using structural induction) of the set of all the
  variables that are used in a propositional formula $\phi$ on variables from a
  set $V$.
\end{exercise}

Let $\phi$ be a formula on the variables from a set $V$. The definition
of a value of a formula requires us to specify all the values of all the
variables from $V$. However, the following theorem shows that in
fact we need to specify only the variables that are actually used in $\phi$.
\begin{theorem}
  Let $\phi$ be a formula $\phi$ on the variables from a set $V$,
  and $U$ be the set of the variables used in $\phi$.

  Consider $\rho_1, \rho_2 : V \to \set{\ltrue, \lfalse}$ such that
  $\rho_1(x) = \rho_2(x)$ for any $x \in U$.
  Then $\substitute{\phi}{\rho_1} = \substitute{\phi}{\rho_2}$.
\end{theorem}
\begin{proof}
  We prove the statement using the structural induction.
  \begin{description}
    \item[(base case)] Let $\phi = x$ for some $x \in V$.
      Note that $x \in U$ and $\substitute{\phi}{\rho_1} = \rho_1(x) =
      \rho_2(x) = \substitute{\phi}{\rho_2}$.
    \item[(induction step)] We need to consider the following three cases.
      \begin{itemize}
        \item Let $\phi$ be equal to $\psi_1 \land \psi_2$ such that
          $\substitute{\psi_1}{\rho_1} = \substitute{\psi_1}{\rho_2}$ and
          $\substitute{\psi_1}{\rho_2} = \substitute{\psi_2}{\rho_2}$.
          In this case,
          \[
            \substitute{\phi}{\rho_1} =
            (\substitute{\psi_1}{\rho_1} \land \substitute{\psi_2}{\rho_1} )=
            (\substitute{\psi_1}{\rho_2} \land \substitute{\psi_2}{\rho_2}) =
            \substitute{\phi}{\rho_2}.
          \]
        \item Let $\phi$ be equal to $\psi_1 \lor \psi_2$ such that
          $\substitute{\psi_1}{\rho_1} = \substitute{\psi_1}{\rho_2}$ and
          $\substitute{\psi_1}{\rho_2} = \substitute{\psi_2}{\rho_2}$.
          In this case,
          \[
            \substitute{\phi}{\rho_1} =
            (\substitute{\psi_1}{\rho_1} \lor \substitute{\psi_2}{\rho_1}) =
            (\substitute{\psi_1}{\rho_2} \lor \substitute{\psi_2}{\rho_2}) =
            \substitute{\phi}{\rho_2}.
          \]
        \item Let $\phi$ be equal to $\psi_1 \limplies \psi_2$ such that
          $\substitute{\psi_1}{\rho_1} = \substitute{\psi_1}{\rho_2}$ and
          $\substitute{\psi_1}{\rho_2} = \substitute{\psi_2}{\rho_2}$.
          In this case,
          \[
            \substitute{\phi}{\rho_1} =
            (\substitute{\psi_1}{\rho_1} \limplies \substitute{\psi_2}{\rho_1}) =
            (\substitute{\psi_1}{\rho_2} \limplies \substitute{\psi_2}{\rho_2}) =
            \substitute{\phi}{\rho_2}.
          \]
      \end{itemize}
  \end{description}
\end{proof}

\begin{exercise}
  Let $\phi_1$, $\phi_2$, and $\phi_3$ be propositional formulas on the
  variables from a set $V$. Show that for any propositional assignement
  $\rho$ to $V$,
  $\substitute{\phi_1 \land (\phi_2 \land \phi_3)}{\rho} =
   \substitute{(\phi_1 \land \phi_2) \land \phi_3}{\rho}$.
\end{exercise}

\section{Conjunctive and Disjuctive Normal Form}

Let $\phi_1$, \dots, $\phi_n$ be some propositional formulas. Then
\begin{itemize}
  \item $\bigland_{i = 1}^1 \phi_i = \phi_1$ and
    $\biglor_{i = 1}^1 \phi_i = \phi_1$, and
  \item $\bigland_{i = 1}^{k + 1} \phi_i =
    (\bigland_{i = 1}^{k} \phi_i) \land \phi_{k + 1}$ and
    $\biglor_{i = 1}^{k + 1} \phi_i =
      (\biglor_{i = 1}^{k} \phi_i) \lor \phi_{k + 1}$.
\end{itemize}
In other words $\bigland_{i = 1}^n \phi_i$ and $\biglor_{i = 1}^n \phi_i$
denotes the conjunction of the formulas $\phi_1$, \dots, $\phi_n$, and
$\biglor_{i = 1}^n \phi_i$ denotes the disjunction of them.

\begin{exercise}
  Let $\phi_1$, \dots, $\phi_n$, $\psi_1$, \dots, $\psi_m$, $\chi_1$, \dots,
  $\chi_{n + m}$ be some propositional formulas on the variables from $V$
  such that $\chi_i = \phi_i$ for $i \le n$ and $\chi_i = \psi_{i - n}$ for
  $n < i \le m$. Show that
  $\substitute{\left(\bigland_{i = 1}^n \phi_i\right) \land
    \left(\bigland_{i = 1}^n \psi_i\right)}{\rho} =
  \substitute{\left(\bigland_{i = 1}^{n + m} \chi_i\right)}{\rho}$
  for any propositional assignement $\rho$ to $V$.
\end{exercise}


Using this notation we may show that propositional formulas can represent all
the Boolean functions (functions from $\set{\ltrue, \lfalse}^n$ to
$\set{\ltrue, \lfalse}$).
\begin{theorem}
\label{theorem:function-to-formula}
  For any function $f : \set{\ltrue, \lfalse}^n \to
  \set{\ltrue, \lfalse}$ there is a
  formula $\phi$ on the variables $x_1$, \dots, $x_n$ such that
  $\substitute{\phi}{x_1 = v_1, \dots, x_n = v_n} = f(v_1, \dots, v_n)$ for all
  $v_1, \dots, v_n \in \set{\ltrue, \lfalse}$.
\end{theorem}

Let $u \in \set{\ltrue, \lfalse}$ and $x \in V$. Then $x^u$ denotes a
formula on the variables from $V$ such that $x^u = x$ if $u = \ltrue$ and
$x^u = \lnot x$ if $u = \lfalse$. Note that $\substitute{x^u}{\rho} = \ltrue$
iff $\rho(x) = u$, for any propositional assignement $\rho$ to $V$.
Indeed, if $u = \ltrue$, then $x^u = x$ and
$\ltrue = \substitute{x^u}{\rho} = \substitute{x}{\rho} = \rho(x)$ so
$\rho(x) = \ltrue = u$;
if $u = \lfalse$, then $x^u = \lnot x$ and
$\ltrue = \substitute{x^u}{\rho} = \substitute{\left(\lnot x\right)}{\rho} =
\lnot \rho(x)$ so $\rho(x) = \lfalse = u$.

\begin{exercise}
  Let $\phi_1$, \dots, $\phi_k$ are propositional formulas
  on the variables from $V$.
  \begin{itemize}
    \item Show that
      $\substitute{\left(\biglor_{i = 1}^k \phi_i\right)}{\rho} = \ltrue$ iff
      $\substitute{\phi}{\rho} = \ltrue$ for some $i \in \range{k}$.
    \item Show that
      $\substitute{\left(\bigland_{i = 1}^k \phi_i\right)}{\rho} = \ltrue$ iff
      $\substitute{\phi}{\rho} = \ltrue$ for all $i \in \range{k}$.
  \end{itemize}
\end{exercise}

Using this observation and the exercise we can
prove~\Cref{theorem:function-to-formula}.
\begin{proof}
  Let $S = \set[f(u_1, \dots, u_n) = \ltrue]{(u_1, \dots, u_n) \in
    \set{\ltrue, \lfalse}^n}$.
  Assume that
  $S = \set{(u_{1, 1}, \dots, u_{1, n}), \dots, (u_{k, 1}, \dots, u_{k, n})}$.
  By the previous observations
  \[
    \substitute{
      \left(
        \biglor_{i = 1}^k
          \bigland_{j = 1}^n x_j^{u_{i, j}}
      \right)
    }{x_1 = v_1, \dots, x_n = v_n}
    =
    f(v_1, \dots, v_n)
  \]
  for all $v_1, \dots, v_n \in \set{\ltrue, \lfalse}$.
  (Note that we have not considered the case when $S = \emptyset$, in this
  case $f$ is a constant $\lfalse$ function and it is equal to
  $x_1 \land \lnot x_1$.)
\end{proof}

One may notice that the formulas we constructed have very specific form,
such a form is called disjunctive normal form (DNF).
\begin{definition}
  We say that a propositional formula $\lambda$ on the variables from $V$
  is a \emph{literal} if it is equal to $x$ or to $\lnot x$ for some
  $x \in V$.

  We say that a propositional formula $\psi$ on the variables from $V$ is
  a \emph{term} if $\psi$ is equal to $\bigland_{i = 1}^\ell \lambda_i$, where $\lambda_1$, \dots, $\lambda_\ell$ are literals.

  Finally, we say that a propositional formula $\phi$ on the variables from
  $V$ is in \emph{disjunctive normal form} (DNF) if $\phi$ is equal to
  $\biglor_{i = 1}^k \psi_i$, where $\psi_1$, \dots, $\psi_k$ are
  terms.
\end{definition}

However, there is nothing special in this order of operations (disjunction of
conjunctions). So we can define conjunctive normal form (CNF) too.
\begin{definition}
  We say that a propositional formula $\psi$ on the variables from $V$ is
  a \emph{clause} if $\psi$ is equal to $\biglor_{i = 1}^\ell \lambda_i$, where
  $\lambda_1$, \dots, $\lambda_\ell$ are literals.

  Finally, we say that a propositional formula $\phi$ on the variables from
  $V$ is in \emph{conjunctive normal form} (DNF) if $\phi$ is equal to
  $\bigland_{i = 1}^k \psi_i$, where $\psi_1$, \dots, $\psi_k$ are
  clauses.
\end{definition}

Using the following simple trick we can prove that any function
has a representation in CNF. First, we define a function
$g(x_1, \dots, x_n) = \lnot f(x_1, \dots, x_n)$. Secondly, we may notice that
\[
  \substitute{
    \left(
      \lnot
      \left(
        \bigland_{i = 1}^k \biglor_{j = 1}^n \phi_{i, j}
      \right)
    \right)
  }{x_1 = v_1, \dots, x_n = v_n}
  =
  \substitute{
    \left(
      \biglor_{i = 1}^k \bigland_{j = 1}^n \lnot \phi_{i, j}
    \right)
  }{x_1 = v_1, \dots, x_n = v_n}
\]
for all $v_1, \dots, v_n \in \set{\ltrue, \lfalse}$
% THE REFERENCE TO THE EXERCISE
(see Exercise 15.7). Therefore the negation
of a formula in DNF can be easily transformed into a formula in CNF.
Finally, we know that the function
$g$ has a representation in DNF, which implies that $f$ has a representation
in CNF.

\begin{chapterendexercises}
  \exercise % CAREFUL, DONT MOVE! THERE IS A REFERENCE TO THIS NUMBER
    Let $\phi_1$ and $\phi_2$ be some propositional formulas on
    the variables from $V$. Show that for any propositional assignement
    $\rho$ to $V$,
    \begin{itemize}
      \item
        $\substitute{
        \lnot\left(
          \phi_1 \land \phi_2
        \right)}{\rho} =
        \substitute{
          \left(
            \lnot \phi_1 \lor \lnot \phi_2
          \right)
         }{\rho}$ and
    \item
    $\substitute{
    \lnot\left(
      \phi_1 \lor \phi_2
    \right)}{\rho} =
    \substitute{
      \left(
        \lnot \phi_1 \land \lnot \phi_2
      \right)
     }{\rho}$.
  \end{itemize}
  \exercise % CAREFUL, DONT MOVE! THERE IS A REFERENCE TO THIS NUMBER
    Let $\phi_1$, \dots, $\phi_n$ be some propositional formulas on
    the variables from $V$. Show that for any propositional assignement
    $\rho$ to $V$,
    \begin{itemize}
      \item
        $\substitute{
          \left(
            \lnot \left(
                    \bigland_{i = 1}^n \phi_i
                  \right)
          \right)}{\rho} =
          \substitute{
            \left(
              \biglor_{i = 1}^n \phi_i
            \right)
           }{\rho}$ and
      \item
        $\substitute{
         \left(
           \lnot \left(
                   \biglor_{i = 1}^n \phi_i
                 \right)
         \right)}{\rho} =
         \substitute{
           \left(
             \bigland_{i = 1}^n \phi_i
           \right)
          }{\rho}$.
    \end{itemize}
  \exercise Let $\phi = \bigvee_{i = 1}^m \lambda_i$ be a clause; we say that
    the width of the clause is equal to $m$.
    Let $\phi = \bigwedge_{i = 1}^\ell \chi_i$ be a formula in CNF
    ($\chi_i$'s are clauses'); we say that the width of $\phi$ is equal to
    the maximal width of $\chi_i$ for $i \in \range{\ell}$.

    Let $p_n : \set{\ltrue, \lfalse}^n \to \set{\ltrue, \lfalse}$ such that
    $p_n(x_1, \dots, x_n) = \ltrue$
    iff the set $\set[x_i = \ltrue]{i}$ has an odd number of elements.
    Show that any CNF representation of $p_n$ has width $n$.
\end{chapterendexercises}
