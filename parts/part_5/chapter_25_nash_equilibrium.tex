\chapter{Nash Equilibrium}
In the previous chapter we discussed several ways to find optimal behaviour for
players. However, there is a big class of games where these method do not work;
e.g. the game corresponding to \Cref{table:heads-and-tales-game}.


\section{No Regrets Strategies}
The core ideas leading to understanding of the optimal behaviour are Nash
equilibria and the notion of mixed strategies.

To understand the equilibria, consider the pair of strategies ``cooperates'',
``cooperates'' in prisoner's dilemma. This situation is bad because each player
can change their behaviour and improve his/her situation; i.e., this situation
is not stable.
\begin{definition}
  Let $(X, Y, A, B)$ be a game in normal form. We say that a pair of strategies
  $x \in X$ and $y \in Y$ are in \emph{Nash equilibrium} if 
  \begin{gather*}
    A(x, y) \ge A(x', y) \\
    \text{and} \\
    B(x, y) \ge B(x, y')
  \end{gather*}
  for all $x' \in X$ and $y' \in Y$.
\end{definition}
In other words, if two players decided to stick to $x$ and $y$ that are in Nash
equilibrium, they do not regret not vialating their agreement.
\begin{exercise}
  \begin{itemize}
    \item Check that the pair of strategies ``defects'', ``defects'' are in Nash
      equilibrium in the prisoner's dilemma.
    \item Check that there are no Nash equilibrium in the game corresponding to
      \Cref{table:heads-and-tales-game}.
  \end{itemize}
\end{exercise}

Note that some games may have several Nash equilibria. For example, let us
consider the following game.
\begin{game}
  Two people are leaving in a flat. They need to call a plumber and stay in the
  flat while the plumber works.
  \begin{itemize}
    \item If none of them call the plumber, the water leaks and each of them
      will need to spend $1$ hours cleaning the appartement.
    \item If one of them calls the plumber, he/she will spend $1$ hour waiting
      pipes to be fixed.
    \item Alternatively they can call together and wait for one our together.
  \end{itemize}
  This game is a version of \emph{volunteer's dilemma}.
\end{game}
It is clear that this game can be described using the following table.
\begin{center}
  \begin{tabular}{l l l  l  l  l  l  l  l}
    \toprule
                       & do nothing  & call the plumber   \\
    \midrule
    do nothing         & -1, -1    & 0, -1    \\
    call the plumber   & -1, 0       & -1, -1   \\
    \bottomrule
  \end{tabular}
\end{center}
We may notice that in this game all pairs except ``call the plumber'', ``call the
plumber'' are in Nash equilibrium. However, the pair of strategies ``do
nothing'', ``do nothing'' is not efficient since one may improve the situation
for one of the players without harming the other. Moreover, if the players know
that this situation is going to repeat it self they may agree to call the
plamber in turns and reduce the total time they wasted. 

\section{Mixed Strategies}
Recall that we still do not have any optimal behaviour in the game corresponding
to \Cref{table:heads-and-tales-game}; however, it seems clear that in real life
the best behaviour would be to flip coins and choose strategies accordingly.

This observation leads to the notion of mixed strategies.
\begin{definition}
  Let $(X, Y, A, B)$ be a game in normal form. We say that a probability
  distribution $\bar{x}$ over $X$ is a mixed strategy for the first player.
  Similarly we say that a probability distribution $\bar{y}$ over $Y$ is a mixed
  strategy for the second player.


  We say that the payoffs of the players if they choose mixed strategies
  $\bar{x}$ and $\bar{y}$ are $A(\bar{x}, \bar{y})$ and $B(\bar{x}, \bar{y})$ 
  such that
  \begin{gather*}
    A(\bar{x}, \bar{y}) = 
      \expectation[x \gets \bar{x}]{\expectation[y \gets \bar{y}]{A(x, y)}} \\ 
    \text{and} \\
    B(\bar{x}, \bar{y}) = 
      \expectation[x \gets \bar{x}]{\expectation[y \gets \bar{y}]{B(x, y)}}.
  \end{gather*}
\end{definition}

\begin{exercise}
  Let $f : \Omega_1 \times \Omega_2 \to \R$ and let $\Distribution{D}_1$ and
  $\Distribution{D}_2$ be probability distributions on $\Omega_1$ and
  $\Omega_2$, respectively. Show that 
  \[
  \expectation[\omega_1 \gets \Distribution{D}_1]{
    \expectation[\omega_2 \gets \Distribution{D}_2]{f(\omega_1, \omega_2)}
  } = 
  \expectation[\omega_2 \gets \Distribution{D}_2]{
    \expectation[\omega_1 \gets \Distribution{D}_1]{f(\omega_1, \omega_2)}}.
  \]
\end{exercise}

Let us find the payoffs in the game corresponding  to
\Cref{table:heads-and-tales-game} if the players choose heads with the
probability $1 / 2$; i.e., for the strategies $\bar{x}$ and $\bar{y}$ such that
$\bar{x}(x) = \bar{y}(y) = 1 / 2$ for all $x \in X$ and $y \in Y$. It is clear
that 
\begin{multline*}
  A(\bar{x}, \bar{y}) = 
  \frac{1}{4} A(\text{heads}, \text{heads}) + 
  \frac{1}{4} A(\text{heads}, \text{tails}) + \\
  \frac{1}{4} A(\text{tails}, \text{heads}) +
  \frac{1}{4} A(\text{tails}, \text{tails}) = 0.
\end{multline*}

It is clear that a mixed strategy $\bar{x}$ such that $\bar{x}(x) = 1$ is
essentialy the same as $x$ so we call such a strategy a \emph{pure strategy}.

The most importnat result about the connection between mixed strategies and Nash
equilibrium is von Neumann's minimax theorem.
\begin{theorem}[von Neumann's minimax theorem]
  Let $(X, Y, A, B)$ be a zero-sum game in normal form. Then there are mixed
  strategies $\bar{x}$ and $\bar{y}$ that are in Nash equilibrium; i.e., 
  \begin{gather*}
    A(\bar{x}, \bar{y}) \ge A(\bar{x}', \bar{y}) \\
    \text{and} \\
    A(\bar{x}, \bar{y}) \le A(\bar{x}, \bar{y}')
  \end{gather*}
  for all mixed strategies $\bar{x}'$ and $\bar{y}'$
\end{theorem}
Note that this theorem can be interpreted as the following equality 
\[
  \max_{\bar{x}} \min_{\bar{y}} A(\bar{x}, \bar{y}) = 
  \min_{\bar{y}} \max_{\bar{x}} A(\bar{x}, \bar{y}),
\]
which give this theorem its name.
