\chapter{Proofs by Induction}
\label{chapter:induction}
\section{Simple Induction}
\marginurl{%
  The Induction Principle:\\\noindent
  Introduction to Mathematical Reasoning \#4
}{youtu.be/jOnZTWGpX_I}

Let us consider a simple problem: what is bigger $2^n$ or $n$? In this chapter,
we are going to study the simplest way to prove that $2^n > n$ for all positive
integers $n$. First, let us check that it is true for small integers $n$.
\begin{center}
  \begin{tabular}{l l l  l  l  l  l  l  l}
    \toprule
          & 1 & 2 & 3 & 4  & 5  & 6  & 7   & 8   \\
    \midrule
    $n$   & 1 & 2 & 3 & 4  & 5  & 6  & 7   & 8   \\
    $2^n$ & 2 & 4 & 8 & 16 & 32 & 64 & 128 & 256 \\
    \bottomrule
  \end{tabular}
\end{center}
We may also note that $2^n$ is growing faster than $n$, so we expect that if
$2^n > n$ for small integers $n$, then it is true for all positive integers $n$.

In order to prove this statement formally, we use the following principle.
\begin{principle}[The Induction Principle]
  Let $P(n)$ be some statement about a positive integer $n$.
  Hence, $P(n)$ is true for every positive integer $n$ iff
  \begin{description}
    \item [(the base case)] $P(1)$ is true and
    \item [(the induction step)] $P(k) \implies P(k + 1)$ is true
      for all positive integers $k$.
  \end{description}
\end{principle}

Let us prove now the statement using this principle.
We define $P(n)$ be the statement that ``$2^n > n$''.
$P(1)$ is true since $2^1 > 1$. Let us assume now that $2^n > n$. Note that
$2^{n + 1} = 2 \cdot 2^n > 2n \ge n + 1$. Hence, we proved the induction step.

\begin{exercise}
  Prove that $(1 + x)^n \ge 1 + nx$ for all positive integers $n$ and real
  numbers $x \ge -1$.
\end{exercise}

\section{Changing the Base Case}
Let us consider functions $n^2$ and $2^n$.

\begin{center}
    \begin{tabular}{l  l  l  l  l  l  l  l  l}
        \toprule
              & 1 & 2 & 3 & 4  & 5  & 6  & 7   & 8   \\
        \midrule
        $n^2$ & 1 & 4 & 9 & 16 & 25 & 36 & 49  & 64  \\
        $2^n$ & 2 & 4 & 8 & 16 & 32 & 64 & 128 & 256 \\
        \bottomrule
    \end{tabular}
\end{center}
Note that $2^n$ is greater than $n^2$ starting from $5$. But without some trick
we can not prove this using induction since for $n = 3$ it is not true!

The trick is to use the statement $P(n)$ stating that $(n + 4)^2 < 2^{n + 4}$.
The base case when $n = 1$ is true.
Let us now prove the induction step. Assume that $P(k)$ is true i.e.
$(k + 4)^2 < 2^{k + 4}$. Note that $2(k + 4)^2 < 2^{k + 1 + 4}$ but
$(k + 5)^2 = k^2 + 10k + 25 \le 2k^2 + 16k + 32 = 2(k + 4)^2$.
Which implies that
$2^{k + 1 + 4} > (k + 5)^2$. So $P(k + 1)$ is also true.

In order to avoid this strange $+4$ we may change the base
case and use the following argument.

\begin{theorem}
\label{theorem:induction-shifted-base}
    Let $P(n)$ be some statement about an integer $n$.
    Hence, $P(n)$ is true for every integer $n > n_0$ iff
    \begin{description}
        \item [(the base case)] $P(n_0 + 1)$ is true and
        \item [(the induction step)] $P(k) \implies P(k + 1)$ is true for all
            integers $k > n_0$.
    \end{description}
\end{theorem}

Using this generalized induction principle we may prove that $2^n \ge n^2$ for
$n \ge 4$. The base case for $n = 4$ is true. The induction step is also true;
indeed let $P(k)$ be true i.e. $(k + 4)^2 < 2^{k + 4}$. Hence,
$2(k + 4)^2 < 2^{k + 1 + 4}$ but
$(k + 5)^2 = k^2 + 10k + 25 \le 2k^2 + 16k + 32 = 2(k + 4)^2$.

Let us now prove the theorem. Note that the proof is based on an idea similar
to the trick with $+4$, we just used.
\begin{proof}[Proof of Theorem~\ref{theorem:induction-shifted-base}]
    \begin{description}
        \item[$\Rightarrow$] If $P(n)$ is true for any $n > n_0$ it is also true
            for $n = n_0 + 1$ which implies the base case. Additionally, it true for
            $n = k + 1$ so the induction step is also true.
        \item[$\Leftarrow$] In this direction the proof is a bit harder. Let us
            consider a statement $Q(n)$ saying that $P(n + n_0)$ is true. Note that
            by the base case for $P$, $Q(1)$ is true; by the induction step for $P$
            we know that $Q(n)$ implies $P(n + 1)$. As a result, by the induction
            principle $Q(n)$ is true for all positive integers $n$. Which implies
            that $P(n)$ is true for all integers $n > n_0$.
    \end{description}
\end{proof}

\section{Inductive Definitions}

We may also define objects inductively. Let us consider the sum
$1 + 2 + \dots + n$ a line of dots indicating ``and so on'' which indicates the
definition by induction. In this case, a more precise notation is
$\sum_{i = 1}^n i$.

\begin{definition}
    Let $a(1)$, \dots, $a(n)$, \dots be a sequence of integers. Then
    $\sum_{i = 1}^n a(i)$ is defined inductively by the following
    statements:
    \begin{itemize}
        \item $\sum_{i = 1}^1 a(i) = a(1)$, and
        \item $\sum_{i = 1}^{k + 1} a(i) =
            \sum_{i = 1}^k a(i) + a(k + 1)$.
    \end{itemize}
\end{definition}
\nomenclature[C]{$\sum_{i = 1}^k \alpha_i$}{denotes $\alpha_1 + \dots +
\alpha_k$}

Let us prove that $\sum_{i = 1}^n i = \frac{n (n + 1)}{2}$.
Note that by definition $\sum_{i = 1}^1 i = 1$ and
$\frac{1 (1 + 1)}{2} = 1$; hence, the base case holds. Assume that
$\sum_{i = 1}^n i = \frac{n (n + 1)}{2}$. Note that
$\sum_{i = 1}^{n + 1} i = \sum_{i = 1}^n i + (n + 1)$ and by the
induction hypothesis $\sum_{i = 1}^n i = \frac{n (n + 1)}{2}$.
Hence, $\sum_{i = 1}^{n + 1} i = \frac{n (n + 1)}{2} + (n + 1) =
\frac{(n + 1)(n + 2)}{2}$.

\begin{exercise}
    Prove that $\sum_{i = 1}^n 2^i = 2^{n + 1} - 2$.
\end{exercise}

\section{Analysis of Algorithms with Cycles}

Induction is very useful for analysing algorithms using cycles. Let us extend
the example we considered in Section~\ref{section:simple-algorithm}.

Let us consider the following algorithm.
\begin{algorithm}
  \begin{algorithmic}[1]
    \Function{Max}{$a_1$, \dots, $a_n$}
      \State{$r \gets a_1$}
      \For{$i$ from $2$ to $n$}
        \If{$a_i > r$}
          \State{$r \gets a_i$}
        \EndIf
      \EndFor
      \State\Return{r}
    \EndFunction
  \end{algorithmic}
  \caption{The algorithm that finds the maximum element of $a_1$, \dots, $a_n$.}
\end{algorithm}
We prove that it is working correctly. First, we need to define $r_1$,
\dots, $r_n$ the value of $r$ during the execution of the algorithm.
It is easy to see that $r_1 = a_1$ and
\[
    r_{i + 1} =
    \begin{cases}
        r_i & \text{if } r_i > a_{i + 1} \\
        a_{i + 1} & \text{otherwise}
    \end{cases}.
\]
Secondly, we prove by induction that $r_i$ is the maximum of $a_1$, \dots,
$a_i$. It is clear that the base case for $i = 1$ is true. Let us prove the
induction step from $k$ to $k + 1$. By the induction hypothesis, $r_k$
is the maximum of $a_1$, \dots, $a_k$. We may consider two following cases.
\begin{itemize}
    \item If $r_k > a_{k + 1}$, then $r_{k + 1} = r_{k}$ is the maximum of $a_1$,
        \dots, $a_{k + 1}$ since $r_k$ is the maximum of $a_1$, \dots, $a_k$.
    \item Otherwise, $a_{k + 1}$ is greater than or equal to $a_1$, \dots, $a_k$,
        hence, $r_{k + 1} = a_{k + 1}$.
\end{itemize}

\begin{exercise}
\label{exercise:selection-sort}
    Show that line~6 in the following sorting algorithm executes
    $\frac{n (n + 1)}{2}$ times.
    \begin{algorithm}
        \begin{algorithmic}[1]
            \Function{SelectionSort}{$a_1$, \dots, $a_n$}
                \For{$i$ from $1$ to $n$}
                    \State{$r \gets a_i$}
                    \State{$\ell \gets i$}
                    \For{$j$ from $i$ to $n$}
                        \If{$a_j > r$}
                            \State $r \gets a_j$
                            \State $\ell \gets j$
                        \EndIf
                    \EndFor
                    \State{Swap $a_i$ and $a_\ell$.}
                \EndFor
            \EndFunction
    \end{algorithmic}
    \caption{The algorithm is selection sort, it sorts $a_1$, \dots, $a_n$.}
  \end{algorithm}
\end{exercise}



\section{Strong Induction}

Sometimes $P(k)$ is not enough to prove $P(k + 1)$ and we need all the
statements $P(1)$, \dots, $P(k)$. In this case we may use the following
induction  principle.

\begin{theorem}[The Strong Induction Principle]
\label{theorem:strong-induction}
    Let $P(n)$ be some statement about positive integer $n$.
    Hence, $P(n)$ is true for every integer $n > n_0$ iff
    \begin{description}
        \item [(the base case)] $P(n_0 + 1)$ is true and
        \item [(the induction step)] If $P(n_0 + 1)$, \dots, $P(n_0 + k)$ are true,
            then $P(n_0 + k + 1)$ is also true for all positive integers $k$.
  \end{description}
\end{theorem}

Before we prove this theorem let us present some applications of this
principle.

The Fibonacci numbers are defined as follows:
$f_0 = 0$, $f_1 = 1$, and $f_k = f_{k - 1} + f_{k - 2}$ for $k \ge 2$ (note
that they are also defined using strong induction since we use not only
$f_{k - 1}$ to define $f_k$).
\begin{theorem}[The Binet formula]
    The Fibonacci numbers are given by the following formula
    \[
        f_n = \frac{\alpha^n - \beta^n}{\sqrt{5}},
    \]
    where $\alpha = \frac{1 + \sqrt{5}}{2}$ and
    $\beta = \frac{1 - \sqrt{5}}{2}$.
\end{theorem}
\begin{proof}
    We use the strong induction principle to prove this statement with $n_0 = -1$.
    Let us first prove the base case,
    $\frac{(\alpha^0 - \beta^0)}{\sqrt{5}} = 0 = f_0$.
    We also need to prove the induction step.
    \begin{itemize}
        \item If $k = 1$, then $\frac{(\alpha^1 - \beta^1)}{\sqrt{5}} = 1 = f_1$.
        \item Otherwise, by the induction hypothesis,
            $f_k = \frac{\alpha^k - \beta^k}{\sqrt{5}}$ and
            $f_{k - 1} = \frac{\alpha^{k - 1} - \beta^{k - 1}}{\sqrt{5}}$.
            By the definition of the Fibonacci numbers $f_{k + 1} = f_k + f_{k - 1}$.
            Hence,
            \[
                f_{k + 1} = \frac{\alpha^k - \beta^k}{\sqrt{5}} +
                \frac{\alpha^{k - 1} - \beta^{k - 1}}{\sqrt{5}}.
            \]
            Note that it is enough to show that
            \begin{equation}
            \label{equation:binet}
                \frac{\alpha^{k + 1} - \beta^{k + 1}}{\sqrt{5}} =
                \frac{\alpha^k - \beta^k}{\sqrt{5}} +
                \frac{\alpha^{k - 1} - \beta^{k - 1}}{\sqrt{5}}.
            \end{equation}
            Note that it is the same as
            \[
                \frac{\alpha^{k + 1} - \alpha^k - \alpha^{k - 1}}{\sqrt{5}} =
                \frac{\beta^{k + 1} - \beta^k - \beta^{k - 1}}{\sqrt{5}}.
            \]
            Additionally, note that $\alpha$ and $\beta$ are roots of the equation
            $x^2 - x - 1 = 0$. Hence,
            $\alpha^{k + 1} - \alpha^k - \alpha^{k - 1} = \alpha^{k - 1}(\alpha^2 -
            \alpha - 1) = 0$ and
            $\beta^{k + 1} - \beta^k - \beta^{k - 1} =
            \beta^{k - 1}(\beta^2 - \beta - 1) = 0$. Which implies
            equality~(\ref{equation:binet}).
    \end{itemize}
\end{proof}

Another example of an application of the strong induction is the proof that
any number can be written in digital numeral systems with any base.
\begin{theorem}
  Let $b > 1$ be an integer. Then there is a unique representation 
  $(c_0, \dots, c_\ell)_b$ of any positive number $n$ in the base-$b$ digital
  numeral system. 
  In other words, for any positive integer $n$, there are unique 
  $0 \le c_0, \dots, c_\ell < b$ such that $n = \sum_{i = 0}^\ell b^i c_i$.
\end{theorem}
\begin{proof}
  We prove the statement using strong induction by $n$. The base case for
  $n < b$ is clear (we can choose $\ell = 0$ and $c_0 = n$).
  Let us now prove the induction step. Assume the statement is true for all
  $k < n$. Let $n$ divided by $b$ be equal to $q$ with the remainder $c_0$.
  Note that $(n - c_0) / b < n$ is a positive integer. Hence,
  by the induction hypothesis, there are
  $0 \le c_1, \dots, c_\ell < b$
  such that $(n - c_0) / b = \sum_{i = 1}^\ell b^{i - 1} c_i$. Hence,
  $n = \sum_{i = 0}^\ell b^i c_i$.
\end{proof}


Now we are ready to prove the strong induction principle.
\begin{proof}[Proof of Theorem~\ref{theorem:strong-induction}]
  It is easy to see that if $P(n)$ is true for all $n > n_0$, then the base
  case and the induction steps are true. Let us prove that if the base case and
  the induction step are true, then $P(n)$ is true for all $n > n_0$.

  Let $Q(k)$ be the statement that $P(n_0 + 1)$, \dots, $P(n_0 + k)$ are true.
  Note that $Q(1)$ is true by the base case for $P$. Additionally, note that if
  $Q(k)$ is true, then $Q(k + 1)$ is also true, by the induction step for $P$.
  Hence, by the induction principle, $Q(k)$ is true for all positive integers
  $k$. Which implies that $P(n_0 + k)$ is true for all positive integers $k$.
\end{proof}


\section{Analysis of Recursive Algorithms}
To illustrate the power of recursive definitions and strong induction, let us
analyze Algorithm~\ref{algorithm:binary-search}.
\begin{algorithm}
  \begin{algorithmic}[1]
    \Function{BinarySearch}{$e$, $a_1$, \dots, $a_n$}
      \If{$n \le 5$}
        \For{$i$ from $1$ to $n$}
          \If{$a_i = e$}
            \State\Return{i}
           \EndIf
        \EndFor
      \Else
        \State{$\ell \gets \floor{\frac{n}{2}}$}
        \If{$a_\ell \le e$}
          \State\Call{BinarySearch}{$e$, $a_1$, \dots, $a_\ell$}
        \Else
          \State\Call{BinarySearch}{$e$, $a_{\ell + 1}$, \dots, $a_n$}
        \EndIf
      \EndIf
    \EndFunction
  \end{algorithmic}
  \caption{The binary search algorithm that finds an element $e$ in the sorted
    list $a_1$, \dots, $a_n$.}
  \label{algorithm:binary-search}
\end{algorithm}
We prove that number of comparisons of this algorithm is bounded by
$6 + 2\log_2(n)$. First step of the proof is to denote the worst
number of comparisons when we run the algorithm on the list of length $n$ by
$C(n)$. It is easy to see that $C(n) = n$ for $n \le 5$. Additionally,
$C(n) \le 1 +
  \max(C\left(\floor{\frac{n}{2}}\right),
        C\left(n - \floor{\frac{n}{2}})\right)$
for $n > 5$. As we mentioned we prove that $C(n) \le 6 + 2\log_2(n)$, we prove
it by induction. The base case is clear; let us now prove the induction step.
By the induction hypothesis,
\[
  C\left(\floor{\frac{n}{2}}\right) \le
    6 + 2\log_2\left(\floor{\frac{n}{2}}\right)
\]
and
\[
  C\left(n - \floor{\frac{n}{2}}\right) \le
    6 + 2\log_2\left(n - \floor{\frac{n}{2}}\right),
\]
where $\floor{\alpha}$ denotes the integer part of a real number $\alpha$.
\nomenclature[C]{$\floor{\alpha}$}{denotes the greatest integer less than or
equal to $\alpha$}
Since $\floor{\frac{n}{2}} \le \frac{n}{2}$ and $n - \floor{\frac{n}{2}} \le
\frac{n}{2} + 1$, $C(n) \le 1 + 2\log_2(\frac{n}{2} + 1)$.
However,
\[
    1 + 6 + 2\log_2\left(\frac{n}{2} + 1\right) \le
    6 + 2\log_2\left(\frac{n}{\sqrt{2}} + \sqrt{2}\right) \le 6 + 2 \log_2(n)
\]
for $n \ge 5$. As a result, we proved the induction step.


\begin{chapterendexercises}
    \exercise Show that there does not exist the largest integer.
    \exercise[recommended] Show that for any positive integer $n$, $n^2 + n$ is even.
    \exercise Show that for any positive integer $n$, $3$ divides
        $n^3 + 2n$.
    \exercise Show that for any integer $n \ge 10$,
        $n^3 \le 2^n$.
    \exercise Show that for any positive integer $n$,
        $\sum_{i = 0}^n x^i = \frac{1 - x^{n + 1}}{1 - x}$.
    \exercise Let $a_0 = 2$, $a_1 = 5$, and $a_n = 5a_{n - 1} - 6 a_{n - 2}$
        for all integers $n \ge 2$. Show that $a_n = 3^n + 2^n$ for all integers
        $n \ge 0$.
        \begin{solution}
            We prove this using induction by $n$. The base case for $n \le 1$ is clear
            since $3^0 + 2^0 = 2$ and $3^1 + 2^1 = 5$.

            Let us prove the induction step. Assume that $a_n = 3^n + 2^n$ and
            $a_{n - 1} = 3^{n - 1} + 2^{n - 1}$, we need to prove that
            $a_{n + 1} = 3^{n + 1} + 2^{n + 1}$. Note that
            \begin{multline*}
                a_{n + 1} = 5a_n - 6 a_{n - 1} =
                5 \cdot 3^n + 5 \cdot 2^n - 6 \cdot 3^{n - 1} -
                6 \cdot 2^{n - 1} = \\
                3^{n - 1} \cdot 9 + 2^{n - 1} 4 = 3^{n + 1} + 2^{n + 1}.
            \end{multline*}
        \end{solution}
    \exercise[recommended] Show that $\sum_{i = 1}^n i^2 =
        \frac{n (n + 1)(2n + 1)}{6}$ for all integers $n \ge 1$.
    \exercise Show that $\sum_{i = 1}^n \frac{1}{i (i + 1)} =
        \frac{n}{n + 1}$ for all integers $n \ge 1$.
    \exercise Show that $\sum_{i = 1}^n \frac{1}{i^2} \le
        2 - \frac{1}{n}$ for all integers $n \ge 1$.
    \exercise Show that $\sum_{i = 1}^n (2i - 1) = n^2$ for any positive
        integer $n$.
    \exercise Prove that $\sum_{i = 1}^n \frac{1}{i (i + 1)} =
        \frac{n}{n + 1}$ for any positive integer $n$.
    \exercise Prove that $\sum_{i = 1}^n (i + 1) 2^i = n 2^{n + 1}$
        for all integers $n > 2$.
    \exercise Let $a_1$, \dots, $a_n$ be a sequence of real numbers.
        We define inductively
        $\prod_{i = k}^n a_i$ as follows:
        \begin{itemize}
            \item $\prod_{i = 1}^1 a_i = a_1$ and
            \item $\prod_{i = 1}^{k + 1} a_i =
                \left( \prod_{i = 1}^k a_i \right) \cdot a_{k + 1}$.
        \end{itemize}
        \nomenclature[C]{$\prod_{i = 1}^k \alpha_i$}{denotes $\alpha_1 \cdot
            \ldots \cdot \alpha_k$}

        Prove that
        $\prod_{i = 1}^{n - 1} \left(1 - \frac{1}{(i + 1)^2} \right) =
        \frac{n + 1}{2n}$ for all integers $n > 1$.

    \exercise Let $f_0 = 1$, $f_1 = 1$, and $f_{n + 2} = f_{n + 1} + f_n$ for
        all integers $n \ge 0$. Show that
        $f_n \ge \left(\frac{3}{2}\right)^{n - 2}$.
    \exercise Show that $f_{n + m} = f_{n - 1} f_{m - 1} + f_n f_m$.
    \exercise Show that two arithmetic formulas $(x_1 + x_2) \cdot x_3$ and
        $x_1 \cdot x_3 + x_2 \cdot x_3$ on the variables $x_1$, $x_2$, and $x_3$
        have the same values.
    \exercise Let us define $n!$ as follows: $1! = 1$ and
      $n! = (n - 1)! \cdot n$. Show that $n! \ge 2^n$ for any $n \ge 4$.
      \nomenclature[C]{$\factorial{n}$}{denotes $n \cdot (n - 1) \cdot (n - 2) \cdot \ldots
        \cdot 1$}
    \exercise[open] Fins all the natural numbers $n$ such that $n! = m^2$ for
      some integer $m$.
    \exercise Show that
        $\int\limits_0^{+\infty} x^n e^{- x} ~ \mathrm{d}x = n!$
        for all $n \ge 0$.
    \exercise Prove that $\sum_{i = 1}^n (i + 1) 2^i = n 2^{n + 1}$
        for all integers $n \ge 1$.
    \exercise Show that $\sum_{k = 1}^n k \cdot k! = (n + 1)! - 1$.
    \exercise Show that the algorithm from \Cref{exercise:selection-sort}
      sorts the array.
\end{chapterendexercises}
