\chapter{G\"odel Universal Functions}
It is known that there are algorithms that given a program in one programming
language can produce a program in another programming language.
\marginurl{%
  You can read more about such translators on Wikipedia.
}{en.wikipedia.org/wiki/Source-to-source_compiler}
However, in this book we are talking about universal functions instead of
programming languages. Hence, we may be interested to study the following
problem. Let $U$ and $V$ be (computable) universal functions for the set of all
univariate computable functions. Given $n \in \N$ find $m \in M$ such that $U_m$
and $V_m$ are equal. Unfortunately, not for every pair of universal sets such
$m$ can be found efficiently (see \Cref{theorem:universal-not-godel}). 
However, there is a special class of universal functions that allow to find such
$m$'s efficiently for any computable $V$.
\begin{definition}
  Let $U : \N^2 \to \N$ be a computable universal function for the class of
  univariate computable functions. We say that $U$ is \emph{G\"odel univeral
  function} if for any computable function $V : \N^2 \to \N$, there is a
  computable function $s : \N \to \N$ such that 
  \[
    V(n, x) = U(s(n), x)
  \]
  for all $n, x \in \N$.
\end{definition}

\begin{theorem}
  There is a G\"odel univeral function.
\end{theorem}

\begin{proof}
  We start the proof of the theorem from proving that there is a computable
  function $T : \N^3 \to \N$ that is universal for the set of bivariate
  computable functions. Let us fix a computable bijection 
  $\pair{\cdot}{\cdot} : \N^2 \to \N$. Let $R$ be a universal function for the set of
  univariate computable functions, and let $T(n, u, v) = R(n, \pair{u}{v})$.
  it is easy to see that $T$ is indeed a universal function for the set of
  bivariate computable functions.

  Let $U : \N^2 \to \N$ be the function such that $U(\pair{n}{u}, v) = 
  T(n, u, v)$. We need to show that $U$ is G\"odel univeral function.
  Let us consider some computable function $V : \N^2 \to \N$. There is $n \in
  \N$ such that $V(u, v) = T(n, u, v)$ for all $u, v \in \N$ since $T$ is
  universal. Therefore $U(\pair{n}{u}, v) = V(u, v)$ for all $u, v \in \N$.
  As a result, we can define $s(u)$ to be equal to $\pair{n}{u}$.
\end{proof}

\begin{exercise}
  Show that there is a computable bijection $\pair{\cdot}{\cdot} : \N^2 \to \N$.
\end{exercise}

G\"odel universal funcitons allow us to efficiently operate with numbers of
computable functions.
For example, \Cref{theorem:composition-computable} proved that composition of
two computable functions is computbale. Moreover, it is easy to see that given
the programs computing funcitons $f$ and $g$ we can automatically obtain the
function $g \circ f$.

However, we would like to avoid specifics of programin languages in our study of
computability theory. Our tool to do so is the notion of a universal function so
we need to prove that there is an algorithm that given numbers of any two
computable functions a computes a number of their composition.
\begin{theorem}
  Let $U$ be a G\"odel universal function for the set of univariate computable
  functions. Then there is a total computble function $c : \N^2 \to \N$ such
  that $U(c(p, q), x) = U(p, U(q, x))$ for any $p, q, x \in \N$.
\end{theorem}
\begin{proof}
  Let us consider a computable function $V : \N^2 \to \N$ such that
  $V(\pair{p}{q}, x) = U(p, U(q, x))$.There is a total computable function $s :
  \N \to \N$ such that $U(s(\pair{p}{q}), x) = V(\pair{p}{q}, x)$ since $U$ is a
  G\"odel universal function. Hence, if we define $c(p, q)$ to be equal to
  $s(\pair{p}{q})$, we get that $U(c(p, q), x) = U(p, U(q, x))$.
\end{proof}

Using the notion of G\"odel universal function we can also prove that
constructing the shortest program solving a given problem is not feasible. In
other words let us consider the problem of producing the shortest algorithm
$\Algorithm{A}$ by a given $\Algorithm{B}$ such that
$\Algorithm{A}(x) = \Algorithm{B}(x)$ for any $x \in \N$. Apparently there is
no algorithm that can find such $\Algorithm{A}$.

To prove this we need to formalize what we mean by the shortest algorithm and
how we encode $\Algorithm{A}$.
\begin{theorem}
\label{theorem:program-optimization}
  Let $U$ be a G\"odel universal function, and let $\Optimize : \N \to \N$ be
  the function such that $U_{\Optimize(n)}$ is the same as $U_n$ and $U_m$ and
  $U_n$ are different for any $m < \Optimize(n)$. Then $O$ is not computable.
\end{theorem}

To prove this statement we need the following auxilary result.
\begin{theorem}
\label{theorem:undefined-functions}
  Let $U$ be a G\"odel universal function. 
  Then the set 
  $S = \set[U(n, x) \text{ is not defined for all } x \in \N]{n \in \N}$ is
  not decidable.
\end{theorem}
\begin{proof}
  Let $K$ be a enumerable but undecidable set.
  Consider the partial function $V : \N^2 \to \N$ such that 
  \[
    V(n, x) = 
    \begin{cases}
      0 & \text{if } n \in K \\
      \text{undefined} & \text{otherwise}
    \end{cases}.
  \]
  Note that $V(n, x)$ terminates for some $x \in \N$ iff $n \in K$. 
  Since $U$ is G\"odel universal function there is a computable function $s$
  such that $V(n, x) = U(s(n), x)$. Hence, $s(n) \in S$ iff $n \in K$.
  As a result, $S$ is undecidable.
\end{proof}

\begin{proof}[Proof of \Cref{theorem:program-optimization}]
  Let $S$ be the set from the previous theorem.
  Assume, for the sake of contradiction, that $\Optimize$ is computable. Let
  $n_0$ be the smallest natural number $n$ such that $U(n, x)$ is not defined
  for all $x \in \N$. It is clear that $n \in S$ iff $\Optimize(n) = n_0$.
  Therefore $S$ is decidable, which is a contradiction.
\end{proof}

\Cref{theorem:undefined-functions,theorem:halting,theorem:program-optimization}
proved that several properties of algorithms cannot be computed or verified.
The following theorem says that this is not a coincidence, and essentially any 
nontrivial property of algorithms cannot be verified efficiently.
\begin{theorem}[Rice -- Uspensky]
  Let $\Computable$ be the set of all computable functions, and let
  $\mathcal{P} \subseteq \Computable$ be some nontrivial set of computable
  funcitons ($\mathcal{P} \neq \emptyset$ and $\mathcal{P} \neq \Computable$).
  Let $U : \N^2 \to \N$ be a universal G\"odel function. Then the set 
  $S = \set[U_n \in \mathcal{P}]{n \in \N}$ is undecidable.
\end{theorem}
\begin{proof}
  This theorem can be proved using almost the same method as
  \Cref{theorem:undefined-functions}.

  Let $f : \N \to \N$ be a partial function that is not defined at all $x \in
  \N$. Without loss of generality we may assume that $f \in \mathcal{P}$.
  Let $g : \N \to \N$ be a function from $\Computable \setminus \mathcal{P}$.

  Let $K$ be a enumerable but undecidable set.
  Consider the partial function $V : \N^2 \to \N$ such that 
  \[
    V(n, x) = 
    \begin{cases}
      g(x) & \text{if } n \in K \\
      \text{undefined} & \text{if } n \not\in K 
    \end{cases}.
  \]
  Note that $V_n \not\in \mathcal{P}$ iff $n \in K$. 
  Since $U$ is G\"odel universal function there is a computable function $s$
  such that $V(n, x) = U(s(n), x)$. Hence, $s(n) \not\in S$ iff $n \in K$.
  As a result, $S$ is undecidable.
\end{proof}

Let $U$ be a G\"odel universal funciton.
A simple corollary of the \Cref{theorem:undefined-functions} is that the set 
$\set[U(n, x) \text{ is not defined for all } x \in \N]{n \in \N}$ has
infinitely many elements but it is not equal to the set of natural numbers. This
simple observation allows us to prove that not all universal functions are
G\"odel universal functions.
\begin{theorem}
\label{theorem:universal-not-godel}
  There is a universal function $U : \N^2 \to \N$ such that $U$ is not  G\"odel
  universal function.
\end{theorem}
\begin{proof}
  Let $U$ be a G\"odel universal function. Note that the set 
  $S = \set[U(n, x) \text{ is defined for some } x \in \N]{n \in \N}$ is
  enumerable. Hence, there is a computable bijection $d : \N \to S$.

  Let us consider $V : \N^2 \to \N$ such that $V(i + 1, x) = U(d(i), x)$ for 
  $i, x \in \N$ and $V(1, x)$ undefined for all $x \in \N$. It is clear that $V$
  is computable universal function. However, the set 
  \[  
    \set[V(n, x) \text{ is not defined for all } x \in \N]{n \in \N} = \set{1}
  \]
  cannot be undecidable. As a result, $V$ is not G\"odel universal
  function.\marginurl[3cm]{%
    In fact, Friedberg (Journal of Symbolic Logic 23 (1958), 309-318)
    constructed a universal function such that any computable function has only
    one number; i.e., it is possible to create a programming language such that
    each programming problem has a unique solution in it.
  }{doi.org/10.2307/2964290}
\end{proof}

\begin{chapterendexercises}
  \exercise Let $U : \N^2 \to \N$ be a computable universal function for the
    class of univariate computable functions. Assume that for any universal
    computable function $V : \N^2 \to \N$, there is a computable function 
    $s : \N \to \N$ such that
    \[
      V(n, x) = U(s(n), x)
    \]
    for all $n, x \in \N$.
  \exercise Let $U$ be a G\"odel universal function. Show that for any
    computable function $V : \N^3 \to \N$, there is a total computable function
    $s : \N^2 \to \N$ such that $U(s(m, n), x) = V(m, n, x)$ for all $m, n, x
    \in \N$.
\end{chapterendexercises}
