\chapter{Simple Induction}
\label{chapter:simple-induction}
\section{Proofs by Induction}
\marginurl{%
  The Induction Principle:\\\noindent
  Introduction to Mathematical Reasoning \#4
}{youtu.be/jOnZTWGpX_I}

Let us consider a simple problem: what is bigger $2^n$ or $n$? In this chapter,
we are going to study the simplest way to prove that $2^n > n$ for all positive
integers $n$. First, let us check that it is true for small integers $n$.
\begin{center}
  \begin{tabular}{l l l  l  l  l  l  l  l}
    \toprule
          & 1 & 2 & 3 & 4  & 5  & 6  & 7   & 8   \\
    \midrule
    $n$   & 1 & 2 & 3 & 4  & 5  & 6  & 7   & 8   \\
    $2^n$ & 2 & 4 & 8 & 16 & 32 & 64 & 128 & 256 \\
    \bottomrule
  \end{tabular}
\end{center}
We may also note that $2^n$ is growing faster than $n$, so we expect that if
$2^n > n$ for small integers $n$, then it is true for all positive integers $n$.

In order to prove this statement formally, we use the following principle.
\begin{principle}[The Induction Principle]
  Let $P(n)$ be some statement about a positive integer $n$.
  Hence, $P(n)$ is true for every positive integer $n$ iff
  \begin{description}
    \item [(the base case)] $P(1)$ is true and
    \item [(the induction step)] $P(k) \implies P(k + 1)$ is true
      for all positive integers $k$.
  \end{description}
\end{principle}

Let us prove now the statement using this principle.
We define $P(n)$ be the statement that ``$2^n > n$''.
$P(1)$ is true since $2^1 > 1$. Let us assume now that $2^n > n$. Note that
$2^{n + 1} = 2 \cdot 2^n > 2n \ge n + 1$. Hence, we proved the induction step.


\begin{template}
  \textbf{Template for proving a statement using simple induction.} \\

  We use induction by $n$. Base case for $n = 1$: \emph{present some argument 
  that proves the statement with $n$ replaced by $1$}. 

  Now we need to prove the induction step from $k - 1$ to $k$. Let us assume now
  that the statement is true for $n = k - 1$ for some $k$. \emph{Present some
  argument of the statement with $n$ replaced by $k$ assuming the statement is
  true if we replace $n$ by $k - 1$}. Hence, the statement is true for all $n$
  by the induction principle.
\end{template}


\begin{exercise}
  Prove that $(1 + x)^n \ge 1 + nx$ for all positive integers $n$ and real
  numbers $x \ge -1$.
\end{exercise}
\begin{solution}
  We are going to prove that by induction by $n$.

  The base case is clear. We prove now the induction step. Assume the
  induction hypothesis: $(1 + x)^n \ge 1 + nx$. Note that  $(1 + x)^{n + 1} \ge
  (1 + nx) \cdot (1 + x) = 1 + nx + x + nx^2 \ge 1 + (n + 1)x$. Which proves the
  induction step.
\end{solution}

\section{Changing the Base Case}
Let us consider functions $n^2$ and $2^n$.

\begin{center}
    \begin{tabular}{l  l  l  l  l  l  l  l  l}
        \toprule
              & 1 & 2 & 3 & 4  & 5  & 6  & 7   & 8   \\
        \midrule
        $n^2$ & 1 & 4 & 9 & 16 & 25 & 36 & 49  & 64  \\
        $2^n$ & 2 & 4 & 8 & 16 & 32 & 64 & 128 & 256 \\
        \bottomrule
    \end{tabular}
\end{center}
Note that $2^n$ is greater than $n^2$ starting from $5$. But without some trick
we can not prove this using induction since for $n = 3$ it is not true!

The trick is to use the statement $P(n)$ stating that $(n + 4)^2 < 2^{n + 4}$.
The base case when $n = 1$ is true.
Let us now prove the induction step. Assume that $P(k)$ is true i.e.
$(k + 4)^2 < 2^{k + 4}$. Note that $2(k + 4)^2 < 2^{k + 1 + 4}$ but
$(k + 5)^2 = k^2 + 10k + 25 \le 2k^2 + 16k + 32 = 2(k + 4)^2$.
Which implies that
$2^{k + 1 + 4} > (k + 5)^2$. So $P(k + 1)$ is also true.

In order to avoid this strange $+4$ we may change the base
case and use the following argument.

\begin{theorem}
\label{theorem:induction-shifted-base}
    Let $P(n)$ be some statement about an integer $n$.
    Hence, $P(n)$ is true for every integer $n > n_0$ iff
    \begin{description}
        \item [(the base case)] $P(n_0 + 1)$ is true and
        \item [(the induction step)] $P(k) \implies P(k + 1)$ is true for all
            integers $k > n_0$.
    \end{description}
\end{theorem}

Using this generalized induction principle we may prove that $2^n \ge n^2$ for
$n \ge 4$. The base case for $n = 4$ is true. The induction step is also true;
indeed let $P(k)$ be true i.e. $(k + 4)^2 < 2^{k + 4}$. Hence,
$2(k + 4)^2 < 2^{k + 1 + 4}$ but
$(k + 5)^2 = k^2 + 10k + 25 \le 2k^2 + 16k + 32 = 2(k + 4)^2$.

Let us now prove the theorem. Note that the proof is based on an idea similar
to the trick with $+4$, we just used.
\begin{proof}[Proof of Theorem~\ref{theorem:induction-shifted-base}]
    \begin{description}
        \item[$\Rightarrow$] If $P(n)$ is true for any $n > n_0$ it is also true
            for $n = n_0 + 1$ which implies the base case. Additionally, it true for
            $n = k + 1$ so the induction step is also true.
        \item[$\Leftarrow$] In this direction the proof is a bit harder. Let us
            consider a statement $Q(n)$ saying that $P(n + n_0)$ is true. Note that
            by the base case for $P$, $Q(1)$ is true; by the induction step for $P$
            we know that $Q(n)$ implies $P(n + 1)$. As a result, by the induction
            principle $Q(n)$ is true for all positive integers $n$. Which implies
            that $P(n)$ is true for all integers $n > n_0$.
    \end{description}
\end{proof}

\section{Inductive Definitions}

We may also define objects inductively. Let us consider the sum
$1 + 2 + \dots + n$ a line of dots indicating ``and so on'' which indicates the
definition by induction. In this case, a more precise notation is
$\sum_{i = 1}^n i$.

\begin{definition}
    Let $a(1)$, \dots, $a(n)$, \dots be a sequence of integers. Then
    $\sum_{i = 1}^n a(i)$ is defined inductively by the following
    statements:
    \begin{itemize}
        \item $\sum_{i = 1}^1 a(i) = a(1)$, and
        \item $\sum_{i = 1}^{k + 1} a(i) =
            \sum_{i = 1}^k a(i) + a(k + 1)$.
    \end{itemize}
\end{definition}
\nomenclature[C]{$\sum_{i = 1}^k \alpha_i$}{denotes $\alpha_1 + \dots +
\alpha_k$}

Let us prove that $\sum_{i = 1}^n i = \frac{n (n + 1)}{2}$.
Note that by definition $\sum_{i = 1}^1 i = 1$ and
$\frac{1 (1 + 1)}{2} = 1$; hence, the base case holds. Assume that
$\sum_{i = 1}^n i = \frac{n (n + 1)}{2}$. Note that
$\sum_{i = 1}^{n + 1} i = \sum_{i = 1}^n i + (n + 1)$ and by the
induction hypothesis $\sum_{i = 1}^n i = \frac{n (n + 1)}{2}$.
Hence, $\sum_{i = 1}^{n + 1} i = \frac{n (n + 1)}{2} + (n + 1) =
\frac{(n + 1)(n + 2)}{2}$.

\begin{exercise}
    Prove that $\sum_{i = 1}^n 2^i = 2^{n + 1} - 2$.
\end{exercise}

\section{Analysis of Algorithms with Cycles}

Induction is very useful for analysing algorithms using cycles. Let us extend
the example we considered in Section~\ref{section:simple-algorithm}.

Let us consider the following algorithm.
\begin{algorithm}
  \begin{algorithmic}[1]
    \Function{Max}{$a_1$, \dots, $a_n$}
      \State{$r \gets a_1$}
      \For{$i$ from $2$ to $n$}
        \If{$a_i > r$}
          \State{$r \gets a_i$}
        \EndIf
      \EndFor
      \State\Return{r}
    \EndFunction
  \end{algorithmic}
  \caption{The algorithm that finds the maximum element of $a_1$, \dots, $a_n$.}
\end{algorithm}
We prove that it is working correctly. First, we need to define $r_1$,
\dots, $r_n$ the value of $r$ during the execution of the algorithm.
It is easy to see that $r_1 = a_1$ and
\[
    r_{i + 1} =
    \begin{cases}
        r_i & \text{if } r_i > a_{i + 1} \\
        a_{i + 1} & \text{otherwise}
    \end{cases}.
\]
Secondly, we prove by induction that $r_i$ is the maximum of $a_1$, \dots,
$a_i$. It is clear that the base case for $i = 1$ is true. Let us prove the
induction step from $k$ to $k + 1$. By the induction hypothesis, $r_k$
is the maximum of $a_1$, \dots, $a_k$. We may consider two following cases.
\begin{itemize}
    \item If $r_k > a_{k + 1}$, then $r_{k + 1} = r_{k}$ is the maximum of $a_1$,
        \dots, $a_{k + 1}$ since $r_k$ is the maximum of $a_1$, \dots, $a_k$.
    \item Otherwise, $a_{k + 1}$ is greater than or equal to $a_1$, \dots, $a_k$,
        hence, $r_{k + 1} = a_{k + 1}$.
\end{itemize}

\begin{exercise}
\label{exercise:selection-sort}
    Show that line~6 in the following sorting algorithm executes
    $\frac{n (n + 1)}{2}$ times.

\end{exercise}
\begin{chapterendexercises}
    \exercise Show that there does not exist the largest integer.
    \exercise[recommended] Show that for any positive integer $n$, $n^2 + n$ is even.
    \exercise Show that for any positive integer $n$, $3$ divides
        $n^3 + 2n$.
    \exercise Show that for any integer $n \ge 10$,
        $n^3 \le 2^n$.
    \exercise Show that for any positive integer $n$,
        $\sum_{i = 0}^n x^i = \frac{1 - x^{n + 1}}{1 - x}$.
    \exercise[recommended] Show that $\sum_{i = 1}^n i^2 =
        \frac{n (n + 1)(2n + 1)}{6}$ for all integers $n \ge 1$.
        \begin{solution}
          We prove the statement using induction by $n$. The base case for $n =
          1$ is clear. Let us prove the induction step now. The induction
          hypothesis is 
          $1^2 + 2^2 + 3^2 + \dots + n^2 = \frac{n (n + 1)(2n + 1)}{6}$. 
          Note that
          \begin{multline*}
            1^2 + 2^2 + 3^2 + \dots + n^2 + (n + 1)^2 = \frac{n (n + 1)(2n + 1)}{6} +
            (n + 1)^2 = \\
            (n + 1)\frac{2n^2 + n + 6n + 6}{6} = (n + 1)\frac{(n + 2)(2n + 3)}{6}.
          \end{multline*}
          Hence, $1^2 + 2^2 + 3^2 + \dots + n^2 + (n + 1)^2 = 
            \frac{(n + 1)(n + 2)(2n + 3)}{6}$.
        \end{solution}
    \exercise Show that $\sum_{i = 1}^n \frac{1}{i (i + 1)} = 
      \frac{n}{n + 1}$ for all integers $n \ge 1$.
    \exercise Show that $\sum_{i = 1}^n \frac{1}{i^2} \le 2$ for all integers 
      $n \ge 1$.
      \begin{solution}
        Let us prove a stronger statement:
        \[
          1 + \frac{1}{2^2} + \dots + \frac{1}{n^2} \le 2 - \frac{1}{n}.
        \]

        The base case is clear. We prove now the induction step. By the induction
        hypothesis the following inequality holds
        \[
          1 + \frac{1}{2^2} + \dots + \frac{1}{(n - 1)^2} \le 2 - \frac{1}{n - 1}.
        \]
        Hence,
        \[
          1 + \frac{1}{2^2} + \dots + \frac{1}{n^2} \le 2 - \frac{1}{n - 1} + \frac{1}{n^2} = 2
        \]
        but $\frac{1}{n - 1} - \frac{1}{n^2} \ge \frac{1}{n}$. Indeed,
        $\frac{1}{n - 1} \ge \frac{n + 1}{n^2}$ is equivalent to
        $\frac{1}{(n - 1)^2} \ge \frac{1}{n^2}$ which is true. As a result we proved
        the induction step.
      \end{solution}
    \exercise Show that $\sum_{i = 1}^n (2i - 1) = n^2$ for any positive
        integer $n$.
    \exercise Prove that $\sum_{i = 1}^n \frac{1}{i (i + 1)} =
        \frac{n}{n + 1}$ for any positive integer $n$.
    \exercise Prove that $\sum_{i = 1}^n (i + 1) 2^i = n 2^{n + 1}$
        for all integers $n > 2$.
        \begin{solution}
          First we prove the base case for $n = 1$. Note that 
          $\sum\limits_{i = 1}^1 (i + 1) 2^i = 2 \cdot 2 = 2^2$; hence, the base
          case is true. Let us check the induction step from $k$ to $k + 1$. By
          the induction hypothesis 
          $\sum\limits_{i = 1}^k (i + 1) 2^i = k 2^{k + 1}$.
          It is clear that
          \begin{multline*}
            \sum_{i = 1}^{k + 1} (i + 1) 2^i =
            \sum_{i = 1}^k (i + 1) 2^i + (k + 2) 2^{k + 1} = \\
            k 2^{k + 1} + (k + 2) 2^{k + 1} =
            2 (k + 1) 2^{k + 1} = (k + 1) 2^{k + 2}.
          \end{multline*}
        \end{solution}
    \exercise Let $a_1$, \dots, $a_n$ be a sequence of real numbers.
        We define inductively
        $\prod_{i = k}^n a_i$ as follows:
        \begin{itemize}
            \item $\prod_{i = 1}^1 a_i = a_1$ and
            \item $\prod_{i = 1}^{k + 1} a_i =
                \left( \prod_{i = 1}^k a_i \right) \cdot a_{k + 1}$.
        \end{itemize}
        \nomenclature[C]{$\prod_{i = 1}^k \alpha_i$}{denotes $\alpha_1 \cdot
            \ldots \cdot \alpha_k$}

        Prove that
        $\prod_{i = 1}^{n - 1} \left(1 - \frac{1}{(i + 1)^2} \right) =
        \frac{n + 1}{2n}$ for all integers $n > 1$.
    \exercise Let us define $n!$ as follows: $1! = 1$ and
      $n! = (n - 1)! \cdot n$. Show that $n! \ge 2^n$ for any $n \ge 4$.
      \nomenclature[C]{$\factorial{n}$}{denotes $n \cdot (n - 1) \cdot (n - 2) \cdot \ldots
        \cdot 1$}
    \exercise[open] Find all the natural numbers $n$ such that $n! = m^2$ for
      some integer $m$.
    \exercise Show that
      $\int\limits_0^{+\infty} x^n e^{- x} ~ \mathrm{d}x = n!$
      for all $n \ge 0$.
      \begin{solution}
        We prove the statement using induction by $n$. The base case is for $n =
        0$. It is easy to see that $\int\limits_0^{+\infty} e^{-x} ~ \mathrm{d}x
        = (-e^{-x})\big\rvert_0^\infty = 1$.
 
        Let us prove the induction step from $k$ to $k + 1$. By the induction
        hypothesis, $\int\limits_0^{+\infty} x^k e^{- x} ~ \mathrm{d}x = k!$.
        Note that
        \begin{multline*}
          \int_0^{+\infty} x^{k + 1} e^{- x} ~ \mathrm{d}x =
          -x^{k + 1} e^{- x}\big\rvert_0^\infty +
            \int_0^{+\infty} (k + 1) x^k e^{- x} ~ \mathrm{d}x = \\
          \int_0^{+\infty} (k + 1) x^k e^{- x} ~ \mathrm{d}x = (k + 1)!.
        \end{multline*}
      \end{solution}
    \exercise Show that $\sum_{k = 1}^n k \cdot k! = (n + 1)! - 1$.
    \exercise Show that \Cref{algorithm:selection-sort} executes line~6 exactly
      $\frac{n (n + 1)}{2}$ times.
      \begin{algorithm}
        \begin{algorithmic}[1]
          \Function{SelectionSort}{$a_1$, \dots, $a_n$}
            \For{$i$ from $1$ to $n$}
              \State{$r \gets a_i$}
              \State{$\ell \gets i$}
              \For{$j$ from $i$ to $n$}
                \If{$a_j > r$}
                  \State $r \gets a_j$
                  \State $\ell \gets j$
                \EndIf
              \EndFor
              \State{Swap $a_i$ and $a_\ell$.}
            \EndFor
          \EndFunction
        \end{algorithmic}
        \caption{The algorithm is selection sort, it sorts $a_1$, \dots, $a_n$.}
        \label{algorithm:selection-sort}
      \end{algorithm}
    \exercise Show that \Cref{algorithm:selection-sort} sorts the array.
\end{chapterendexercises}
