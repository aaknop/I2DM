\chapter{Arithmetical Hierarchy}

\Cref{theorem:enumerable-via-computable-funcitons} proves that any decidable set
is the image of some total computable function from $\N$; identifying sets and
predicates we give the following reformulation.
\begin{theorem}
  A predicate$A(x)$ of natural numbers is enumerable iff there is a decidable
  predicate $B(x, y)$ such that the following formula is always true:
  \[
    A(x) \iff \exists y \  B(x, y).
  \]
  (Here and in the sequel, we are going to write $\exists x \  P(x)$ denotes the
  statement ``there is an $x$ from the domain of $P$ such that $P(x)$ is true.)
\end{theorem}

A natural questions can be asked: ``What can be said about other combinations of
quantifiers?''.

It is easy to see that if $A(x)$ is a predicate such that 
\[
  A(x) \iff \exists y \  \exists z \  C(x, y, z),
\]
where $C(x, y, z)$ is decidable, then $A(x)$ is enumerable.
Indeed, let $B(x, \pair{y}{z}) = C(x, y, z)$; then $B(x, w)$ is decidable and 
\[
  A(x) \iff \exists y \  \exists z \  B(x, \pair{y}{z}) \iff 
    \exists w \  B(x, w).
\]
If $A(x)$ is a predicate such that 
\[
  A(x) \iff \forall y \  B(x, y),
\]
where $B(x, y)$ is decidable, then the negation of $A(x)$ is enumerable (we call
such sets and predicates \emph{coenumerable}). Indeed, 
\[
  \lnot A(x) \iff \exists y \  \lnot B(x, y),
\]
and $\lnot B(x, y)$ is decidable since the compliment of a decidable set is
decidable.

This question gives rise to the following definition.
\begin{definition}
  We say that a predicate $A$ on $\N$ belongs to the class $\Sigma_n$ iff there
  is a deciable predicate $B$ on $\N^{n + 1}$ such that 
  \[
    A(x) \iff 
    \exists y_1 \  \forall y_2 \  \exists y_3 \  \dots B(x, y_1, \dots, y_n).
  \]
  Similarly, a predicate $A$ on $\N$ belongs to the class $\Sigma_n$ iff there
  is a deciable predicate $B$ on $\N^{n + 1}$ such that 
  \[
    A(x) \iff 
    \forall y_1 \  \exists y_2 \  \forall y_3 \  \dots B(x, y_1, \dots, y_n).
  \]
\end{definition}

Previous observations can be generalized as follows.
\begin{theorem}
\label{theorem:arythmetic-hierarchy-basics}
  \begin{enumerate}
    \item The set $\Sigma_1$ consists of enumerable sets.
    \item If a predicate belongs to $\Sigma_n$, then its negation belongs to
      $\Pi_n$.
    \item The set $\Sigma_n$ does not change if we allow groups of quantifiers
      of the same type instead of single quantifiers.
  \end{enumerate}
\end{theorem}

Like enumerable and decidable sets, $\Sigma_n$ and $\Pi_n$ sets have several
good properties.
\begin{theorem}
  Union and intersection of two $\Sigma_n$ ($\Pi_n$) sets are also $\Sigma_n$
  ($\Pi_n$) sets.
\end{theorem}
\begin{proof}
  We prove the statement for the union of two $\Sigma_n$ sets, all other cases
  have similar proofs. Let $A_1, A_2 \in \Sigma_n$. By the definition of
  $\Sigma_n$, there are decidable $B_1$ and $B_2$ such that 
  \begin{gather*}
    A_1(x) \iff \exists y_1 \  \forall y_2 \  \dots B_1(x, y_1, \dots, y_n) \\
    \text{and} \\
    A_2(x) \iff \exists z_1 \  \forall z_2 \  \dots B_2(x, z_1, \dots, z_n).
  \end{gather*}
  Therefore 
  \begin{multline*}
      A_1(x) \land A_2(x) \iff \\
      \exists y_1, z_1 \  \forall y_2, z_2 \  \dots 
        B_1(x, y_1, \dots, y_n) \land B_2(x, z_1, \dots, z_n);
  \end{multline*}
  which implies that $A_1 \land A_2$ is a $\Sigma_n$ predicate
  (\Cref{theorem:arythmetic-hierarchy-basics}).
\end{proof}

\begin{exercise}
  Prove that $\Sigma_n \cup \Pi_n \subseteq \Sigma_{n + 1} \cup \Pi_{n + 1}$ for
  any $n \in \N_0$.
\end{exercise}

We can also prove an extension of
\Cref{theorem:m-reduction-complexity-preservation}.
\begin{theorem}
  Let $n \in \N$, and let $A, B \subseteq \N$ such that $A \le_m B$.
  If $B \in \Sigma_n$, then $A \in \Sigma_n$.
\end{theorem}
\begin{proof}
  Let $f$ be the reduction from $A$ to $B$. Let $C$ be a decidable predicate
  such that
  \[
    B(x) \iff \exists y_1 \  \forall y_2 \  \dots C(x, y_1, \dots, y_n).
  \]
  Note that, by the definition of the reduction, 
  \[
    A(x) \iff \exists y_1 \  \forall y_2 \  \dots C(f(x), y_1, \dots, y_n);
  \]
  hence, $A(x) \in \Sigma_n$ since $C(f(x), y_1, \dots, y_n)$ is decidable.
\end{proof}


\begin{exercise}
  Prove that if a set $A \in \Sigma_n$, then $A \times A \in \Sigma_n$.
\end{exercise}

We defined infinitely many classes of sets; however, we have not shown that all
these sets are different (except $\Sigma_0$ and $\Pi_0$ which are equal to the
class of all decidable sets). Like in the case of enumebrable sets, to prove
this we need to use universal sets.
\begin{theorem}
  For any $n \in \N$, there is a set $U \in \Sigma_n$ universal for $\Sigma_n$
  subsets of $\N$.
\end{theorem}
\begin{proof}
  We prove the statement using induction by $n$. The base case for $n = 1$ is
  true by
  \Cref{theorem:arythmetic-hierarchy-basics,theorem:universal-set-enumerable}.

  Let us prove the induction step. Assume $V \in \Sigma_n$ is a universal set
  for $\Sigma_n$ subsets of $\N$. First we show that $(\N \setminus V) \in \Pi_n$
  is a universal set for $\Pi_n$ subsets of $\N$. Let $A \in \Pi_n$. Then there
  is $k \in \N$ such that $\N \setminus A = V_k$ since $(\N \setminus A) \in
  \Sigma_n$. Therefore, $A = \N \setminus V_k$ which implies that 
  $\N \setminus V$ is universal for $\Pi_n$ subsets of $\N$.
  
  
  Note that there is $W' \in \Pi_n$ that is universal for $\Pi_n$ subsets
  of $\N^2$. Indeeed, let $W' = \set[(n, \pair{x}{y}) \in W]{(n, x, y) \in \N^3}
  \in \Sigma_n$.
  Let $A' \in \Pi_n$ and $B'$ be decidable set such that
  \[
    A'(x, y) \iff \exists z_1 \  \forall z_2 \  \dots B'(x, y, z_1, \dots, z_n).
  \]
  Consider $A \subseteq \N$ such that 
  \[
    A(\pair{x}{y}) \iff \exists z_1 \  \forall z_2 \  \dots 
      B'(x, y, z_1, \dots, z_n);
  \]
  it is clear that $A \in \Pi_n$. Hence, there is $k \in \N$ such that 
  $W_k = A$, which implies that $W'_n = A'$. Therefore $W'$ is universal for
  $\Pi_n$ subsets of $\N^2$.

  Let us consider the set 
  $U = \set[\exists y \  (k, x, y) \in W']{(k, x) \in \N^2}$. We claim that it is
  universal for $\Sigma_{n + 1}$ subsets of $\N$.
  Let $A \in \Sigma_{n + 1}$. Then there is $B \in \Pi_n$ such that 
  \[
    A(x) \iff \exists y \  B(x, y).
  \]
  In addittion, there is $k \in n$ such that $W'_k = B$. Therefore, 
  \[
    A(x) \iff \exists y \  W'(k, x, y) \iff U(k, x).
  \]
  Hence, $U$ is universal for $\Sigma_{n + 1}$ subsets of $\N$.
\end{proof}

We proved that the compliment of an enumerable universal set for the class of
enumerable sets is not enumerable. Similar result can be shown for $\Sigma_n$.
\begin{theorem}
  Let $n \in \N$ and let $W \subseteq \N^2$ be universal for $\Sigma_n$ subsets
  of $\N$. Then $W \not\in \Pi_n$.
\end{theorem}
\begin{proof}
  Assume that $W \in \Pi_n$, this implies that that the set 
  $D = \set[(n, n) \not\in W]{n \in \N}$ belongs to $\Sigma_n$.
  Note that there is $k \in \N$ such that $W_k = D$ since $W$ is universal for
  $\Sigma_n$ subsets if $\N$. However, $W_k(k)$ is true only if $W_k(k)$ is
  false, which is a contradiction. Therefore, $W \not\in \Pi_n$.
\end{proof}

\begin{corollary}
  For any $n \in \N$, $\Pi_n \not\subseteq \Sigma_{n + 1}$ and 
  $\Sigma_n \not\subseteq \Pi_{n + 1}$.
\end{corollary}

\begin{chapterendexercises}
  \exercise Prove that if $A, B \in \Sigma_n$, then 
    $A \setminus B \in \Sigma_{n + 1} \cap \Pi_{n + 1}$.
\end{chapterendexercises}
