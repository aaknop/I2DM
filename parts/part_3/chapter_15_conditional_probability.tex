\chapter{Conditional Probability}
In general, the occurrence of an event $B$ changes the probability that another
event $A$ occurs, $\Pr(A \cond B)$ denotes the latter probability. More
formally, $\Pr(A \cond B) = \Pr(A \cap B) / \Pr(B)$, where $(\Omega, \Pr)$ is
finite discrete probability space, $A, B \subseteq \Omega$, and $\Pr(B) \neq 0$. 
We say that $\Pr(A \cond B)$ is the conditional probability that $A$ occurs
given that $B$ occurs.

For example, let us consider $\Omega = [6]^2$ and uniform distribution $\Pr$ on
$\Omega$; i.e., we consider an experiment consisting of rolling two dice. Let
us compute the probability that the sum of numbers on the dice exceeds $6$
given that the first dice's number is $3$. In other words we need to compute
$\Pr(A \cond B)$, where $A = \set[i + j > 6]{(i, j) \in [6]^2}$ and 
$B = \set[{j \in [6]}]{(3, j)}$. It is clear that $\Pr(B) = 1 / 6$ and 
$\Pr(A \cap B) = \set{(3, 4), (3, 5), (3, 6)} = 1 / 12$. Hence, $\Pr(A \cond B)
= 1 / 2$.

\begin{exercise}
  A family has two children. What is the probability that both are boys, given
  at least one is a boy? What if it is given that \emph{the first child} is a
  boy. (You assume that the probability distribution of families is uniform.)
\end{exercise}

Let us consider another example known as ``Monty Hall Problem''. On the
television game \emph{Let’s make a deal}, a contestant is presented with a
choice of three closed doors. Behind exactly one door is a prize; the other
doors conceal cheap items. First, the contestant is asked to choose a door. Then
Monty Hall, the host of the show,  shows the contestant one of the worthless
prizes behind one of the other doors. At this point, there are two closed
doors, and the contestant is given the opportunity to switch from his original
choice to the other closed door. The question is, is it better for the
contestant to stick to his original choice or to switch doors?


Let us analyze this question using the conditional probabilities. Without loss
of generality, we may assume that the contestant chooses door $1$.
Note that the sample space is equal to $\set{(1, 2), (1, 3), (2, 3), (3, 2)}$,
where the first number denotes the door with the prize and the second number
denotes the door opened by the host. The probability distribution is equal to 
\[
  \Pr(x) = 
  \begin{cases}
    1 / 6 & x = (1, 2) \\
    1 / 6 & x = (1, 3) \\
    1 / 3 & x = (2, 3) \\
    1 / 3 & x = (3, 2)
  \end{cases}
\]
since in the first two cases Monty has two possible choices to show a door
without the prize.
Suppose the host revealed the door number $2$ (the probability of this is
$3 / 6$). Then the probability that we win the price if we stick to the original
choice is $(1 / 6) / (3 / 6) = 1 / 3$. However, the probability to win the prize
in case of us swithcing the door is $(1 / 3) / (3 / 6) = 2 / 3$. Which implies,
paradoxically, that it is beneficial to switch the door!

\begin{theorem}[Bayes’ Rules]
  Let $(\Omega, \Pr)$ be a finite discrete probability space. 
  \begin{itemize}
    \item Let $A, B \subseteq \Omega$ be two events such that $\Pr(A) > 0$ and
      $\Pr(B) > 0$. Then $\Pr(A \cond B) = \frac{\Pr(B \cond A) \Pr(B)}{\Pr(A)}$.
    \item Let $A, B \subseteq \Omega$ be two events such that $\Pr(A) < 1$ and
      $\Pr(B) < 1$; i.e., $\Pr(\bar{A}) > 0$ and $\Pr(\bar{A}) > 0$, where
      $\bar{A} = \Omega \setminus A$ and $\bar{B} = \Omega \setminus B$.
      Then $\Pr(A) = \Pr(A \cond B)\Pr(B) + \Pr(A \cond \bar{B})\Pr(\bar{B})$.
  \end{itemize}
\end{theorem}


Usefulness of this result can be illustrated with the following example. Assume
that there is a rare disease that has the property that if a patient is affected
by the disease, then the test is positive in $99\%$ of the cases. However, it
happens in $2\%$ of the cases that a healthy patient tests positive. Statistical
data shows that one person out of $1000$ has the disease. What is the
probability for a patient with a positive test to be affected by the disease?

Let $S$ be the event that the patient has the disease, and $P$ and $N$ the
events that the test is positive or negative. We know that $\Pr(S) = 0.001$,
$\Pr(P \cond S) = 0.99$, and $\Pr(P \cond S) = 0.02$, where $\bar{S}$ is the
event that the patient does not have the desease. Therefore $\Pr(S \cond P) =
\frac{\Pr(P \cond S) \Pr(S)}{\Pr(P)}$ and $\Pr(P) = \Pr(P \cond S) \Pr(S) +
\Pr(P \cond \bar{S}) \Pr(\bar{S})$. As a result $\Pr(S \cond P) = 
\frac{0.99 \cdot 0.001}{0.99 \cdot 0.001 + 0.02 \cdot 0.999} \approx
\frac{1}{20}$.
