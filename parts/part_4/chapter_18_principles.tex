\chapter{Counting Principles}
\label{chapter:principles}
\marginurl{%
  Counting Principles:\\\noindent
  Introduction to Combinatorics \#2
}{youtu.be/dAoperLCjb8}
\section{The Additive Principle}
The first principle is called \emph{additive} principle and it states that if
you have two disjoint sets, then their union have size equal to the sum of their
sizes.

A simple illustration of this statement is the following. Assume you have three
pencils and two pens; how many ways to choose a writing accessory. According to
this principle the answer is $2 + 3 = 5$.
\begin{theorem}[The Additive Principle]
\label{theorem:additive-principle}
  Let $X$ and $Y$ be finite sets. If $X \cap Y = \emptyset$, then
  $\cardinality{X \cup Y} = \cardinality{X} + \cardinality{Y}$.
\end{theorem}


\begin{corollary}
\label{corollary:additive-principle}
  Let $X_1, \dots, X_n$ be some pairwise disjoint sets. Then
  $\cardinality{\bigcup_{i = 1}^n X_i} = \sum_{i = 1}^n \cardinality{X_i}$.
\end{corollary}

\begin{exercise}
  Prove Corollary~\ref{corollary:additive-principle}.
\end{exercise}


\section{The Multiplicative Principle}
The next principle is called the \emph{multiplicative} principle and it can be
illustrated as follows: imagine that you are given two postal stamps and
three envelopes, how many ways are there to pack the letters? The answer is obviously
$2 \cdot 3 = 6$.
\begin{theorem}[The Multiplicative Principle]
\label{theorem:multiplicative-principle}
  Let $X$ and $Y$ be finite sets. Then $\cardinality{X \times Y} =
  \cardinality{X} \times \cardinality{Y}$.
\end{theorem}

\begin{exercise}
  Assume you have $5$ types of envelopes and $6$ types of postal stamps.
  \begin{enumerate}
    \item How many ways to put one stamp on one envelope?
    \item How many ways to put two stamps on one envelope?
  \end{enumerate}
\end{exercise}

By analogy with unions and intersections of many sets we can define the cross
product of many sets.
Let $X_1$, \dots, $X_n$ be some sets. Then $\bigtimes_{i = 1}^1 X_i = A_1$ and
$\bigtimes_{i = 1}^{k + 1} X_i =
  \left(\bigtimes_{i = 1}^k X_i\right) \times X_{k + 1}$\footnote{%
    Note that cross product is not associative and different definitions of the
    product of several sets are not equivalent. However, the bijection
    constructed in the previous section allow us to think about these
    definitions as if they are equivalent.
}.

\begin{corollary}
\label{corollary:multiplicative-principle}
  Let $X_1, \dots, X_n$ be some finite sets. Then
  $\cardinality{\bigtimes_{i = 1}^n X_i} = \prod_{i = 1}^n \cardinality{X_i}$.
\end{corollary}

\begin{exercise}
  Prove Corollary~\ref{corollary:multiplicative-principle}.
\end{exercise}

\begin{theorem}
\label{theorem:cardinality-of-power-set}
  For any set $X$, $\cardinality{\subsets{X}} = 2^{\cardinality{X}}$.
\end{theorem}
\begin{proof}
  By Corollary~\ref{corollary:power-set-and-set-of-binary-strings},
  $\cardinality{\subsets{X}} =
  \cardinality{\functions{\cardinality{X}}{\set{0, 1}}}$, so it is enough to
  prove that $\cardinality{\set{0, 1}^{\cardinality{X}}} = \subsets{\cardinality{X}}$. 
  This statement is true by \Cref{corollary:multiplicative-principle} since 
  $\cardinality{\set{0, 1}^{\cardinality{X}}} =
  \prod_{i = 1}^{\cardinality{X}} \cardinality{\set{0, 1}} = 
    2^{\cardinality{X}}$.
\end{proof}

\section{The Inclusion-exclusion Principle}

The last principle we are going to discuss in this chapter is the
inclusion-exclusion principle which helps us to find the size of the union
of sets when they are not disjoint.
\begin{theorem}[The Inclusion-exclusion Principle]
\label{theorem:inclusion-exclusion-principle}
  Let $X$ and $Y$ be finite sets. Then $\cardinality{X \cup Y} = \cardinality{X}
  + \cardinality{Y} - \cardinality{X \cap Y}$.
\end{theorem}
\begin{proof}
  Note that $X \cup Y = (X \setminus Y) \cup (Y \setminus X) \cup (X \cap Y)$.
  Hence, $\cardinality{X \cup Y} = \cardinality{X \setminus Y} + 
  \cardinality{Y \setminus X} + \cardinality{X \cap Y}$. But it
  is possible to note that $\cardinality{Y \setminus X} + \cardinality{X \cap Y}
  = \cardinality{Y}$ and $\cardinality{X \setminus Y} + \cardinality{X \cap Y} =
  \cardinality{X}$.
\end{proof}

\begin{corollary}
\label{corollary:inclusion-exclusion-principle}
  Let $X_1, \dots, X_n$ be some finite sets. Then
  \[
    \cardinality{\bigcup_{i = 1}^n X_i} =
    \sum_{S \subseteq \range{n} ~:~ S \neq \emptyset} 
      (-1)^{\cardinality{S} + 1} \cardinality{\bigcap_{i \in S} X_i}.
  \]
\end{corollary}
\begin{proof}
  As always, we prove this statement using induction by $n$. The base case for
  $n = 2$ is true by Theorem~\ref{theorem:inclusion-exclusion-principle}.

  By the induction hypothesis,
  \[
    \cardinality{\bigcup_{i = 1}^k X_i} =
    \sum_{S \subseteq \range{k} ~:~ S \neq \emptyset} 
      (-1)^{\cardinality{S} + 1} \cardinality{\bigcap_{i \in S} X_i}.
  \]
  In addition, by Theorem~\ref{theorem:inclusion-exclusion-principle},
  \[
    \cardinality{\bigcup_{i = 1}^{k + 1} X_i} =
    \cardinality{\bigcup_{i = 1}^k X_i} + \cardinality{X_{k + 1}} -
      \cardinality{\left(\bigcup_{i = 1}^k X_i\right) \cap X_{k + 1}}.
  \]
  We need to simplify two elements of the sum on the right of the equality.
  By the induction hypothesis,
  \[
    \cardinality{\bigcup_{i = 1}^k X_i} =
    \sum_{S \subseteq \range{k} ~:~ S \neq \emptyset}
        (-1)^{\cardinality{S} + 1} \cardinality{\bigcap_{i \in S} X_i}.
  \]
  In addition, it is easy to note that
  \[
    \cardinality{\left(\bigcup_{i = 1}^k X_i\right) \cap X_{k + 1}} =
    \cardinality{\bigcup_{i = 1}^k \left(X_i \cap X_{k + 1}\right)}.
  \]
  Thus using the induction hypothesis,
  \begin{multline*}
    \cardinality{\left(\bigcup_{i = 1}^k X_i\right) \cap X_{k + 1}} = \\
    \sum_{S \subseteq \range{k} ~:~ S \neq \emptyset}
      (-1)^{\cardinality{S} + 1} \cardinality{\bigcap_{i \in S} (X_i \cap X_{k + 1})} = \\
    \sum_{S \subseteq \range{k + 1} ~:~ (k + 1) \in S \text{ and } S \neq \set{k + 1}}
          (-1)^{\cardinality{S}} \cardinality{\bigcap_{i \in S} X_i}.
  \end{multline*}
  As a result,
  \[
    \cardinality{X_{k + 1}} -
      \cardinality{\left(\bigcup_{i = 1}^k X_i\right) \cap X_{k + 1}} =
    \sum_{S \subseteq \range{k + 1} ~:~ (k + 1) \in S}
          (-1)^{\cardinality{S} + 1} \cardinality{\bigcap_{i \in S} X_i}.
  \]
  Which implies that
  \begin{multline*}
    \cardinality{\bigcup_{i = 1}^{k + 1} X_i} =
    \sum_{S \subseteq \range{k} ~:~ S \neq \emptyset}
      (-1)^{\cardinality{S} + 1}\cardinality{\bigcap_{i \in S} X_i} + \\
    \sum_{S \subseteq \range{k + 1} ~:~ (k + 1) \in S}
          (-1)^{\cardinality{S} + 1} \cardinality{\bigcap_{i \in S} X_i} = \\
    \sum_{S \subseteq \range{k + 1} ~:~ S \neq \emptyset}
      (-1)^{\cardinality{S} + 1} \cardinality{\bigcap_{i \in S} X_i}.
  \end{multline*}
\end{proof}



\begin{chapterendexercises}
  \exercise
    Find the cardinality of the set
    \[
      \set[{x, y \in \range{9} \text{ and } x \neq y}]{(x, y)}.
    \]
  \exercise Find the number of ordered pairs $(A, B)$ of subsets of $\range{n}$
    such that $A \cap B = \emptyset$ (the answer is not supposed to contain the
    summation sign).
  \exercise How many functions from $\set{0, 1}^n$ to $\set{0, 1}$?
  \exercise[recommended] How many numbers from $\range{999}$ are not divisible neither by $3$,
    nor by $5$, nor by $7$.
    \begin{solution}
      Let $D_n = \set[i \text{is divisible by n}]{i \in \range{999}}$.
      Note that $D_3 \cap D_5 = D_{15}$, $D_3 \cap D_7 = D_{21}$, $D_5 \cap D_7
      = D_{35}$, and finally, $D_3 \cap D_5 \cap D_7 = D_{105}$. Additinally,
      $\cardinality{D_3} = 999 / 3 = 333$,
      $\cardinality{D_5} = \floor{999 / 5} = 199$,
      $\cardinality{D_7} = \floor{999 / 7} = 142$,
      $\cardinality{D_{15}} = \floor{999 / 15} = 66$,
      $\cardinality{D_{21}} = \floor{999 / 21} = 47$,
      $\cardinality{D_{35}} = \floor{999 / 35} = 28$, and
      $\cardinality{D_{105}} = \floor{999 / 105} = 9$.
      As a result, by the inclusion-exclusion principle, the answer is
      $999 - 333 - 199 - 142 + 66 + 47 + 28 - 9 = 457$.
    \end{solution}
  \exercise How many numbers $x$ from $1$ to $999$ such that at least one
    of the digits of $x$ is $7$?
  \exercise How many numbers $x$ from $1$ to $999$ such that exactly one
    of the digits of $x$ is $7$?
  \exercise Let $A$, $B$ be some finite sets such that $A \subseteq B$.
    Show that $\cardinality{B \setminus A} = \cardinality{B} - \cardinality{A}$.
  \exercise King Athur invited $n$ knights for a feast. Each knight has an enemy
    (being an enemy is a mutual relation; i.e., if $A$ is an enemy of $B$, then
    $B$ is an enemy of $A$). How many wasy for knights to seat around the Round
    Table so that enemies are not sitting together?
  \exercise[recommended] Let $n$ be some positive integer.
    Find the cardinality of the set
    \[
      \set[{A, B \subseteq \range{n} \text{ and } A \cap B \neq \emptyset}]{(A, B)}?
    \]
  \exercise Let $X$ and $Y$ be some finite sets, and $f : X \to Y$ be a function
    such that $\cardinality{f^{-1}(y)} = k$ for all $y \in Y$. Prove that
    $\cardinality{X} = k\cardinality{Y}$.
  \exercise[recommended] Show that if $U$ and $X_1, \dots, X_n \subseteq U$ are
    some finite sets, then
    \[
      \cardinality{\bigcap_{i = 1}^n X_i} =
      \sum_{S \subseteq \range{n}} (-1)^{\cardinality{S}}
        \cardinality{\bigcap_{i \in S} \overline{X}_i},
    \]
    where $\overline{X}_i = U \setminus X_i$ and
    $\bigcap_{i \in \emptyset} \overline{X}_i = U$.
\end{chapterendexercises}
