\chapter{Matchings}
Every year universities are accepting students. Imagine that we want to develpe
an optimal procedure to do this. First, we formalize the
problem.
\begin{itemize}
  \item Each student has a list of universities she likes, and
    moreover, she has an ordering of preferences of the universities.
  \item Additionally, each university knows the students is would
    like to accept, and has an ordering of preferences of the students. In
    addition all the universities know how many students they wish to accept.
\end{itemize}
It is easy to see that we may define several ways to define what is the optimal
way to allocate students to universities. In the end of the section we discuss,
probably, the most famous one.

Now let us simplify the problem and assume that each university accept only one
student, that for each student all the universities he likes are equally good,
and that for each univeristy all the students the would like to accept are
equally good. Note that in this case the problem can be decribed using a graph
where vertices are students and universities, and there is an edge between a
univeristy $U$ and a student $S$ iff $S$ likes $U$ and $U$ would like to accept
$S$. We need to pair some universities and some students so that there is an
edge within the pairs.
\begin{definition}
  Let $G = (V, E)$ be a graph. We say that $M \subseteq E$ is a matching in $G$
  iff in the graph $G[M]$ degrees of all the vertices are at most $1$, i.e.,
  all the edges in $M$ have different vertices. We say that $M$ is perfect
  iff every vertex in $G[M]$ has degree equal to $1$.

  We also say that a matching $M$ covers a set $U \subseteq V$ iff all the
  vertices from $U$ have degree $1$ in $G[M]$.
\end{definition}
Using this definition we may notice that the problem of accepting stundets is a
problem of finding a matching.

\section{Bipartite Graphs}
In the example we discussed the graph is bipartite. So let us first study
matchings in bipartite graphs.

\begin{definition}
  Let $G = (A \cup B, E)$ be a bipartite graph with parts $A$ and $B$. We
  say that a matching $M$ is a perfect matching of $A$ into $B$ if
  degrees of all the vertices from $A$ have degree $1$ in $G[M]$.
\end{definition}

The question we are intrested is whether exists a perfect maching of $A$ into
$B$ for a bipartite graph $G$ with parts $A$ and $B$. A famous Hall's marriage
theorem gives a way to answer this question.
\begin{theorem}[Hall's marriage theorem]
  Let $G = (A \cup B, E)$ be a bipartite graph with parts $A$ and $B$,
  and
  \[
    N(X) = \set[{(u, v) \in E \text{ for some } u \in X}]{v \in B}.
  \]

  Then $G$ has a perfect matching of $A$ into $B$ iff for any $X \subseteq A$,
  $|N(X)| \ge |X|$.
\end{theorem}


\section{Unrestricted Graphs}

\begin{theorem}[Tutte's theorem]
  Let $G = (V, E)$ be a grpah, and $c_0(H)$ be the number of components of a
  graph $H$ that have an odd number of vertices.
  The graph $G$ has a perfect matching iff, for all subsets
  $S \subseteq V$, the inequality $c_0(G - S) \le |S|$ holds.
\end{theorem}
