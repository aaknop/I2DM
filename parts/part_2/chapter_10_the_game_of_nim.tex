\chapter{The Game of Nim}

This chapter discusses probably the most famous combinatorial game, the game
of \emph{Nim}.
In this game there are several piles of chips on the table. On each turn
the current player may remove some number of chips from \emph{one} of the piles;
however, the player should remove \emph{at least one chip}.
We say that a game of Nim is a $k$-pile game of Nim if there are $k$ piles.
\marginurl{%
  You can play Nim on this website
}{dotsphinx.com/games/nim/}

We start from analysis of the game when we have one pile of chips. It is clear
that the first player to move wins since he/she may remove all the chips.

Consider a more complicated case when we have two piles of size $n$ and $m$
respectively. We need to consider two cases:
\begin{enumerate}
  \item If $n = m$, then the second player to move wins. Indeed, we can
    use the symmetric strategy; i.e., if the first player removes $s$
    chips from one pile we also remove $s$ chips from the other pile.
    It is clear that we can always make a move as long as the first player can.
  \item Otherwise, the first player wins because it can move to the state
    with two equal piles.
\end{enumerate}

The case of three piles is even more complicated. So we spend the rest of the
chapter studying it.

\section{Nim Sum}

We start from a definition of the XOR operation
$\lxor : \set{0, 1} \times \set{0, 1} \to \set{0, 1}$,
also known as ``exclusive or''), this operation is defined as follows:
$a \lxor b = 1$ iff $a \neq b$.

It is well-known that any number $n \in \N_0$
can be represented as a binary number ($\N_0$ denotes nonnegative integers);
\nomenclature[S]{$\N_0$}{denotes the set of nonnegative integers}
we write $n = (a_\ell, \dots, a_0)_2$ if $n = \sum_{i = 0}^\ell a_i 2^i$.
For example,
$5 = 4 + 1 = 1 \cdot 2^2 + 0 \cdot 2^1 + 1 \cdot 2^0 = (1, 0, 1)_2$ and
$6 = 4 + 2 = 1 \cdot 2^2 + 1 \cdot 2^1 + 0 \cdot 2^0 = (1, 1, 0)_2$.
So we can define the Nim sum $\bitwisexor : \N_0 \times \N_0 \to \N_0$, also
known as bitwise xor, as follows:
$(a_\ell, \dots, a_0)_2 \bitwisexor (b_\ell, \dots, b_0)_2 =
    (a_\ell \lxor b_\ell, \dots, a_0 \lxor b_0)_2$.
For example, $5 \bitwisexor 6 = (1, 0, 1)_2 \bitwisexor (1, 1, 0)_2 =
(1 \lxor 1, 0 \lxor 1, 1 \lxor 0)_2 = (0, 1, 1)_2$.
\nomenclature[N]{$\bitwisexor$}{denotes the Nim sum}

\begin{exercise}
  Show that $a \bitwisexor (b \bitwisexor c) = (a \bitwisexor b) \bitwisexor c$
  for any $a, b, c \in \N_0$.
\end{exercise}
Hence, we are going to write $a \bitwisexor b \bitwisexor c$ instead of
$a \bitwisexor (b \bitwisexor c)$ and $(a \bitwisexor b) \bitwisexor c$.

\section{Bouton's Theorem}

Now we may notice that $a \bitwisexor b = 0$ iff $a = b$. So
our result about $2$-pile Nim can be rephrased:
a position $(a, b)$ in the $2$-pile Nim is a P-position iff
$a \bitwisexor b = 0$. Which leads us to the next theorem.
\begin{theorem}[Bouton]
\label{theorem:bouton}
  A position $(a, b, c)$ in the $3$-pile Nim is a P-position iff
  $a \bitwisexor b \bitwisexor c = 0$
\end{theorem}
\begin{proof}
  We prove the statement using structural induction.
  First note that the only terminal position the $3$-pile Nim is $(0, 0, 0)$
  and $(0, 0, 0)$ and $0 \bitwisexor 0 \bitwisexor 0 = 0$.

  Let us consider
  some $(a, b, c)$ such that $a \bitwisexor b \bitwisexor c \neq 0$.
  We need to show that there is a move from this position to a P-position.
  Let $a \bitwisexor b \bitwisexor c =
      (0, \dots, 0, 1, r_{k - 1}, \dots, r_0)_2$. So among $a$, $b$, and $c$
  there is a number that has $1$ in the $k$th position.
  Note that without loss of generality
  $a = (p_\ell, \dots, p_{k + 1}, 1, p_{k - 1}, \dots, p_0)_2$.
  Consider $a' = (p_\ell, \dots, p_{k + 1}, 0,
    p_{k - 1} \lxor r_{k - 1}, \dots, r_0 \lxor p_0)_2$. It is clear that
  $a' < a$ and $a' \bitwisexor b \bitwisexor c = 0$.
  Hence, $(a', b, c)$ is a P-position and therefore, $(a, b, c)$ is an
  N-position.

  Finally, let us consider $(a, b, c)$ such that
  $a \bitwisexor b \bitwisexor c = 0$. Assume that there is a move to a
  position $(a', b, c)$ such that $a' \bitwisexor b \bitwisexor c = 0$.
  This implies that
  $(a' \bitwisexor b \bitwisexor c) \bitwisexor
      (a \bitwisexor b \bitwisexor c) =  a \bitwisexor a' = 0$, whence $a = a'$.
\end{proof}
