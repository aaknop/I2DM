\chapter{Universal Functions}
It is known that we may write a program that gets another program as an argument
and run it. (Such programs are known as interpreters.) To use this observation
we give the following definition and theorem.
\begin{definition}
  We say that a function $U$ is a \emph{universal function} (for the set of
  univariate computable functions) iff for each $n \in \N$,
  \[
    U_n : x \mapsto U(n, x)
  \]
  (we say that $U_n$ is a section of $U$)
  is computable and any univariate computable function is among $U_n$'s.
\end{definition}

\begin{theorem}
\label{theorem:universal-function-computable}
  There is a computable universal function $U$.
\end{theorem}

\begin{exercise}
  Assume that every section of a function $U$ is computable.
  Is it necessarily true that $U$ is computable?
\end{exercise}

Similarly to the notion of universal function we may define the notion
of universal sets.
\begin{definition}
  Let $F \subseteq 2^\N$. We say that $W \subseteq \N^2$ is universal for $F$
  if $W_n = \set[(n, x) \in F]{x \in \N}$ is an element of $F$ for all
  $n \in \N$ and any set $S \in F$ is among $W_n$'s.
\end{definition}

\begin{theorem}
\label{theorem:universal-set-enumerable}
  There is an enumerable set $W$ such that it is universal for the set of
  all enumerable subsets of $\N$.
\end{theorem}

\section{Enumerable but Not Decidable Set}
\begin{theorem}
\label{theorem:enumerable-not-decidable}
  There is a set $S \subseteq \N$ such that $S$ is enumerable but it is not
  decidable.
\end{theorem}
\begin{proof}
  Let $U$ be be a universal computable function, it exists by
  \Cref{theorem:universal-function-computable}.
  To prove the statement we are going to use the diagonalization method.
  Let us consider $S = \set[U(n, n) = 0]{n \in \N}$.

  It is easy to see that $S$ is enumerable. Assume for the sake of
  contradiction that $S$ is decidable. Let $\Algorithm{A}$ be the algorithm
  deciding $S$. There is $n \in \N$ such that $\Algorithm{A}$ computes $U_n$
  since $U$ is universal. Let us now consider two following cases.
  \begin{enumerate}
    \item Assume that $n \in S$. In this case $\Algorithm{A}(n) = 1$ since
      $\Algorithm{A}$ decides $S$. However, $U_n(n) \neq 1$ by the definition of
      $S$. These two equalities together leads us to a contradiction since
      $\Algorithm{A}$ computes $U_n$.
    \item Assume that $n \notin S$. In this case $\Algorithm{A}(n) = 0$ since
      $\Algorithm{A}$ decides $S$. However, $U_n(n) = 1$ by the definition of
      $S$. These two equalities together leads us to a contradiction since
      $\Algorithm{A}$ computes $U_n$.
  \end{enumerate}
\end{proof}

Using the proof of this result we can prove the following surprising
observation.
\begin{theorem}
\label{theorem:halting}
  Let $U$ be a universal function.
  Let $\Halting : \N^2 \to \set{0, 1}$ be the function such that
  $\Halting(n, x) = 1$ iff $U_n(x)$ is defined. Then $\Halting$ is not
  computable.
\end{theorem}
Informally, this theorem says that it is impossible to check whether a given
algorithm terminates or not on some input.


\section{Diagonalization Method}
This section will give several other applications of the diagonalization method.

In the previous section we constructed a computable universal function for
the set of computable functions of one variable. Now we can prove that
it is impossible for total functions.
\begin{theorem}
    There is no computable universal total function for
    the set of computable total functions of one variable.
\end{theorem}
\begin{proof}
  Assume that such a function $U$ exists. Let us consider the total computable
  function $d : \N \to \N$ such that $d(n) = U(n, n) + 1$. Since $U$ is a
  computable universal total function for the set of computable total functions
  of one variable, there is $m \in \N$ such that $d(n) = U(m, n)$ for any
  $n \in \N$. Note that this implies that $U(m, m) = d(m) = U(m, m) + 1$,
  which is a contradiction.
\end{proof}
Note that the crucial point in this argument is that $U(m, m)$ is different
from $U(m, m) + 1$; however, if the functions are not total it is possible
that $U(m, m)$ is simply not defined. However, a part of the argument can be
used, nonetheless.
\begin{theorem}
\label{theorem:intersecting-function}
    There is a computable function $f : \N \to \N$ such that
    no computable function $g : \N \to \N$ can differ from $f$
    everywhere; i.e., for any computable function $g : \N \to \N$
    there is $n \in \N$ such that $f(n) = g(n)$.
    (The last equality says that the values are either equal or
    both values are not defined for $n$.)
\end{theorem}
\begin{proof}
  Let $U$ be a universal function for computable functions, and
  let $d : \N \to \N$ be a partial function such that $d(n) = U(n, n)$.
  It is clear that $d$ satisfies the statement of the theorem. Indeed,
  any computable function $f$ is equal to $U_n$  for some $n$. Hence,
  $d(n) = U(n, n) = f(n)$.
\end{proof}

\begin{theorem}
    There is a computable function that does not admit a total computable
    extension.
\end{theorem}
\begin{proof}
  Let $d$ be the function from the previous theorem. Let us consider the
  partial function $e : \N \to \N$ such that $e(n) = d(n) + 1$.
  We wish to prove that $e$ does not have a total extension. Let us assume
  the opposite, let $e'$ be a computable total extension of $e$. Then $e'$
  differs from $d$ everywhere, therefore $e'$ is not computable.
\end{proof}

Note that the last theorem gives another proof of
\Cref{theorem:enumerable-not-decidable}. Indeed, let $f$ be the function
without total computable extension. Let $S = \Im{f}$.
By \Cref{theorem:enumerable-via-computable-funcitons},
$S$ is enumerable. Assume for the sake of contradiction, that $S$ is decidable.
Then the total function $g : \N \to \N$ such that
\[
    g(x) =
    \begin{cases}
        f(x) & \text{if } x \in S \\
        0 & \text{if } x \notin S
    \end{cases}
\]
is computable. Moreover, $g$ is an extension of $f$ which leads to a
contradiction.

\begin{chapterendexercises}
    \exercise Let $U \subseteq \N^2$ be any enumerable set of pairs of
        natural numbers that is universal for the set of all
        enumerable sets of natural numbers.
        Prove that its ``diagonal section''
        $K = \set[(x, x) \in U]{x}$ is an enumerable undecidable set.
    \exercise Let $S \subseteq \N$ be decidable and let
        \[
            D =
            \set[
                p \text{ is prime and }
                p \text{ divides some } n \in S
            ]{p \in \N}.
        \]
        Is the set $D$ always decidable?
    \exercise Show that there exist countably many disjoint
        enumerable sets such that any two of them are inseparable
        (cannot be separated by a decidable set).
\end{chapterendexercises}
