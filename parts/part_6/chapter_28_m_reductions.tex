\chapter{m-Reductions}
In the proofs of \Cref{theorem:halting,theorem:undefined-functions}, to prove
that a set $S$ is undecidable we proved that $S$ is decidable iff an undecidable
set $K$ should be also decidable, which lead to the conclusion that $S$ is
undecidable. This was done using ``reduction'' argument. We constructed a total
computable function $f : \N \to \N$ such that $f(x) \in S$ iff $x \in K$ for all
$x \in \N$. In
this chapter we will study this method in more details.

\begin{definition}
  Let $A, B \subseteq \N$. We say that $A$ is \emph{$m$-reducible} to
  $B$\footnote{%
    The letter ``m'' here stands for ``many-to-one''; however, Sipser's
    ``Introduction to the Theory of Computation'' suggests to call such
    reductions ``mapping reductions'' giving another life for the letter $m$ in
    this notation.
  }
  ($A \le_m B$) if there is a total computable function $f : \N \to \N$ such
  that $x \in A$ iff $f(x) \in B$ for all $x \in \N$. We say that $f$
  \emph{$m$-reduces} $A$ to $B$.
\end{definition}

This notion allows us to translate properties from one set to another.
\begin{theorem}
\label{theorem:m-reduction-complexity-preservation}
  Let $A, B \subseteq \N$ such that $A \le_m B$.
  \begin{itemize}
    \item If $B$ is decidable, then $A$ is decidable.
    \item If $B$ is enumerable, then $A$ is enumerable.
  \end{itemize}
\end{theorem}

\begin{exercise}
  Let $A, B, C \subseteq \N$. Show that
  \begin{itemize}
    \item $A \le_m A$, and 
    \item if $A \le_m B$ and $B \le_m C$, then $A \le_m C$.
  \end{itemize}
\end{exercise}

Notice that the sets $\emptyset$ and $\N$ behave differently than other
decidable sets with respect to $\le_m$.
\begin{remark}
  Let $A, B \subseteq \N$.
  \begin{itemize}
    \item $A \le_m \emptyset$ iff $A = \emptyset$, 
    \item $A \le_m \N$ iff $A = \N$, and
    \item $A \le_m B$ provided that $A$ and $B$ are dicidable and $B \not\in
      \set{\emptyset, \N}$.
  \end{itemize}
\end{remark}

However, in case of enumerable sets, the situation is not that simple.
\begin{exercise}
  Show that there are enumerable sets $A, B \subseteq \N$ such that $A \not\le_m
  B$.
\end{exercise}

In other words, enumerable sets form layers of sets increasing with repspect to
$\le_m$. So the questions is whether there is the last layer or not, the
following theorem give an affirmative answer to this question.
\begin{theorem}
  In the class of enumerable sets, there are sets maximal with respect to
  $m$-reducibility; i.e., there is an enumerable set $B$ sucht that $A \le_m B$
  for any enumerable set $A$.
\end{theorem}
\begin{proof}
  Let $W$ be a enumerable universal set for the set of all enumerable subsets of
  $\N$ (it exists by \Cref{theorem:universal-set-enumerable}). 
  We claim that the set $B = \set[(n, x) \in W]{\pair{n}{x}}$ satisfies the
  requirement of the theorem. Indeed, let $A$ be enumerable set. Then there is
  $n \in \N$ such that $W_n = A$. Hence, it is easy to see that $f(x) \in B$ iff 
  $n \in A$, where $f(x) = \pair{n}{x}$.
\end{proof}

\begin{definition}
  An enumerable set $B$ maximal with respect to $m$-reducibility is called
  \emph{$m$-complete for the class of enumerable sets}.
\end{definition}

\begin{theorem}
\label{theorem:m-complete-enumerable}
  Let $U$ be a G\"odel universal function. 
  Then $\set[U(x, x) \text{ terminates}]{x \in \N}$ is $m$-complete for the
  class of enumerable sets.
\end{theorem}
\begin{proof}
  Let $K \subseteq \N$ be a enumerable set. Let us consider a computable
  function $V : \N^2 \to \N$s such that $V(n, x) = 1$ if $n \in K$ and undefined
  otherwise. Since $U$ is G\"odel universal function, there is a total
  computable $s : \N \to \N$ such that $V_n = U_{s(n)}$. Hence,  $U_{s(n)}$
  decides $\N$ if $n \in K$ and $U_{s(n)}$ decides $\emptyset$ if $K \not\in K$.
  Therefore $s(n) \in D$ iff $n \in K$.
\end{proof}

\begin{chapterendexercises}
  \exercise Prove that the set of all programs that halt on the input $0$ is
    $m$-complete for the class of enumerable sets.
  \exercise Prove that the set of all programs that halt on at least one input
  is $m$-complete for the class of enumerable sets.
\end{chapterendexercises}
