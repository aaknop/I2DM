\chapter{True Statements in Propositional Logic}
\label{chapter:propositional-truth}
\marginurl{%
  Proofs Using Truth Tables:\\\noindent
  Introduction to Mathematical Logic \#2
}{youtu.be/DOAxnmScpPc}


Typical theorem in mathematics have the following template:
``if some statements are true, then another statement is also true''.
In propositional logic statements are described using propositional formulas.
So our goal is to present a way to describe proofs of results that looks like:
if $\phi_1$, \dots, $\phi_k$ are true, then $\psi$ is also true.

This section discusses the method which is based on truth tables (we discussed
them before in \Cref{chapter:predicates}).

We start from an example similar to
the proof gaven in the beginning of the \Cref{chapter:proofs}. Assume that we
know that if $x$ is a real number such that $x < -2$ or $x > 2$, then $x^2 > 4$.
We can derive that if $\lnot (x^2 > 4)$, then $\lnot (x < -2)$ and 
$\lnot (x > 2)$.

In order to emphasize the logical structure of the argument let us denote the
statement $x > 2$ by $p$, the statement $x < -2$ by $q$, and the statement 
$x^2 > 4$ by $r$. In this case the argument is as follows:
if $(p \lor q) \limplies r$ is true, then
$\lnot r \limplies (\lnot p \land \lnot q)$ is true as well.

The simplest way to explain why this argument is true is to use
a truth table.
\begin{center}
  \begin{tabular}{l l l l l}
    \toprule
    $p$ & $q$ & $r$ & $(p \lor q) \limplies r$ &
    $\lnot r \limplies (\lnot p \land \lnot q)$ \\
    \midrule
    T & T & T & T & T \\
    T & T & F & F & F \\
    T & F & T & T & T \\
    T & F & F & F & F \\
    F & T & T & T & T \\
    F & T & F & F & F \\
    F & F & T & T & T \\
    F & F & F & T & T \\
    \bottomrule
  \end{tabular}
\end{center}
Note that each line where $(p \lor q) \limplies r$ is true has
$\lnot r \limplies (\lnot p \land \lnot q)$ true as well. So we proved that
the argument is indeed correct.

We may also note that we showed that
\[
  \bigl((p \lor q) \limplies r\bigr) \iff
  \bigl(\lnot r \limplies (\lnot p \land \lnot q) \bigr)
\]
is always true (we say that this propositional formula is a
\emph{tautology}). A generalization of this saying the if $p \limplies q$ is
true, then $\lnot q \limplies \lnot p$ is also true is called the
\emph{contraposition} argument.

Let us now consider another argument. If we know that Joe was a good boy and we
know that if Joe is a good boy, then Santa gives a present to Joe. We may
conclude that Santa gives a present to Joe. We can similarly to the previous
example write this argument using variables and connectives.
If we know that $p$ and $p \limplies q$, we may conclude that $q$ is true.
\begin{exercise}
  Show that $(p \land (p \limplies q)) \limplies q$ is a tautology.
\end{exercise}
Such an argument is called \emph{modus ponens}.

A notion connected to being a tautology is the notion of being satisfiable.
We say that a formula (a set of formulas) is \emph{satisfiable} iff there
is a substitution to the variables such that the value of the formula is true
(the values of all the formulas are true). Note that a formula is not
satisfiable (the formula is \emph{unsatisfiable}) iff its negation is
a tautology. Therefore, using truth tables one may check whether a formula
is satisfiable or not.\footnote{%
  Note that the procedure is awfully not efficient since if the formula
  uses $n$ variables we need to do $2^n$ operations. Unfortunately,
  we do not know anything that always works better since
  satisfiability problem (the problem of determining whether a given formula is
  satisfiable or not) is NP complete.
}

\section{Semantic Implication}
As we mentioned at the beginning of the chapter, most of the
statements in mathematics are in the form
``if some statements are true, then some statement is also true'';
this type of statements can be described using the notion of semantic
implication. We say that a set $\Sigma$ of propositional formulas with
variables from a set $V$ \emph{semantically implies} a propositional formula
$\phi$ with variables from the set $V$ (we denote it by $\Sigma \models \phi$)
iff whenever all the formulas from $\Sigma$ are true under some propositional
assignement to $V$, the formula $\phi$ is also true under this propositional
assignement; i.e.,
$\Sigma \models \phi$ iff for any $\rho : V \to \set{\ltrue, \lfalse}$,
$\substitute{\phi}{\rho} = \ltrue$ provided that
$\substitute{\psi}{\rho} = \ltrue$ for all $\psi \in \Sigma$.
(Note that the set $\Sigma$ may be infinite.)

We explained that if we have a finite set $\Sigma$, then it is possible to check
whether a formula $\phi$ is semantically implied by $\Sigma$. Let us try to find
out whether we can do the same for infinite sets $\Sigma$.

Partial answer to this question is given by the following theorem.
\begin{theorem}[compactness theorem]
  A set $\Sigma$ of propositional formulas is satisfiable iff every finite
  subset is satisfiable.
\end{theorem}
\begin{proof}
  We say that a set is \emph{finitely satisfiable} if every finite subset
  is satisfiable.

  Let us enumerate all the propositional formulas $\alpha_1$, $\alpha_2$, \dots.
  We define a family of sets $\Delta_1$, \dots, $\Delta_n$, \dots
  such that $\Delta_1 = \Sigma$ and
  \[
    \Delta_{n + 1} =
    \begin{cases}
      \Delta_n \cup \set{\alpha_{n + 1}} & \text{if }
        \Delta_n \cup \set{\alpha_{n + 1}} \text{is finitely satisfiable,} \\
      \Delta_n \cup \set{\lnot \alpha_{n + 1}} & \text{otherwise.}
    \end{cases}
  \]
  Note that all the $\Delta_n$ are finitely satisfiable.

  Let $\Delta = \cup_{n \in \N} \Delta_n$. It is clear that $\Delta$ is
  finitely satisfiable and for any propositional formula $\alpha$,
  either $\alpha$ or $\lnot \alpha$ belongs to $\Delta$.

  Let us consider a substitution $v_1$, \dots, $v_n$, \dots to the variables
  $x_1$, \dots, $x_n$, \dots such that $v_i = \ltrue$ iff the formula $x_i$
  belongs to $\Delta$. We may note that this substitution satisfies any
  formula $\phi \in \Delta$.
\end{proof}

Using this theorem, we can show that any implication of an infinite set is
actually an implication of a finite subset of it.
\begin{corollary}
  Let $\Sigma$ be a set of propositional formulas over the variables
  $x_1$, $x_2$, \dots, $x_n$, \dots, and $\phi$ be a propositional formula
  over the same set. If $\Sigma \models \phi$, then there is a finite
  $\Sigma' \subseteq \Sigma$ such that $\Sigma' \models \phi$.
\end{corollary}
\begin{proof}
  Note that $\Sigma \not\models \phi$ iff $\Sigma \cup \set{\phi}$ is
  satisfiable.

  Let us now assume that for any finite $\Sigma' \subseteq \Sigma$,
  $\Sigma' \not\models \phi$. This implies that $\Sigma' \cup \set{\phi}$
  is satisfiable for all finite $\Sigma'$. Therefore, $\Sigma \cup \set{\phi}$
  is satisfiable, which is a contradiction to the assumption that
  $\Sigma \models \phi$.
\end{proof}

Therefore if we wish to check whether a formula $\phi$ is semantically implied
by $\Sigma$, we just need to brute-force all the finite subsets of $\Sigma$ and
check whether they semantically imply $\phi$. By the previous argument, if
$\phi$ is implied by $\Sigma$, this procedure reports ``yes'' at some point,
and in the opposite case it will work infinitely long.

\begin{chapterendexercises}
  \exercise Show that $((A \land B) \lor C) \limplies (A \lor C)$ is a
    tautology.
  \exercise Show that $((A \limplies B) \land (\lnot A \limplies C)) \limplies
    (B \lor C)$ is a tautology.
  \exercise Show that 
    $((W \limplies X) \land (Y \limplies Z)) \limplies 
      ((W \lor Y) \limplies (X \lor Z))$
     is a tautology.
\end{chapterendexercises}
