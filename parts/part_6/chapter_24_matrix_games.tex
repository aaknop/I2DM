\chapter{Matrix Games}
This part uses mathematical methods to model interactions and conflicts between
different actors. To simplify the analysis we are going to consider interactions
between \emph{two} actors that are trying to maximize their payoffs.

\begin{definition}
  The \emph{strategic form}, or \emph{normal form}, of a two-person game is
  a tuple $(X, Y, A, B)$ such that $X$ and $Y$ are some nonempty sets, and 
  $A, B  : X \times Y \to \R$.

  We say that this game is a matrix game if $X$ and $Y$ are finite.
\end{definition}
The interpretation is as follows. Simultaneously, the first player chooses 
a strategy $x \in X$ and the second player chooses a strategy $y \in Y$ , each
unaware of the choice of the other. Then their choices are made known and the
first player wins $A(x, y)$ and the second player wins $B(x, y)$. (Depending on
the monetary unit involved, $A(x, y)$ and $B(x, y)$ will be dollars, rubles,
euros, etc.) If $A(x, y)$ or $B(x, y)$ is negative, the corresponding player
loses the absolute value of this amount.

It is important to note that the notion of a strategy is very broad; e.g., a
strategy for a game of chess, is a complete description of how to play the game,
of what move to make in every possible situation that could occur. We are going
to ignore the fact that in the game of chess it is physically impossible to
describe all possible strategies since there are too many of them (in fact,
there are more strategies than there are atoms in the known universe). On the
other hand, the number of games of tic tac toe is rather small, so that it is
possible to study all strategies and find an optimal strategy for each player.

In cases when $X$ and $Y$ are small sets it is convenient to describe such games
using tables. Let us consider the game described by
\Cref{table:heads-and-tales-game}.
\begin{table}
  \begin{center}
    \begin{tabular}{l l l  l  l  l  l  l  l}
      \toprule
            & heads  & tales   \\
      \midrule
      heads & 1, -1 & -1, 1   \\
      tales & -1, 1 & 1, -1   \\
      \bottomrule
    \end{tabular}
  \end{center}
  \caption{Heads and tales game}
  \label{table:heads-and-tales-game}
\end{table}
In this game two players put one coin each on the table: if the the coins have
the same side up, then the second player pays $1$ dollar to the first player,
otherwise the first player pays $1$ dollar to the second player. In other words,
the firs number denotes the payoff of the first player and the second denotes
the payoff of the second player.

\begin{exercise}
  Describe the game in normal form corresponding to rock paper scissors.
\end{exercise}

An important class of games is zero-sum games.
\begin{definition}
  A game $(X, Y, A, B)$ is \emph{zero-sum} if $A(x, y) = -B(x, y)$ for all 
  $x \in X$ and $y \in Y$.
\end{definition}
It is clear that the game we described is a zero-sum game.

\section{Domination and Pareto Optimal Strategies}
In \Cref{part:combinatorial-games} we studies optimal strategies; however, in
case of games in the normal form it is not clear what does it mean optimal. To
illustrate this difficulty, let us discuss the most famous game, the prisoner's
dilemma:
\begin{game}
  Two members of a criminal gang are arrested and imprisoned. Each prisoner is in
  solitary confinement with no means of communicating with the other. The
  prosecutors lack sufficient evidence to convict the pair on the principal
  charge, but they have enough to convict both on a lesser charge. Simultaneously,
  the prosecutors offer each prisoner a bargain. Each prisoner is given the
  opportunity either to betray the other by testifying that the other committed
  the crime, or to cooperate with the other by remaining silent. The possible
  outcomes are:
  \begin{itemize}
    \item If A and B each betray the other, each of them serves two years in
      prison.
    \item If A betrays B but B remains silent, A will be set free and B will serve
      three years in prison (and vice versa).
    \item If A and B both remain silent, both of them will serve only one year in
      prison (on the lesser charge).
  \end{itemize}
\end{game}

It is clear that this game can be described using the following table.
\begin{center}
  \begin{tabular}{l l l  l  l  l  l  l  l}
    \toprule
               & cooperates  & defects   \\
    \midrule
    cooperates & -1, -1      & -3, 0    \\
    defects    & 0, -3       & -2, -2   \\
    \bottomrule
  \end{tabular}
\end{center}


Let us try to put ourselves into these prisoners shoes. If our partner is silent,
then it is better for us to defect (in this case we are free immediately); if
our partner defects, then is is also better for us to defect (we'll get two
years instead of three). Therefore, no matter what our partner does it is always
better to defect. Since the game is symmetric both players come to this
conclusion and get two years each; however, if both of them cooperates they
would serve only one year\footnote{%
  In 1993 Frank, Gilovich, and Regan conducted an experimental study of the
  prisoner's dilemma. The subjects were students in their first and final years
  of undergraduate economics, and undergraduates in other disciplines. Subjects
  were paired, placed in a typical game scenario, then asked to choose either to
  ``cooperate'' or to ``defect''. 

  First year economics students, and students doing disciplines other than
  economics, overwhelmingly chose to cooperate. But 4th year students in
  economics tended to not cooperate. Therefore, the authors concluded that
  that the study of economics reduces cooperation in games. The idea is
  that much of the time cooperation and consideration of other's perspective are
  irrational in the narrow sense of the word. Thus, learning that cooperation is
  irrational in some situations is influencing the behavior of the students
  towards less cooperation, presumably to the negative.
}.

Let us generalise the argument we used to justify why defecting is better than
cooperating.
\begin{definition}
  Let $(X, Y, A, B)$ be a game in normal form. We say that $x_1 \in X$
  \emph{dominates} $x_2 \in X$ iff $A(x_1, y) \ge A(x_2, y)$ for all $y \in Y$.
  We also say that $x_1$ \emph{strictly dominates} $x_2$ if $A(x_1, y) > A(x_2,
  y)$ for all $y \in Y$.

  Similarly we may define domination of strategies for the second player.
\end{definition}
It seems reasonable to never choose $x_2$ provided that $x_1$ dominates $x_2$; in
the prisoners dilemma ``defects'' dominates ``cooperates''; hence, it seems
choosing ``defects'' is the best behaviour for rational agents.

However, it also obvious that if both players cooperate is way better than if
both of them defects; this observation leads to the following definition.
\begin{definition}
  Let $(X, Y, A, B)$  be a game in the normal form. We say that a pair of
  strategies $(x, y) \in X \times Y$ is Pareto optimal if either 
  $A(x', y') < A(x, y)$ or $B(x', y') < B(x, y)$ for any 
  $(x', y') \in X \times Y$.
\end{definition}
In other words, a pair of strategies is Pareto optimal if any other choice would
decrease the payoff for at least one of the players.

\section{Prisoners Dilemma In Real Life}
The reason that prisoner's dilemma is that famous is because there are many
examples human interactions as well as interactions in nature that have the same
payoff matrix. Let us consider two of them.

\paragraph{Political science.} In political science, the prisoner's dilemma is
often used to demonstrate the coherence of strategic realism, which holds that
in international relations, all countries (regardless of their internal policies
or professed ideology), will act in their rational self-interest given
international anarchy. A standard example is an arms race like the Cold War:
during the Cold War the opposing alliances of NATO and the Warsaw Pact both had
the choice to arm or disarm. From each side's point of view, disarming when
their opponent continued to arm would have led to military inferiority and
possible annihilation. Conversely, arming whilst their opponent disarmed would
have led to superiority. If both sides chose to arm,
neither could afford to attack the other, but both incurred the high cost of
developing and maintaining a nuclear arsenal. If both sides chose to disarm, war
would be avoided and there would be no costs.

Although the ``best'' overall outcome is for both sides to disarm, the rational
course for both sides is to arm, and this is indeed what happened. Both sides
poured enormous resources into military research and armament in a war of
attrition for the next thirty years until the Soviet Union could not withstand
the economic cost. 

\paragraph{Sports.} Another example is doping in sports. Two competing athletes
have the option to use an illegal and/or dangerous drug to boost their
performance. If neither athlete takes the drug, then neither gains an advantage.
If only one does, then that athlete gains a significant advantage over their
competitor, reduced by the legal and/or medical dangers of having taken the
drug. If both athletes take the drug, however, the benefits cancel out and only
the dangers remain, putting them both in a worse position than if neither had
used doping.
