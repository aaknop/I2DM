\chapter{Cardinalities of Sets}
\label{chapter:cardinality}
\Cref{part:combinatorics} discusses sizes of sets; however, this notion was
never defined formally. This part is going to close this gap and study
properties of sizes of sets.


\section{Definition}
One may notice that if we have a bijection $f$ from $\range{n}$ to a set $S$ we
enumerate all the elements of $S$: $f(1)$, \dots, $f(n)$.
This observation allows us to define the cardinality of a set.
\begin{definition}
  Let $S$ be a set, we say that cardinality of $S$ is equal to $n$ (we write
  that $|S| = n$) iff there is a bijection from $\range{n}$ to $S$.

  We also say that a set $T$ is finite if there is an integer $n$ such that
  $|T| = n$.
\end{definition}

Note that this definition does not guarantee that cardinality is unique so
we need the following theorem.
\begin{theorem}
\label{theorem:correctness-of-cardinality}
    For any set $S$, if there are bijections $f : \range{n} \to S$ and
    $g : \range{m} \to S$, then $n = m$.
\end{theorem}
\begin{proof}[Proof of Theorem~\ref{theorem:correctness-of-cardinality}]
    Let us consider the inverse $g^{-1}$ of $g$ (it exists by
    Theorem~\ref{theorem:inverse-of-bijections} since $g$ is a
    bijection). Note that $h = g^{-1} \circ f$
    is a bijection from $\range{n}$ to $\range{m}$.

    We prove using induction by $n$ that for any $n, m \in \N$,
    if there is a bijection $h'$ from $\range{n}$ to $\range{m}$, then $n = m$.
    The base case is for $n = 1$; if $m \ge 2$,
    then there are $x, y \in [1]$ such that $h'(x) = 1$ and $h'(y) = 2$, but
    $x \neq y$ and we have only one element in $[1]$.

    The induction step is also simple. Assume that there is a bijection $h'$ from
    $[n + 1]$ to $\range{m}$. We define a function
    $h'' : \range{n} \to \range{m - 1}$ as follows:
    \[
        h''(i) =
        \begin{cases}
            h'(i) & \text{if } h'(i) < h'(n + 1) \\
            h'(i) - 1 & \text{otherwise}
        \end{cases}.
    \]
    We prove that $h''$ is a bijection.
    \begin{itemize}
        \item Let $i_1 \neq i_2 \in \range{n}$. If
            $h'(i_1), h'(i_2) < h'(n + 1)$ or $h'(i_1), h'(i_2) \ge h'(n + 1)$,
            then $h''(i_1) \neq h''(i_2)$ since $h'(i_1) \neq h'(i_2)$.
            Otherwise, without loss of generality we may assume that
            $h'(i_1) < h'(n + 1) < h'(i_2)$ but it implies that
            $h''(i_1) = h'(i_1) < h'(n + 1) \le h'(i_2) - 1 = h''(i_2)$.
        \item Let $j \in [m - 1]$. We need to consider two cases.
            \begin{enumerate}
                \item Let $j < h(n + 1)$. There is $i \in [n + 1]$ such that
                    $h'(i) = j$ since $h'$ is a bijection
                    (note that $i \neq n + 1$). Thus $h''(i) = j$.
                \item Otherwise, there is $i \in [n + 1]$ such that
                    $h'(i) = j + 1$ since $h'$ is a bijection
                    (note that $i \neq n + 1$). Thus $h''(i) = j$.
            \end{enumerate}
    \end{itemize}
    Since $h''$ is a bijection, the induction hypothesis implies that
    $n = m - 1$. As a result, $n + 1 = m$.
\end{proof}


Also, using this definition we may finally prove
\Cref{theorem:bijection-to-equality}.
\begin{proof}[Proof of \Cref{theorem:bijection-to-equality}]
    Let $|X| = n$, and $g : \range{n} \to X$ be a bijection.
    Note that $f \circ g : \range{n} \to Y$ is a bijection, hence $|Y| = n$.
\end{proof}

\section{Proofs of Combinatorial Principles}
We start from proving the additive principle. Recall that it says that if two
finite sets $X = \set{x_1, \dots, x_n}$ and $Y = \set{y_1, \dots, y_m}$ are
disjoint, then their union contains $n + m$ elements. Informally this statement
is very easy since we can simply enumerate elements of the set $X$ first and
after that enumerate elements of $Y$:
\[
  \begin{matrix}
    x_1 & x_2 & \dots & x_n & y_1    & y_2   & \dots & y_m \\
    1   & 2   & \dots & n   & n + 1  & n + 2 & \dots & n + m.
  \end{matrix}
\]
Using the definition of cardinality, this informal proof can be formalized as
follows. 
\begin{proof}[Proof of \Cref{theorem:additive-principle}]
  Let $|X| = n$, $|Y| = m$ and $g : \range{n} \to X$ and $h : \range{m} \to Y$ be
  bijections. In order to prove it we just construct a bijection
  $f : [n + m] \to (X \cup Y)$.

  \[
    f(i) =
    \begin{cases}
        g(i) & i < n \\
        h(i  - n) & i > n
    \end{cases}.
  \]

  It's easy to see that $f$ is an injection. Let us start by assuming the opposite i.e.
  that $i_0 \neq i_1 \in X \cup Y$ such that $f(i_0) = f(i_1)$.
  There are three cases.
  \begin{itemize}
    \item The first is when $i_0, i_1 \in \range{n}$. In this case $g(i_0) = g(i_1)$
      which contradicts the assumption that $g$ is a bijection.
    \item The second is when $i_0, i_1 \in \set{n + 1, n + 2, \dots, m}$.
      In this case $h(i_0 - n) = h(i_1 - n)$ which contradicts the assumption
      that $h$ is a bijection.
    \item Finally, the last case is when $i_0 \in \range{n}$ and
      $i_1 \in \set{n + 1, n + 2, \dots, m}$. It is easy to see that this
      implies that $g(i_0) = h(i_1 - n)$. However, it
      means that $g(i_0) = h(i_1 - n) \in (X \cap Y)$, which contradicts the
      assumption that $X \cap Y = \emptyset$.
  \end{itemize}

  To finish the proof we need to show that $f$ is a surjection. Let
  $w \in (X \cup Y)$. Consider the following two cases.
  \begin{itemize}
    \item Let $w \in X$. There is $i \in \range{n}$ such that $f(i) = g(i) = w$ since
      $g$ is a bijection.
    \item Otherwise, $w \in Y$. In this case, there is $i \in \range{m}$ such that
      $f(i + n) = h(i) = w$ since $h$ is a bijection.
  \end{itemize}
\end{proof}

Using the additive principle it is pretty easy to prove the multiplicative
principle. Recall that the multiplicative principle says that the Cartesian
product of two finite sets $X = \set{x_1, \dots, x_n}$ and 
$Y = \set{y_1, \dots, y_m}$ has size $nm$. It is easy to justify this statement
informally; indeed, note that $X \times Y$ has the following elements:
\[
  \begin{matrix}
    (x_1, y_1) & (x_1, y_2) & \dots & (x_1, y_m) \\
    (x_2, y_1) & (x_2, y_2) & \dots & (x_2, y_m) \\
    \dots \\
    (x_n, y_1) & (x_n, y_2) & \dots & (x_n, y_m);
  \end{matrix}
\]
each line has $m$ elements and there are $n$ lines, so there are $nm$ elements
altogether that can be enumerated as follows:
\[
  \begin{matrix}
    (x_1, y_1) & (x_1, y_2) & \dots & (x_1, y_m) \\
    1          & 2          & \dots & m \\
    (x_2, y_1) & (x_2, y_2) & \dots & (x_2, y_m) \\
    m + 1      & m + 2      & \dots & m + m \\
    \dots \\
    (x_n, y_1) & (x_n, y_2) & \dots & (x_n, y_m) \\
    (n - 1) m + 1 & (n - 1) m + 2 & \dots & (n - 1) m + m
  \end{matrix}
\]
The definition of cardinality gives us a simple formalization of this argument.
\begin{proof}[Proof of \Cref{theorem:multiplicative-principle}]
  If one of the sets $X$ and $Y$ is empty, then $X \times Y$ is empty as well
  and the statement is as follows.

  Assume that none of the sets are empty. Let $|X| = n$, $|Y| = m$, and
  $f : \range{n} \to X$ and $g : \range{m} \to Y$ be bijections.
  Note that
  \[
    \bigcup_{i = 1}^n (\set{f(i)} \times Y) = X \times Y.
  \]
  Additionally, note that $(\set{f(i)} \times Y) \cap (\set{f(j)} \times Y) =
  \emptyset$ for $i \neq j$. Finally, it is easy to see that $g_i : \range{m} \to
  (\set{f(i)} \times Y)$ such that $g_i(j) = (f(i), g(j))$ is a bijection.
  Hence,
  $|X \times Y| = \sum_{i = 1}^n |\set{f(i)} \times Y| = n \cdot m$.
\end{proof}


\section{Generalized Commutative Operations}
\label{section:generalized-sum}
Using the notion of cardinality we can formalize the summation operation over
sets:
\[
    \sum_{i \in S ~:~ P(i)} f(i) = \sum_{j = 1}^k f(i_j),
\]
where $\set[P(i)]{i \in S} = \set{i_1, \dots, i_k}$. More formally,
\[
    \sum_{i \in S ~:~ P(i)} f(i) = \sum_{j = 1}^k f(g(j)),
\]
where $k = |\set[P(i)]{i \in S}|$ and $g : \set[P(i)]{i \in S} \to \range{k}$ is
a bijection.

\begin{theorem}
\label{theorem:sum-correctness}
    The definition of $\sum_{i \in S : P(i)} f(i)$ does not depend on the choice
    of $g$;
    i.e. $\sum_{i = 1}^k f(g_1(i)) = \sum_{i = 1}^k f(g_2(i))$
    for any two bijections $g_1, g_2 : \set[P(i)]{i \in S} \to \range{k}$.
\end{theorem}
Before we prove this statement we need to give a couple of definitions.
We say that a function $h : \range{n} \to \range{n}$ is a \emph{permutation} of
$\range{n}$ iff $h$ is a bijection. We also say that
$i, j \in \range{k}$ form the inversion in $h$ iff $h(i) > h(j)$ and $i < j$.
We denote by $I(h)$ the number of inversions in $h$; i.e. $I(h) =
|\set[i, j \text{ form an inversion in } h]{(i, j)}|$.
\nomenclature[C]{$I(h)$}{denotes the number of inversions in $h$}

Important examples of permutations are transposition: for any
$i, j \in \range{n}$, $\tau_{i, j} : \range{n} \to \range{n}$ such that
\[
    \tau_{i, j}(x) =
    \begin{cases}
        j & \text{if } x = i \\
        i & \text{if } x = j \\
        x & \text{otherwise}
    \end{cases}.
\]
is called a transposition of $i$ and $j$.
\nomenclature[F]{$\tau_{i, j}$}{denotes the transposition of $i$ and $j$}

It is easy to see that $I(h) = 0$ iff $h(i) = i$ for any $i \in \range{k}$.
It is also clear that if $i, j$ form an inversion in $h$, then $I(h) > I(h')$,
where $h' = h \circ \tau_{i, j}$, i.e.
\[
    h'(x) =
    \begin{cases}
        h(j) & \text{if } x = i \\
        h(i) & \text{if } x = j \\
        h(x) & \text{otherwise}
    \end{cases}.
\]


\begin{proof}[Proof of Theorem~\ref{theorem:sum-correctness}]
    Proof of this theorem consists of two parts.
    First, we prove that
    \begin{equation}
        \label{equation:summation-with-respect-to_bijection}
        \sum_{i = 1}^k f(g(i)) = \sum_{i = 1}^k f(g(h(i)))
    \end{equation}
    for any bijections $g : \set[P(i)]{i \in S} \to \range{k}$ and
    $h : \range{k} \to \range{k}$.

    We prove Equation~\ref{equation:summation-with-respect-to_bijection} using the
    induction by $I(h)$.
    \begin{description}
        \item[(the base case)] If $I(h) = 0$, then $h$ is the identity function and
            $g(i) = g(h(i))$. Hence,
            Equation~\ref{equation:summation-with-respect-to_bijection} is true.
        \item[(the induction step)] By the induction hypothesis, for any permutation
            $h' : \range{k} \to \range{k}$,
            if $I(h') < \ell$, then
            \[
                \sum_{i = 1}^k f(g(i)) = \sum_{i = 1}^k f(g(h'(i))).
            \]
            Let us consider a permutation $h : \range{k} \to \range{k}$ such that
            $I(h) = \ell$. Let $i$ and $j$ form an inversion in $h$ (such $i$ and $j$
            exist since $I(h) \neq 0$). Let $h' = h \circ \tau_{i, j}$.
            Note that by the induction hypothesis,
            \[
                \sum_{i = 1}^k f(g(i)) = \sum_{i = 1}^k f(g(h'(i)))
            \]
            since $I(h') < I(h) = \ell$ and it is clear that
            \[
                \sum_{i = 1}^k f(g(h'(i))) = \sum_{i = 1}^k f(g(h(i))).
            \]
            As a result,
            Equation~\ref{equation:summation-with-respect-to_bijection} is true.
    \end{description}

    Now we are ready to finish proof of the theorem.
    Consider $g_1, g_2 : \set[P(i)]{i \in S} \to \range{k}$ and define
    $h = g_1^{-1} \circ g_2$. Note that $h : \range{k} \to \range{k}$ is a
    permutation and $g_1(h(i)) = g_2(i)$. Thus we proved that
    \[
        \sum_{i = 1}^k f(g_1(i)) = \sum_{i = 1}^k f(g(h(i))) =
        \sum_{i = 1}^k f(g_2(i)).
    \]
\end{proof}

Similarly one may define a generalized union and intersection of sets.
Let $\Omega$ and $S$ be some sets, $X : S \to 2^\Omega$ and $P(i)$ be a
predicate. Then
\begin{gather*}
    \bigcup_{i \in S ~:~ P(i)} X(i) = \bigcup_{i = 1}^k X(g(i)) \\
    \text{and}\\
    \bigcap_{i \in S ~:~ P(i)} X(i) = \bigcap_{i = 1}^k X(g(i)),
\end{gather*}
where $k = |\set[P(i)]{i \in S}|$ and $g : \set[P(i)]{i \in S} \to \range{k}$ is
a bijection.
\nomenclature[S]{$\bigcup_{i \in S ~:~ P(i)} A_i$}{denotes $A_{i_1} \cup \dots
\cup A_{i_k}$, where $\set[P(i)]{i \in S} = \set{i_1, \dots, i_k}$}
\nomenclature[S]{$\bigcap_{i \in S ~:~ P(i)} A_i$}{denotes $A_{i_1} \cap \dots
\cap A_{i_k}$, where $\set[P(i)]{i \in S} = \set{i_1, \dots, i_k}$}

\begin{exercise}
    Show that the definitions of $\bigcup_{i \in S : P(i)} X(i)$ and
    $\bigcap_{i \in S : P(i)} X(i)$ are correct,
    i.e. that they do not depend on the choice of $g$.
\end{exercise}


\begin{chapterendexercises}
    \exercise Prove Theorem~\ref{theorem:injections-surjections-inequalities}.
\end{chapterendexercises}
