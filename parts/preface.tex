\epigraph{
    - Why is a math book so sad? \\
    - Because it's full of problems.
}{Anonymous, Unknown}


If you are reading this book, you probably have never studied proofs before.
So let me give you some advice: mathematical books are very different from
fiction, and even books in other sciences. Quite often you may see that some
steps are missing, and some steps are not really explained and just claimed as
obvious. The main reason behind this is to make the ideas of the proof more
visible and to allow grasping the essence of proofs quickly.

Since the steps are skipped, you cannot just read the book and believe that you
learned the topic; the best way to actually learn the topic is to try to prove
every statement before you read the actual proof in the book. In addition to
this, I recommend trying to solve all the exercises in the book --- you may find
them in the middle and at the end of every chapter --- and especially the
exercises labeled as recommended.

Additionally, many topics in this book have a corresponding short videos
explaining the material of the chapter, it is useful to watch them before you go
into the topic.

\section*{Organization}
\Cref{part:mathematical-reasoning} covers the basics of mathematics and
provide the language we use in the next parts. We start from the explanation of
what a mathematical proof is (in \Cref{chapter:proofs}).
\Cref{chapter:indirect-proofs} shows how to prove theorems indirectly
using proof by contradiction. \Cref{chapter:induction} explains the most
powerful method in our disposal, proof by induction. Finally,
\Cref{chapter:predicates,chapter:sets,chapter:functions,chapter:relations}
define several important objects such as sets, functions, and relations.

\Cref{part:combinatorics} studies the basics of combinatorics, a branch of
mathematics that answers the question ``how many objects of this kind?''.
\Cref{chapter:bijections-surjections-injections} gives a formal
definition of ``size'' of a set and show how to compare sizes of two sets.
\Cref{chapter:principles} proves several simple principles that allow to
find sizes of sets. In \Cref{chapter:pigeonhole} we learn how to prove existence
of an object with some properties using simple inequalities between sizes of
sets. \Cref{chapter:binomials,chapter:partitions,chapter:permutations} prove
several properties of standard combinatorial objects. Finally,
\Cref{chapter:generating-functions} provides a framework that helps to find
sizes of sets in many cases.

\Cref{part:logic} returns back to proofs; however, instead of studying \emph{how}
to prove something we study what can we prove and how to define ``proof''
so that we can use computer to generate proofs and verify them.

In \Cref{part:graph-theory} we study basics of graph theory.
\Cref{chapter:graph-theory-basics} gives the definition of a graph and prove
the one of the simplest and at the same time most important theorems in graph
theory. In \Cref{chapter:paths} we define what it means beeing connected and
how to use this notion in real-life applications. Finally, \Cref{chapter:trees}
defines a tree and show how to use these objects in computer networks.

\begin{flushright}
  Alexander Knop \\
  San Diego, California, USA
\end{flushright}
