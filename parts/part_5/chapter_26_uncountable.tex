\chapter{Uncountable Sets}

In the previous chapter we proved that many sets are countable; moreover, we
proved that standard operations such as union, intersection, and product of
countable sets is also countable. This chapter shows that we know at least one
uncountable set. Moreover, this chapter gives tools to compare cardinalities of
sets.

\section{Cardinality of Reals}
\begin{theorem}[Cantor]
\label{theorem:reals-are-uncountable}
  The set of real number $\R$ is uncountable.
\end{theorem}

To prove this theorem we need the following two lemmas.
\begin{lemma}
\label{lemma:segment-is-uncountable}
  The set $\set[0 < x < 1]{x \in \R}$ of real numbers between $0$ and $1$ is
  uncountable.
\end{lemma}

\begin{lemma}
\label{lemma:segment-equipotent-to-reals}
  The set $\R$ is equipotent to $\set[0 < x < 1]{x \in \R}$.
\end{lemma}

\begin{proof}[Proof of \Cref{theorem:reals-are-uncountable}]
  Assume that $\R$ is denumerable. Then by
  \Cref{remark:denumerable-equipotent-to-denumerable,lemma:segment-equipotent-to-reals},
  $\set[0 \le x \le 1]{x \in \R}$ is also denumerable, which contradicts
  \Cref{lemma:segment-is-uncountable}.
\end{proof}

To complete the proof of \Cref{theorem:reals-are-uncountable} we need to prove
these two lemmas.
\begin{proof}[Proof of \Cref{lemma:segment-is-uncountable}]
  Assume $\set[0 < x < 1]{x \in \R}$ is countable; this implies that 
  $S = \set[0 \le x \le 1]{x \in \R}$ is also countable. 

  Note that each number $x \in S$ can be represented as an infinite decimal 
  \[
    x = 0.a_1 a_2 a_3 \dots
  \]
  where $0 \le a_i \le 9$ for all $i \in \N$ (the number $1$ can be represented
  as $0.\dot{9}$).

  Let $f : \N \to S$ be a bijection, and let $f(n) = .a_{n, 1} a_{n, 2} \dots$.
  Consider the sequence $b_1$, $b_2$, \dots such that 
  \[
    b_i = 
    \begin{cases}
      0 & \text{if } a_{i, i} \neq 0 \\
      1 & \text{otherwise}
    \end{cases}.
  \]
  Let $y = 0.b_1 b_2 \dots$ and let $m \in \N$ be an integer such that $f(m) =
  y$. Note that 
  \[
    .a_{m, 1} a_{m, 2} \dots = f(m) =  0.b_1 b_2 \dots
  \]
  However, $a_{m, m} \neq b_m$ which is a contradiction.
\end{proof}

\begin{proof}[Proof of \Cref{lemma:segment-equipotent-to-reals}]
  To prove this lemma we start from proving that $\R$ is equipotent to 
  $\set[-\pi / 2 < x < \pi / 2]{x \in \R}$; indeed, we may consider the tangent
  function which is a bijection between these two sets.

  The next step is to note that the sets $\set[-\pi / 2 < x < \pi / 2]{x \in \R}$ and 
  $\set[0 < x < 1]{x \in \R}$ are also equipotent. Consider the function 
  $f : \set[0 < x < 1]{x \in \R} \to \set[-\pi / 2 < x < \pi / 2]{x \in \R}$
  such that $f(x) = \pi (x - 1 / 2)$. It is clear that $f$ is a bijection, which
  finishes the proof.
\end{proof}

\begin{chapterendexercises}
    \exercise[recommended] TBA
\end{chapterendexercises}
