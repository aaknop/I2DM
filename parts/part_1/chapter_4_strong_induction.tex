\chapter{Strong Induction}
Sometimes $P(k)$ is not enough to prove $P(k + 1)$ and we need all the
statements $P(1)$, \dots, $P(k)$. In this case we may use the following
induction  principle.

\begin{theorem}[The Strong Induction Principle]
\label{theorem:strong-induction}
    Let $P(n)$ be some statement about positive integer $n$.
    Hence, $P(n)$ is true for every integer $n > n_0$ iff
    \begin{description}
        \item [(the base case)] $P(n_0 + 1)$ is true and
        \item [(the induction step)] If $P(n_0 + 1)$, \dots, $P(n_0 + k)$ are true,
            then $P(n_0 + k + 1)$ is also true for all positive integers $k$.
  \end{description}
\end{theorem}

Before we prove this theorem let us present some applications of this
principle.

The Fibonacci numbers are defined as follows:
$f_0 = 0$, $f_1 = 1$, and $f_k = f_{k - 1} + f_{k - 2}$ for $k \ge 2$ (note
that they are also defined using strong induction since we use not only
$f_{k - 1}$ to define $f_k$).\footnote{%
  Fibonacci numbers are named after Italian mathematician Leonardo of Pisa,
  later known as Fibonacci. In his 1202 book ``Liber Abaci'', Fibonacci
  introduced the sequence to Western European mathematics. However, the sequence
  had been described as early as 200 BC in work by Indian mathematician Pingala
  on enumerating possible patterns of Sanskrit poetry formed from syllables of
  two lengths.
  
  Fibonacci numbers appear unexpectedly often in mathematics, so much so that
  there is an entire journal dedicated to their study, the ``Fibonacci
  Quarterly''. Applications of Fibonacci numbers include computer algorithms
  such as the Fibonacci search technique and the Fibonacci heap data structure,
  and graphs called Fibonacci cubes used for interconnecting parallel and
  distributed systems.
}

\begin{theorem}[The Binet formula]
    The Fibonacci numbers are given by the following formula
    \[
        f_n = \frac{\alpha^n - \beta^n}{\sqrt{5}},
    \]
    where $\alpha = \frac{1 + \sqrt{5}}{2}$ and
    $\beta = \frac{1 - \sqrt{5}}{2}$.
\end{theorem}
\begin{proof}
    We use the strong induction principle to prove this statement with $n_0 = -1$.
    Let us first prove the base case,
    $\frac{(\alpha^0 - \beta^0)}{\sqrt{5}} = 0 = f_0$.
    We also need to prove the induction step.
    \begin{itemize}
        \item If $k = 1$, then $\frac{(\alpha^1 - \beta^1)}{\sqrt{5}} = 1 = f_1$.
        \item Otherwise, by the induction hypothesis,
            $f_k = \frac{\alpha^k - \beta^k}{\sqrt{5}}$ and
            $f_{k - 1} = \frac{\alpha^{k - 1} - \beta^{k - 1}}{\sqrt{5}}$.
            By the definition of the Fibonacci numbers $f_{k + 1} = f_k + f_{k - 1}$.
            Hence,
            \[
                f_{k + 1} = \frac{\alpha^k - \beta^k}{\sqrt{5}} +
                \frac{\alpha^{k - 1} - \beta^{k - 1}}{\sqrt{5}}.
            \]
            Note that it is enough to show that
            \begin{equation}
            \label{equation:binet}
                \frac{\alpha^{k + 1} - \beta^{k + 1}}{\sqrt{5}} =
                \frac{\alpha^k - \beta^k}{\sqrt{5}} +
                \frac{\alpha^{k - 1} - \beta^{k - 1}}{\sqrt{5}}.
            \end{equation}
            Note that it is the same as
            \[
                \frac{\alpha^{k + 1} - \alpha^k - \alpha^{k - 1}}{\sqrt{5}} =
                \frac{\beta^{k + 1} - \beta^k - \beta^{k - 1}}{\sqrt{5}}.
            \]
            Additionally, note that $\alpha$ and $\beta$ are roots of the equation
            $x^2 - x - 1 = 0$. Hence,
            $\alpha^{k + 1} - \alpha^k - \alpha^{k - 1} = \alpha^{k - 1}(\alpha^2 -
            \alpha - 1) = 0$ and
            $\beta^{k + 1} - \beta^k - \beta^{k - 1} =
            \beta^{k - 1}(\beta^2 - \beta - 1) = 0$. Which implies
            equality~(\ref{equation:binet}).
    \end{itemize}
\end{proof}

Another example of an application of the strong induction is the proof that
any number can be written in digital numeral systems with any base.
\begin{theorem}
  Let $b > 1$ be an integer. Then there is a unique representation 
  $(c_0, \dots, c_\ell)_b$ of any positive number $n$ in the base-$b$ digital
  numeral system. 
  In other words, for any positive integer $n$, there are unique 
  $0 \le c_0, \dots, c_\ell < b$ such that $n = \sum_{i = 0}^\ell b^i c_i$.
\end{theorem}
\begin{proof}
  We prove the statement using strong induction by $n$. The base case for
  $n < b$ is clear (we can choose $\ell = 0$ and $c_0 = n$).
  Let us now prove the induction step. Assume the statement is true for all
  $k < n$. Let $n$ divided by $b$ be equal to $q$ with the remainder $c_0$.
  Note that $(n - c_0) / b < n$ is a positive integer. Hence,
  by the induction hypothesis, there are
  $0 \le c_1, \dots, c_\ell < b$
  such that $(n - c_0) / b = \sum_{i = 1}^\ell b^{i - 1} c_i$. Hence,
  $n = \sum_{i = 0}^\ell b^i c_i$.
\end{proof}


Now we are ready to prove the strong induction principle.
\begin{proof}[Proof of Theorem~\ref{theorem:strong-induction}]
  It is easy to see that if $P(n)$ is true for all $n > n_0$, then the base
  case and the induction steps are true. Let us prove that if the base case and
  the induction step are true, then $P(n)$ is true for all $n > n_0$.

  Let $Q(k)$ be the statement that $P(n_0 + 1)$, \dots, $P(n_0 + k)$ are true.
  Note that $Q(1)$ is true by the base case for $P$. Additionally, note that if
  $Q(k)$ is true, then $Q(k + 1)$ is also true, by the induction step for $P$.
  Hence, by the induction principle, $Q(k)$ is true for all positive integers
  $k$. Which implies that $P(n_0 + k)$ is true for all positive integers $k$.
\end{proof}


\section{Analysis of Recursive Algorithms}
\label{section:strong-induction-recursive}
To illustrate the power of recursive definitions and strong induction, consider
following game: Alice have chosen a number from $1$ to $1000$. Bob wants to
guess the number so he is asking Alice ``yes'' or ``no'' questions.
How many questions does Bob need to ask to determine the number in the
worst-case scenario?

The following simple algorithm allows Bob to learn the number using $10$
questions.

\begin{enumerate}
  \item Bob start with two numbers $\ell = 0$ and $u = 1000$.
  \item \label{algorithm-step:binary-search-result-guess-number}
    If there is only one integer $x$ such that $\ell < x \le u$, Bob says that
    Alice's number is $x$ and terminates the algorithm.
  \item \label{algorithm-step:binary-search-decision-guess-number}
    Bob asks whether the Alice's number is at most $(\ell + u) / 2$. If the
    answer is yes, then Bob replaces $u$ by $(\ell + u) / 2$; otherwise Bob
    replaces $\ell$ by $(\ell + u) / 2$.
    Bob goes to step~\ref{algorithm-step:binary-search-decision-guess-number}.
\end{enumerate}

We need to prove that Bob's algorithm is correct and that it makes at most $10$
questions. We prove a bit stronger statement. If on
step~\ref{algorithm-step:binary-search-decision-guess-number} $u - \ell < n$ and
Alice's number is between $\ell$ and $u$ ($u$ and $\ell$ are some reals), then
the algorithm returns Alice's number using at most $\log_2 n$ questions 
($\log_2 1000 \le 10$). We are going to use strong induction by $n$. 

The base case is clear since if $u - \ell < 1$ there is at most one integer
between $u$ and $\ell$ and by the assumption there is Alice's number between
$\ell$ and $u$. Hence, Bob is going to guess Alice's number correctly without asking
any questions.

Assume that the statement is true for all $m < n$. Note that 
$\ell < (\ell + u) / 2 < u$ since $n \ge 2$. Therefore Alice's number is either
between $\ell$ and $(\ell + u) / 2$ or between $(\ell + u) / 2$ and $u$. Hence,
we go to step~\ref{algorithm-step:binary-search-decision-guess-number} with new
$\ell'$ and $u'$ such that $u' - \ell' < n / 2$ and Alice's number is between
$\ell'$ and $u'$. As a result, by the induction hypothesis, Bob is going to
guess Alice's number correctly using at most $1 + \log_2 (n / 2) = \log_2 n$
questions.

\begin{chapterendexercises}
    \exercise Let $a_0 = 2$, $a_1 = 5$, and $a_n = 5a_{n - 1} - 6 a_{n - 2}$
        for all integers $n \ge 2$. Show that $a_n = 3^n + 2^n$ for all integers
        $n \ge 0$.
        \begin{solution}
            We prove this using induction by $n$. The base case for $n \le 1$ is clear
            since $3^0 + 2^0 = 2$ and $3^1 + 2^1 = 5$.

            Let us prove the induction step. Assume that $a_n = 3^n + 2^n$ and
            $a_{n - 1} = 3^{n - 1} + 2^{n - 1}$, we need to prove that
            $a_{n + 1} = 3^{n + 1} + 2^{n + 1}$. Note that
            \begin{multline*}
                a_{n + 1} = 5a_n - 6 a_{n - 1} =
                5 \cdot 3^n + 5 \cdot 2^n - 6 \cdot 3^{n - 1} -
                6 \cdot 2^{n - 1} = \\
                3^{n - 1} \cdot 9 + 2^{n - 1} 4 = 3^{n + 1} + 2^{n + 1}.
            \end{multline*}
        \end{solution}
    \exercise Let $f_0 = 1$, $f_1 = 1$, and $f_{n + 2} = f_{n + 1} + f_n$ for
        all integers $n \ge 0$. Show that
        $f_n \ge \left(\frac{3}{2}\right)^{n - 2}$.
    \exercise Show that $f_{n + m} = f_{n - 1} f_{m - 1} + f_n f_m$.
    \exercise Give a nonadaptive algorithm for Bob that allows him to guess the
      number using $10$ queries. In other words, write $10$ questions such that
      answers to these questions allow Bob to guess the number.
    \exercise Show that \Cref{algorithm:binary-search} makes at most
      $6 + 2\log_2(n)$ comparisons.
      \begin{algorithm}
        \begin{algorithmic}[1]
          \Function{BinarySearch}{$e$, $a_1$, \dots, $a_n$}
            \If{$n \le 5$}
              \For{$i$ from $1$ to $n$}
                \If{$a_i = e$}
                  \State\Return{i}
                 \EndIf
              \EndFor
            \Else
              \State{$\ell \gets \floor{\frac{n}{2}}$}
              \If{$a_\ell \le e$}
                \State\Call{BinarySearch}{$e$, $a_1$, \dots, $a_\ell$}
              \Else
                \State\Call{BinarySearch}{$e$, $a_{\ell + 1}$, \dots, $a_n$}
              \EndIf
            \EndIf
          \EndFunction
        \end{algorithmic}
        \caption{The binary search algorithm that finds an element $e$ in the sorted
          list $a_1$, \dots, $a_n$.}
        \label{algorithm:binary-search}
      \end{algorithm}
\end{chapterendexercises}
